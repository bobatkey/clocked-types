\documentclass[natbib]{sigplanconf}

\usepackage[usenames]{color}
\definecolor{citationcolour}{rgb}{0,0.4,0.2}
\definecolor{linkcolour}{rgb}{0,0,0.8}
\usepackage{hyperref}
\hypersetup{colorlinks=true,
            urlcolor=linkcolour,
            linkcolor=linkcolour,
            citecolor=citationcolour,
            pdftitle=Productive Coprogramming with Guarded Recursion,
            pdfauthor={Robert Atkey, Conor McBride},
            pdfkeywords={}}  
\def\sectionautorefname{Section}
\def\subsectionautorefname{Section}

\title{Productive Coprogramming with Guarded Recursion}

\authorinfo{Robert Atkey}
           { }
           {bob.atkey@gmail.com}

\authorinfo{Conor McBride}
           {University of Strathclyde}
           {Conor.McBride@strath.ac.uk}

\usepackage{mathpartir}
\usepackage{amsmath}
\usepackage{amssymb}
\usepackage{amsthm}
\usepackage{stmaryrd}

% Colouring in...
\definecolor{highlight}{rgb}{0.8,0.1,0.1}
\definecolor{cons}{rgb}{0.8,0.1,0.2}
\definecolor{elim}{rgb}{0.1,0.5,0.35}
\definecolor{defnd}{rgb}{0.1,0.2,0.8}
\definecolor{greybg}{rgb}{0.8,0.8,0.8}
\newcommand{\highlight}[1]{\textcolor{highlight}{#1}}
\newcommand{\cons}[1]{\textcolor{cons}{\textsf{#1}}}
\newcommand{\elim}[1]{\textcolor{elim}{\textsf{#1}}}
\newcommand{\defnd}[1]{\mathit{#1}}
\newcommand{\kw}[1]{\textbf{#1}}
\newcommand{\tyname}[1]{\textrm{#1}}
\newcommand{\ident}[1]{\textit{#1}}
\newcommand{\defn}[1]{\textcolor{defnd}{\ident{#1}}}

% other stuff
\newcommand{\sepbar}{\mathrel|}
\newcommand{\Set}{\mathsf{Set}}
\newcommand{\sem}[1]{\llbracket #1 \rrbracket}
\newcommand{\op}{\mathsf{op}}
\newcommand{\sortType}{\texttt{\textup{type}}}

\newcommand{\delay}[1]{\mathord{\rhd\kern-0.4em^{#1}}}
\newcommand{\delayX}[1]{\mathord{\rhd\kern-0.1em^{#1}}}

\newcommand{\clkSem}[1]{\llbracket #1 \rrbracket}
\newcommand{\ClkPER}{\mathrm{ClkPER}}
\newcommand{\PER}{\mathrm{PER}}
\newcommand{\semCons}[1]{\mathsf{#1}}
\newcommand{\tySem}[1]{\llbracket #1 \rrbracket}

\newtheorem{lemma}{Lemma}
\newtheorem{theorem}{Theorem}
\newtheorem{corollary}{Corollary}
\newtheorem{definition}{Definition}

\newcommand{\lemref}[1]{\hyperref[#1]{Lemma \ref*{#1}}}
\newcommand{\thmref}[1]{\hyperref[#1]{Theorem \ref*{#1}}}

\newcommand{\etal}{\textit{et} \textit{al}.}

\begin{document}

\exclusivelicense
\conferenceinfo{ICFP~'13}{September 25--27, 2013, Boston, MA, USA}
\copyrightyear{2013}
\copyrightdata{978-1-4503-2326-0/13/09}
\doi{2500365.2500597}

\maketitle

\begin{abstract}
  Total functional programming offers the beguiling vision that, just
  by virtue of the compiler accepting a program, we are guaranteed
  that it will always terminate. In the case of programs that are not
  intended to terminate, e.g., servers, we are guaranteed that
  programs will always be \emph{productive}. Productivity means that,
  even if a program generates an infinite amount of data, each piece
  will be generated in finite time. The theoretical underpinning for
  productive programming with infinite output is provided by the
  category theoretic notion of final coalgebras. Hence, we speak of
  \emph{co}programming with non-well-founded \emph{co}data, as a dual
  to programming with well-founded data like finite lists and trees.

  Systems that offer facilities for productive coprogramming, such as
  the proof assistants Coq and Agda, currently do so through syntactic
  guardedness checkers. Syntactic guardedness checkers ensure that all
  self-recursive calls are guarded by a use of a constructor. Such a
  check ensures productivity. Unfortunately, these syntactic checks
  are not compositional, and severely complicate coprogramming.

  Guarded recursion, originally due to Nakano, is tantalising as a
  basis for a flexible and compositional type-based approach to
  coprogramming. However, as we show, by itself, guarded recursion is
  not suitable for coprogramming due to the fact that there is no way
  to make finite observations on pieces of infinite data. In this
  paper, we introduce the concept of \emph{clock variables} that index
  Nakano's guarded recursion. Clock variables allow us to ``close
  over'' the generation of infinite data, and to make finite
  observations, something that is not possible with guarded recursion
  alone.
\end{abstract}

\category{D.1.1}{Programming techniques}{Applicative (functional)
  programming} \category{D.2.4}{Software Engineering}{Software/Program
  Verification} \category{D.3.3}{Programming Languages}{Language
  Constructs and Features---Data types and structures}

\terms
  Languages, Theory, Types, Recursion

\keywords
  coalgebras, corecursion, guarded recursion, total functional programming

\section{Introduction}
\label{sec:introduction}

Coprogramming refers to the practice of explicitly manipulating
codata, the non-well-founded dual of well-founded data like finite
lists and trees. Codata are useful for representing the output of an
infinite process as a never ending stream, or for representing
potentially infinite search trees of possibilities. The natural way to
express codata is as a recursively defined coprogram. However, for
recursively defined coprograms to be safely used in a total setting,
the system must ensure that all recursive definitions are
productive. The state of the art for productivity checking is
currently either awkward syntactic guardedness checkers, or more
flexible sized type systems that explicitly label everything with size
information.

In this paper, we investigate the use of \emph{guarded recursion} (due
to Hiroshi Nakano \cite{nakano00modality}) as a lightweight typing
discipline for enabling productive coprogramming. As we show in this
introduction, guarded recursion by itself is not suitable for
coprogramming, due to the strict segmentation of time inherent in
putting guardedness information in types. We therefore introduce the
concept of \emph{clock variables} to take us from the
finite-time-sliced world of guarded recursion to the infinite world of
codata.

The contributions we make in this paper are the following:
\begin{enumerate}
\item We define a core type system for comfortable productive
  coprogramming, combining three key features: initial algebras,
  Nakano's guarded recursion and our main conceptual contribution:
  quantification over clock variables. We show that by combining the
  three key features of our system, we obtain a system for programming
  with \emph{final coalgebras}, the category theoretic description of
  codata. The combination of guarded recursion and clock
  quantification allows us to dispense with the clunky syntactic
  guardedness checks, and move to a flexible, compositional and
  local type-based system for ensuring guardedness.
\item We define a domain-theoretic denotational model that effectively
  interprets our system in the untyped lazy $\lambda$-calculus. We use
  a multiply-step-indexed model of types to prove that all well-typed
  programs are either terminating or productive and to show the
  correctness of our reconstruction of final coalgebras.
\item A side benefit of the combination of guarded recursion and clock
  quantification is that, due to the careful tracking of what data is
  available when through guarded types, we are able to accurately
  type, and show safe, strange circular functional programs like
  Bird's replaceMin example.
\end{enumerate}

Despite the large amount of recent work on guarded recursion (we
provide references throughout this introduction), this is the first
paper that formally links guarded recursion to productive
coprogramming. As recently noted by Birkedal and
M{\o}gelberg~\cite{birkedal13intensional}, ``guarded recursive types
can be used to [sic] as a crutch for higher-order programming with
coinductive types''. In this paper, we show that this is indeed
possible.

\subsection{Guardedness checkers and guarded recursion}

Several systems with support for corecursion, such as Coq (as
described by Gim{\'e}nez \cite{gimenez94codifying}) and Agda (as
described by Danielsson and Altenkirch \cite{danielsson09mixing}),
make use of syntactic guardedness checks to ensure productivity of
programs defined using corecursion. Without these checks, the
soundness of these systems cannot be guaranteed. An unfortunate aspect
of these guardedness checks is their non-compositionality. We
demonstrate the problem with an example, and see how Nakano's guarded
recursion can be used to provide a compositional type-based
guardedness check.

We will use Haskell notation for each of our informal examples, since
it serves to illustrate the issues concisely. Consider the following
Haskell declaration of a type of infinite streams of integers:
\begin{displaymath}
  \kw{data}\ \tyname{Stream} = \cons{StreamCons}\ \tyname{Integer}\ \tyname{Stream}
\end{displaymath}
% FIXME: something about (co)data coincidence

An example of a $\tyname{Stream}$ is the infinite stream of $1$s:
\begin{displaymath}
  \begin{array}{l}
    \defn{ones} :: \tyname{Stream} \\
    \defn{ones} = \cons{StreamCons}\ 1\ \ident{ones}
  \end{array}
\end{displaymath}
It is easy to see that this recursive definition is \emph{guarded};
$\ident{ones}$ is only invoked recursively within an application of
the stream constructor $\cons{StreamCons}$. This definition of
$\ident{ones}$ defines an infinite data structure, but each piece of
the data structure is delivered to us in finite time. Hence,
$\ident{ones}$ is productive.

An example of a non-guarded definition is the filter function,
extended from finite lists to infinite streams:
\begin{displaymath}
  \begin{array}{l}
    \defn{filter} :: (\tyname{Integer} \to \tyname{Bool}) \to \tyname{Stream} \to \tyname{Stream} \\
    \defn{filter}\ \ident{f}\ (\cons{StreamCons}\ \ident{z}\ \ident{s}) = \\
    \hspace{0.5cm}\kw{if }\ident{f}\ \ident{z}\kw{ then }\cons{StreamCons}\ \ident{z}\ (\ident{filter}\ \ident{f}\ \ident{s})\kw{ else }\ident{filter}\ \ident{f}\ \ident{s}
  \end{array}
\end{displaymath}
This definition is not guarded: in the $\kw{then}$-case of the
conditional, the recursive call to $\ident{filter}$ is not within an
application of the stream constructor $\cons{StreamCons}$. A syntactic
guardedness checker is right to reject this definition: consider the
case when all of the elements of the stream are filtered out;
$\ident{filter}$ will never return anything, and so will be
non-productive.

Syntactic guardedness checkers are not always so helpful. The
following higher-order function defines a general merge function on
pairs of streams:
\begin{displaymath}
  \begin{array}{l}
    \defn{mergef} :: (\tyname{Integer} \to \tyname{Integer} \to \tyname{Stream} \to \tyname{Stream}) \to \\
    \textcolor{white}{\textrm{mergef} :: }\tyname{Stream} \to \tyname{Stream} \to \tyname{Stream} \\
    \defn{mergef}\ \ident{f}\ (\cons{StreamCons}\ \ident{x}\ \ident{xs})\ (\cons{StreamCons}\ \ident{y}\ \ident{ys}) = \\
    \quad\quad \ident{f}\ \ident{x}\ \ident{y}\ (\ident{mergef}\ \ident{f}\ \ident{xs}\ \ident{ys})
  \end{array}
\end{displaymath}
Any syntactic checker looking for constructors to guard recursive
calls will reject this function: there are no constructors anywhere in
the definition! This rejection is with good reason: there are
functions that we could pass to $\ident{mergef}$ that would render it
unproductive. For example, this function will cause $\ident{mergef}$
to hang on all pairs of streams:
\begin{displaymath}
  \begin{array}{l}
    \defn{badf} :: \tyname{Integer} \to \tyname{Integer} \to \tyname{Stream} \to \tyname{Stream} \\
    \defn{badf}\ \ident{x}\ \ident{y}\ \ident{s} = \ident{s}
  \end{array}
\end{displaymath}
On the other hand, there are plenty of good functions $\ident{f}$ that
we could use with $\ident{mergef}$ to obtain productive functions, but
a syntactic guardedness checker does not allow us to express this
fact.

A possible way out of this problem is to retain the syntactic
guardedness check, and work around it by changing the type of
$\ident{f}$. For instance, we could change the required type of
$\ident{f}$ to:
\begin{displaymath}
  \ident{f} :: \tyname{Integer} \to \tyname{Integer} \to \tyname{Integer}
\end{displaymath}
to allow for merging functions that operate element-wise. Or we could
change the type to
\begin{displaymath}
  \ident{f} :: \tyname{Integer} \to \tyname{Integer} \to (\tyname{Integer}, [\tyname{Integer}])
\end{displaymath}
which would allow for the functional argument to replace each pair of
elements from the input streams with a non-empty list of values. The
general trick is to make $\ident{f}$ return ``instructions'' on how to
transform the stream back to $\ident{mergef}$, which then executes
them in a way that the syntactic guardedness checker can see is
guarded. This technique has been elaborated in detail by Danielsson
\cite{danielsson10beating}. However, it is difficult to see whether we
could capture all the possibilities for good $\ident{f}$s in a single
type. We could give yet another type which would accommodate the
behaviour of the following possible value of $\ident{f}$:
\begin{displaymath}
  \begin{array}{l}
    \defn{f}\ \ident{x}\ \ident{y}\ \ident{s} = \cons{StreamCons}\ \ident{x}\ (\ident{map}\ (+\ident{y})\ \ident{s})
  \end{array}
\end{displaymath}
Incorporating the forms of all the possible productive definitions
into a single type seems impractical. Moreover, we complicate the
definition of $\ident{mergef}$, which is now forced to become
essentially an implementation of a virtual machine for running little
stream transformer programs returned by its functional argument,
rather than the simple one line definition we gave above.

A more promising type-based solution is provided by Nakano
\cite{nakano00modality}. Nakano introduces a guardedness type
constructor that represents values that may only be used in a guarded
way. Thus Nakano transports the guardedness checks from the syntactic
level into the type system. We write this constructor, applied to a
type $A$, as $\rhd A$ (Nakano uses the notation $\bullet A$, but the
$\rhd A$ notation has become more common, and conveys an intuitive
idea of displacement in time). A useful way to think of $\rhd A$ is as a
value of $A$ that is only available ``tomorrow'', and the temporal gap
between today and tomorrow can only be bridged by the application of a
constructor such as $\cons{StreamCons}$. The reading of $\rhd A$ as a
value of $A$ tomorrow is explicitly supported in the step-indexed
semantics we present in \autoref{sec:semantics} by means of a clock
counting down to zero.

We now alter the type of $\ident{f}$ to be:
\begin{displaymath}
  \ident{f} :: \tyname{Integer} \to \tyname{Integer} \to \rhd\tyname{Stream} \to \tyname{Stream}
\end{displaymath}
This type captures the intuition that the stream argument to
$\ident{f}$ must only be used in a guarded way. To introduce
guarded types into the system, we must make some changes to the types
of our primitives. We alter the type of the constructor
$\cons{StreamCons}$ as follows, and also introduce a stream
deconstructor that we previously implicitly used via pattern matching:
\begin{displaymath}
  \begin{array}{l}
    \cons{StreamCons} :: \tyname{Integer} \to \rhd\tyname{Stream} \to \tyname{Stream} \\
    \elim{deStreamCons} :: \tyname{Stream} \to (\tyname{Integer}, \rhd\tyname{Stream})
  \end{array}
\end{displaymath}
The type of $\cons{StreamCons}$ shows that it takes a guarded stream
``tomorrow'', and produces a non-guarded stream ``today''. This is in
line with the temporal intuition we have of the type
$\rhd\tyname{Stream}$. The inverse operation $\elim{deStreamCons}$
takes a full stream and returns the integer and the \emph{guarded}
remainder of the stream.

To give the guardedness type constructor force, Nakano proposes the
following alternative type of the fixpoint operator for defining
recursive values:
\begin{displaymath}
  \kw{fix} :: (\rhd A \to A) \to A
\end{displaymath}
(The standard typing of the $\kw{fix}$ operator is $(A \to A) \to
A$.) Nakano's alternative typing ensures that recursive definitions
must be guarded by only allowing access to recursively generated
values ``tomorrow''.

To actually program with the guardedness constructor, we equip it with
the structure of an applicative functor \cite{mcbride08applicative}:
\begin{displaymath}
  \begin{array}{l}
    \kw{pure} :: A \to \rhd A \\
    \kw{$(\circledast)$} :: \rhd (A \to B) \to \rhd A \to \rhd B
  \end{array}
\end{displaymath}
Note that $\rhd$ is not a monad: a join operation of type $\rhd(\rhd
A) \to \rhd A$ would allow us to collapse multiple guardedness
obligations.

This method for integrating Nakano's guardedness type constructor into
a typed $\lambda$-calculus is not the only one.  Nakano uses a system
of subtyping with respect to the guardedness type constructor that has
a similar effect to assuming that $\rhd$ is an applicative functor. We
propose to treat $\rhd$ as an applicative functor here, due to the greater
ease with which it can be incorporated into a standard functional
programming language like Haskell. Krishnaswami and Benton
\cite{krishnaswami11semantic,krishnaswami11ultrametric} have presented
an alternative method that annotates members of the typing context
with the level of guardedness that applies to them. Severi and de
Vries \cite{severi12pure} have generalised Krishnaswami and Benton's
approach for any typed $\lambda$-calculus derived from a Pure Type
System (PTS), including dependently typed systems. These calculi have
nice proof theoretic properties, such as being able to state
productivity in terms of infinitary strong normalisation.

With the guardedness type constructor, and its applicative functor
structure, we can rewrite the $\ident{mergef}$ function to allow the
alternative typing which only allows functional arguments that will
lead to productive definitions.
\begin{displaymath}
  \begin{array}{l}
    \defn{mergef} :: (\tyname{Integer} \to \tyname{Integer} \to \rhd\tyname{Stream} \to \tyname{Stream}) \to \\
    \textcolor{white}{\textrm{mergef} :: }\ \tyname{Stream} \to \tyname{Stream} \to \tyname{Stream} \\
    \defn{mergef}\ \ident{f} = \kw{fix}\ (\lambda \ident{g}\ \ident{xs}\ \ident{ys}. \\
    \hspace{2.5cm} \kw{let}\ (\ident{x}, \ident{xs$'$}) = \elim{deStreamCons}\ \ident{xs} \\
    \hspace{2.5cm} \textcolor{white}{\kw{let}\ }(\ident{y}, \ident{ys$'$}) = \elim{deStreamCons}\ \ident{ys} \\
    \hspace{2.5cm} \kw{in}\ \ident{f}\ \ident{x}\ \ident{y}\ (\ident{g} \circledast \ident{xs$'$} \circledast \ident{ys$'$}))
  \end{array}
\end{displaymath}
Aside from the explicit uses of $\kw{fix}$ and $\elim{deStreamCons}$,
which were hidden by syntactic sugar in our previous definition, the
only substantial changes to the definition are the uses of applicative
functor apply ($\circledast$). These uses indicate operations that are
occurring under a guardedness type constructor.

\subsection{From the infinite to the finite}
\label{sec:infinite-to-finite}

The type system for guarded recursion that we described above allowed
us to remove the syntactic guardedness check and replace it with a
compositional type-based one. However, aside from proposing
applicative functor structure as a convenient way to incorporate the
guardedness type constructor into a functional language, we have not
gone much beyond Nakano's original system. We now describe a problem
with guarded recursion when attempting to combine infinite and finite
data. We propose a solution to this problem in the
\hyperref[sec:clock-vars]{next subsection}.

Consider the $\ident{take}$ function that reads a finite prefix of an
infinite stream into a list. This function will have the following type:
\begin{displaymath}
  \defn{take} :: \tyname{Natural} \to \tyname{Stream} \to [\tyname{Integer}]
\end{displaymath}
where we wish to regard the type $[\tyname{Integer}]$ as the type of
\emph{finite} lists of integers. The explicit segregation of
well-founded and non-well-founded types is important for our intended
applications of total functional programming and theorem proving.

We also assume that the type $\tyname{Natural}$ of natural numbers is
well-founded, so we may attempt to write $\ident{take}$ by structural
recursion on its first argument. However, we run into a difficulty,
which we have \colorbox{greybg}{highlighted}.
\begin{displaymath}
  \begin{array}{l}
    \defn{take} :: \tyname{Natural} \to \tyname{Stream} \to [\tyname{Integer}] \\
    \defn{take}\ 0\ \ident{s} = [] \\
    \defn{take}\ (\ident{n}+1)\ \ident{s} = \ident{x} \mathop{\cons{:}} \ident{take}\ \ident{n}\ \colorbox{greybg}{$\ident{s$'$}$} \\
    \quad \kw{where}\ (\ident{x}, \ident{s$'$}) = \elim{deStreamCons}\ \ident{s}
  \end{array}
\end{displaymath}
The problem is that the variable $\ident{s$'$}$, which we have obtained
from $\elim{deStreamCons}$, has type $\rhd\tyname{Stream}$. However,
to invoke the $\ident{take}$ function structurally recursively, we
need something of type $\tyname{Stream}$, without the guardedness
restriction.

We analyse this problem in terms of our intuitive reading of
$\rhd\tyname{Stream}$ as a stream that is available
``tomorrow''. Nakano's typing discipline slices the construction of
infinite data like $\tyname{Stream}$ into discrete steps. In contrast,
a well-founded data type like $[\tyname{Integer}]$ lives entirely in
the moment. While a stream is being constructed, this slicing is
required so that we do not get ahead of ourselves and attempt to build
things today from things that will only be available tomorrow. But
once a stream has been fully constructed, we ought to be able to blur
the distinctions between the days of its construction. We accomplish
this in our type system by means of \emph{clock variables}, and
quantification over them.

\subsection{Clock variables}
\label{sec:clock-vars}

We extend the type system with \emph{clock variables} $\kappa$. A
clock variable $\kappa$ represents an individual time sequence that
can be used for safe construction of infinite data like streams. In
our model, clock variables are interpreted as counters running down to
zero in the style of step-indexed models. By quantifying over all
counters we are able to accommodate all possible finite observations
on an infinite data structure.

We annotate the guardedness type constructor with a clock variable to
indicate which time stream we are considering: $\delay\kappa
A$. Infinite data types such as streams are now annotated by their
clock variable, indicating the time stream that they are being
constructed on. We have the following new types for the constructor
and deconstructor:
\begin{displaymath}
  \begin{array}{l}
    \cons{StreamCons$^\kappa$} :: \tyname{Integer} \to \delay\kappa \tyname{Stream}^\kappa \to \tyname{Stream}^\kappa \\
    \elim{deStreamCons$^\kappa$} :: \tyname{Stream}^\kappa \to (\tyname{Integer}, \delay\kappa\tyname{Stream}^\kappa)    
  \end{array}
\end{displaymath}
We now regard the type $\tyname{Stream}^\kappa$ as the type of
infinite streams in the process of construction. A finished infinite
stream is represented by \emph{quantifying} over the relevant clock
variable: $\forall \kappa. \tyname{Stream}^\kappa$. If we think of
each possible downward counting counter that $\kappa$ could represent,
then this universally quantified type allows for any counter large
enough to allow us any finite extraction of data from the stream. We
use the notation $\Lambda\kappa. e$ and $e[\kappa]$ to indicate clock
abstraction and application respectively, in the style of explicitly
typed System F type abstraction and application.

The fixpoint operator is parameterised by the clock variable:
\begin{displaymath}
  \kw{fix} :: \forall \kappa. (\delay\kappa A \to A) \to A
\end{displaymath}
as is the applicative functor structure carried by $\delay\kappa$:
\begin{displaymath}
  \begin{array}{l}
    \kw{pure} :: \forall \kappa. A \to \delay\kappa A \\
    \kw{$(\circledast)$} :: \forall \kappa. \delay\kappa (A \to B) \to \delay\kappa A \to \delay\kappa B
  \end{array}
\end{displaymath}
We also add an additional piece of structure to eliminate guarded
types that have been universally quantified:
\begin{displaymath}
  \kw{force} :: (\forall \kappa.\ \delay\kappa A) \to (\forall \kappa.\ A)
\end{displaymath}
Intuitively, if we have a value that, for any time stream, is one time
step away, we simply instantiate it with a time stream that has at
least one step left to run and extract the value. Note that the
universal quantification is required on the right hand side since
$\kappa$ may appear free in the type $A$. The $\kw{force}$ operator
has a similar flavour to the $\kw{runST} :: (\forall s. \tyname{ST}\
s\ a) \to a$ for the $\tyname{ST}$ monad
\cite{DBLP:journals/jfp/MoggiS01,launchbury94lazy}. In both cases,
rank-2 quantification is used to prevent values from ``leaking'' out
from the context in which it is safe to use them.

Using $\kw{force}$ we may write a deconstructor for completed streams
of type $\forall \kappa. \tyname{Stream}^\kappa$.
\begin{displaymath}
  \begin{array}{l}
    \elim{deStreamCons} :: (\forall \kappa. \tyname{Stream}^\kappa) \to (\tyname{Integer}, \forall \kappa. \tyname{Stream}^\kappa) \\
    \elim{deStreamCons}\ \ident{x} = \\
    \quad\quad
    \begin{array}[t]{l}
      (\Lambda \kappa. \kw{fst}\ (\elim{deStreamCons}^\kappa\ (\ident{x}[\kappa])),\\
      \quad\kw{force}\ (\Lambda \kappa. \kw{snd}\ (\elim{deStreamCons}^\kappa\ (\ident{x}[\kappa]))))
    \end{array}
  \end{array}
\end{displaymath}
In this definition we have made use of a feature of our type system
that states that the type equality $\forall \kappa. A \equiv A$ holds
whenever $\kappa$ is not free in $A$. Our system also includes other
type equalities that demonstrate how clock quantification interacts
with other type formers. These are presented in
\autoref{sec:type-equality}.

In the presence of clock variables, our definition of $\ident{take}$
is well-typed and we get the function we desire: taking a finite
observation of a piece of infinite data:
\begin{displaymath}
  \begin{array}{l}
    \defn{take} :: \tyname{Natural} \to (\forall \kappa. \tyname{Stream}^\kappa) \to [\tyname{Integer}] \\
    \defn{take}\ 0\ \ident{s} = [] \\
    \defn{take}\ (\ident{n}+1)\ \ident{s} = \ident{x} \mathop{\cons{:}} \ident{take}\ \ident{n}\ \ident{s$'$} \\
    \quad \kw{where}\ (\ident{x}, \ident{s$'$}) = \elim{deStreamCons}\ \ident{s}
  \end{array}
\end{displaymath}

Clock variables also allow us to make clear in the types fine
distinctions between functions that are extensionally equal, but
differ in their productivity. In plain Haskell, the following two
stream mapping functions have the same behaviour on any
\emph{completely constructed} stream. The first $\ident{map}$
processes each element one at a time
\begin{displaymath}
  \begin{array}{l}
    \defn{map}~f~s = \cons{StreamCons}~(f~x)~(\ident{map}~f~s') \\
    \quad \kw{where}~(\ident{x}, \ident{s$'$}) = \elim{deStreamCons}~s
  \end{array}
\end{displaymath}
while $\ident{maap}$ processes elements two at a time by pattern matching:
%\begin{displaymath}
%  \begin{array}{@{}l}
%    \defn{maap}~f~s = \cons{StreamCons}~(f~x)~(\cons{StreamCons}~(f~y)~(\ident{maap}~f~s'')) \\
%    \quad \kw{where}~(\ident{x}, \ident{s$'$}) = \elim{deStreamCons}~s;~(\ident{y}, \ident{s$''$}) = \elim{deStreamCons}~s'
%  \end{array}
%\end{displaymath}
\begin{displaymath}
  \begin{array}{@{}l}
    \defn{maap}~f~(\cons{StreamCons}~x~(\cons{StreamCons}~y~s'')) \\
    \quad = \cons{StreamCons}~(f~x)~(\cons{StreamCons}~(f~y)~(\ident{maap}~f~s''))
  \end{array}
\end{displaymath}
If all the elements of the input are available at once, then
$\ident{map}$ and $\ident{maap}$ have the same behaviour. However, if
these functions are passed streams that are still being constructed,
then their behaviour can differ significantly. Consider these two
definitions of a stream of natural numbers, using $\ident{map}$ and $\ident{maap}$:
\begin{displaymath}
  \begin{array}{l}
    \defn{nats} = \cons{StreamCons}~0~(\ident{map}~(\lambda x.~x+1)~\ident{nats})\\
    \defn{badnats} = \cons{StreamCons}~0~(\ident{maap}~(\lambda x.~x+1)~\ident{badnats})
  \end{array}
\end{displaymath}
Running $\ident{nats}$ produces the infinite stream of natural
numbers, as expected, but running $\ident{badnats}$ produces
nothing. The function $\ident{maap}$ expects two elements to be
available on its input stream, while $\ident{badnats}$ has only
provided one.

Even though $\ident{map}$ and $\ident{maap}$ are extensionally equal,
they have different behaviour on partially constructed streams. We can
state this different behaviour using the following types in our system:
\begin{displaymath}
  \begin{array}{l}
    \defn{map} :: \forall \kappa. (\tyname{Integer} \to \tyname{Integer}) \to \tyname{Stream}^\kappa \to \tyname{Stream}^\kappa \\
    \defn{maap} :: (\tyname{Integer} \to \tyname{Integer}) \to (\forall \kappa. \tyname{Stream}^\kappa) \to (\forall \kappa. \tyname{Stream}^\kappa)
  \end{array}
\end{displaymath}
Thus, $\ident{map}$'s type states that the input stream and output
stream run on the same clock $\kappa$, so each element of the output
will only require one element of the input. This is exactly what
$\ident{nats}$ requires, and our type system will accept the
definition of $\ident{nats}$. In contrast, $\ident{maap}$'s type
states that it requires a fully constructed stream as input. The
definition of $\ident{badnats}$ does not provide this, and so our type
system will reject it.

By using clock quantification to ``close over'' the guardedness, our
approach thus \emph{localises} the discipline which makes the
productivity of definitions plain, without affecting the types at
which the defined objects are used. Abel's \emph{sized types} impose a
similar discipline, but globally~\cite{abel04termination}.

\subsection{Final coalgebras from initial algebras, guards and clocks}
\label{sec:main-results-intro}

In the previous subsection, we informally stated that the type
$\tyname{Stream} = \forall \kappa. \tyname{Stream}^\kappa$ represents
exactly the type of infinite streams of integers. We make this
informal claim precise by using the category theoretic notion of final
coalgebra. A key feature of our approach is that we decompose the
notion of final coalgebra into three parts: initial algebras,
Nakano-style guarded types, and clock quantification.

\paragraph{Initial algebras}
In \autoref{sec:type-system} we present a type system that includes a
notion of strictly positive type operator. We include a least fixpoint
operator $\mu X. F[X]$. For normal strictly positive type operators
$F$ (i.e. ones that do not involve the $\delay\kappa$ operator), $\mu
X. F[X]$ is exactly the normal least fixpoint of $F$. Thus we can
define the normal well-founded types of natural numbers, lists, trees
and so on. Our calculus contains constructors $\cons{Cons$_F$} ::
F[\mu X. F] \to \mu X. F$ for building values of these types, and
recursion combinators for eliminating values:
\begin{displaymath}
  \elim{primRec$_{F,A}$} :: (F[(\mu X. F) \times A] \to A) \to \mu X. F[X] \to A 
\end{displaymath}
We opt to use primitive recursion instead of a fold operator
(i.e. $\ident{fold} :: (F[A] \to A) \to \mu X. F[X] \to A$) because
it allows for constant time access to the top-level of an inductive
data structure.

We show in the proof of \thmref{thm:initial-f-algebra} that $\mu X. F$
is the carrier of the initial $F$-algebra in our model. Therefore, we
may use standard reasoning techniques about initial algebras to reason
about programs written using the $\elim{primRec}$ combinator.

\paragraph{Guarded final coalgebras}
When we consider strictly positive type operators of the form
$F[\delay\kappa X]$, where $\kappa$ does not appear in $F$ then, in
addition to $\mu X. F[\delay\kappa X]$ being the least fixpoint of the
operator $F[\delay\kappa -]$, it is also the \emph{greatest}
fixpoint. This initial-final coincidence is familiar from domain
theoretic models of recursive types (see, e.g., Smyth and Plotkin
\cite{smyth82category}), and has previously been observed as a
feature of guarded recursive types by Birkedal
\etal~\cite{birkedal12first}.

The $\ident{unfold}$ combinator that witnesses $\mu X. F[\delay\kappa
X]$ as final can be implemented in terms of the $\kw{fix}$ operator,
and $\cons{Cons}$:
\begin{displaymath}
  \begin{array}[t]{l}
    \defn{unfold$_F$} :: \forall \kappa. (A \to F[\delay\kappa A]) \to A \to \mu X. F[\delay\kappa X] \\
    \defn{unfold$_F$} = \Lambda \kappa.\lambda \ident{f}. \kw{fix}\ (\lambda \ident{g}\ \ident{a}. \cons{Cons}\ (\ident{fmap}_F\ (\lambda \ident{x}. \ident{g} \circledast \ident{x})\ (\ident{f}\ \ident{a})))
  \end{array}
\end{displaymath}
where $\ident{fmap}_F :: (A \to B) \to (F[A] \to F[B])$ is the
functorial mapping combinator associated with the strictly positive
type operator $F$. Note that without the additional $\delay\kappa$ in
$F[\delay\kappa -]$, there would be no way to define $\ident{unfold}$,
due to the typing of $\kw{fix}$.

To make observations on elements, we define a $\ident{deCons$_F$}$
combinator for every $F$, using the primitive recursion
\begin{displaymath}
  \begin{array}{l}
    \defn{deCons$_F$} :: \forall \kappa. (\mu X. F[\delay\kappa X]) \to F[\delay\kappa \mu X. F[\delay\kappa X]] \\
    \defn{deCons$_F$} = \Lambda \kappa. \elim{primRec}\ (\lambda \ident{x}.\ \ident{fmap}_{F[\delayX\kappa-]}\ (\lambda \ident{y}.\kw{fst}\ \ident{y})\ \ident{x})
  \end{array}
\end{displaymath}
The proof of \thmref{thm:final-f-de-coalgebra} shows that these
definitions of $\ident{unfold$_F$}$ and $\ident{deCons$_F$}$ exhibit
$\mu X. F[\delay\kappa X]$ as the final $F[\delay\kappa-]$-coalgebra,
using $\ident{deCons$_F$}$ as the structure map. This means that
$\ident{unfold$_F$}[\kappa] f$ is the unique $F[\delay\kappa-]$-coalgebra
homomorphism from $A$ to $\mu X. F[\delay\kappa-]$.

\paragraph{Final coalgebras}
\thmref{thm:final-f-de-coalgebra} only gives us final coalgebras in
the case when the recursion in the type is guarded by the type
constructor $\delay\kappa-$. So we do not yet have a dual to the least
fixpoint type constructor $\mu X. F$. However, as we hinted above in
the case of streams, we can use clock quantification to construct a
final $F$-coalgebra, where $F$ need not contain an occurrence of
$\delay\kappa-$.

For the type $\forall \kappa. \mu X. F[\delay\kappa-]$ we define an
$\ident{unfold$^\nu_F$}$ operator, in a similar style to the
$\ident{unfold$_F$}$ operator above. However, this operator takes an
argument of type $(A \to F[A])$, with no mention of guardedness.
\begin{displaymath}
  \begin{array}{l}
    \defn{unfold$^\nu_F$} :: (A \to F[A]) \to A \to \forall \kappa. \mu X. F[\delay\kappa X] \\
    \defn{unfold$^\nu_F$} = \lambda \ident{f}\ \ident{a}. \Lambda \kappa. \\
    \quad\quad\kw{fix}\ (\lambda \ident{g}\ \ident{a}. \cons{Cons}_F\ (\ident{fmap}_F (\lambda \ident{x}. \ident{g} \circledast \kw{pure}\ \ident{x})\ (\ident{f}\ \ident{a})))\ \ident{a}
  \end{array}
\end{displaymath}
We can also define the corresponding deconstructor, building on the
definition of $\ident{deCons$_F$}$ above, but also using the $\kw{force}$
construct in conjunction with clock quantification. This is a
generalisation of the definition of $\elim{deStreamCons}$ above.
\begin{displaymath}
  \begin{array}{l}
    \defn{deCons$^\nu_F$} :: (\forall \kappa. \mu X. F[\delay\kappa X]) \to F[\forall \kappa. \mu X. F[\delay\kappa X]] \\
    \defn{deCons$^\nu_F$} = \lambda x.~\ident{fmap$_F$}\ (\lambda x. \kw{force}\ x)\ (\Lambda \kappa.~\ident{deCons$_F$}[\kappa]\ (x[\kappa]))
  \end{array}
\end{displaymath}
In the application of $\ident{fmap$_F$}$, we make use of the type
equalities in our calculus, which allow us to treat the types $\forall
\kappa. F[X]$ and $F[\forall \kappa. X]$ as equivalent, when $\kappa$
does not appear in $F$. We present the type equalities in
\autoref{sec:type-equality}, and prove them sound in
\autoref{sec:type-interp}.

Our last main technical result, \thmref{thm:final-f-coalgebra}, states
that, in the semantics we define \autoref{sec:semantics}, $\forall
\kappa. \mu X. F[\delay\kappa-]$ actually is the final
$F$-coalgebra. The use of clock quantification has allowed us to
remove mention of the guardedness type constructor, and given us a way
to safely program and reason about $F$-coalgebras.

\subsection{Circular traversals of trees}
\label{sec:repmin}

We now present a perhaps unexpected application of the use of clock
quantification that does not relate to ensuring productivity of
recursive programs generating infinite data. An interesting use of
laziness in functional programming languages is to perform single-pass
transformations of data structures that would appear to require at
least two passes. A classic example, due to Bird
\cite{bird84circular}, is the $\ident{replaceMin}$ function that
replaces each value in a binary tree with the minimum of all the
values in the tree, in a single pass. In normal Haskell, this function
is written as follows:
\begin{displaymath}
  \begin{array}{l}
    \defn{replaceMin} :: \tyname{Tree} \to \tyname{Tree} \\
    \defn{replaceMin}\ \ident{t} = \kw{let}\ (\ident{t$'$}, \ident{m}) = \ident{replaceMinBody}\ \ident{t}\ \ident{m}\ \kw{in}\ \ident{t$'$} \\
    \hspace{0.6cm} \kw{where} \\
    \hspace{1cm} \defn{replaceMinBody}\ (\cons{Leaf}\ \ident{x})\ \ident{m} = (\cons{Leaf}\ \ident{m}, \ident{x}) \\
    \hspace{1cm} \defn{replaceMinBody}\ (\cons{Br}\ \ident{l}\ \ident{r})\ \ident{m} = \\
    \hspace{2cm} \kw{let}\ (\ident{l$'$}, \ident{m$_l$}) = \ident{replaceMinBody}\ \ident{l}\ \ident{m} \\
    \hspace{2cm} \textcolor{white}{\kw{let}\ }(\ident{r$'$}, \ident{m$_r$}) = \ident{replaceMinBody}\ \ident{r}\ \ident{m} \\
    \hspace{2cm} \kw{in}\ (\cons{Br}\ \ident{l$'$}\ \ident{r$'$},\ \ident{min}\ \ident{m$_l$}\ \ident{m$_r$})
  \end{array}
\end{displaymath}
The interesting part of this function is the first $\kw{let}$
expression. This takes the minimum value $\ident{m}$ computed by the
traversal of the tree and passes it into the \emph{same} traversal to
build the new tree. This function is a marvel of declarative
programming: we have declared that we wish the tree $t'$ to be
labelled with the minimum value in the tree $t$ just by stating that
they be the same, without explaining at all why this definition makes
any sense. Intuitively, the reason that this works is that the overall
minimum value is never used in the computation of the minimum, only
the construction of the new tree. The computation of the minimum and
the construction of the new tree conceptually exist at different
moments in time, and it is only safe to treat them the same after both
have finished.

Using explicit clock variables we can give a version of the
$\ident{replaceMin}$ definition that demonstrates that we have
actually defined a total function from trees to trees. The use of
clock variables allows us to be explicit about the time when various
operations are taking place.

Out first step is to replace the circular use of $\kw{let}$ with a
\emph{feedback} combinator defined in terms of the $\kw{fix}$
operator:
\begin{displaymath}
  \begin{array}{l}
    \defn{feedback} : \forall \kappa.~(\delay\kappa U \to (B[\kappa], U)) \to B[\kappa] \\
    \defn{feedback} = \Lambda \kappa. \lambda f. \kw{fst}~(\kw{fix}~(\lambda x.~f~(\kw{pure}~(\lambda x. \kw{snd}~x) \circledast x)))
  \end{array}
\end{displaymath}
In the application of the $\kw{fix}$ operator, $x : \delay\kappa (B
\times U)$.  The notation $B[\kappa]$ indicates that $\kappa$ may
appear free in the type $B$.

% We have named this definition $\ident{trace}$ due to similarity of its
% type to the trace operator of traced symmetric monoidal categories
% \cite{something}. Trace operators provide an axiomatisaion of the
% notion of feedback operator that has many applications in denotational
% semantics. The key difference with our definition is that the first
% occurence of the ``feedback type'' $U$ is guarded by the guardedness
% type constructor. This will prevent us from writing definitions that
% get ahead of themselves and access data that is not yet ready.

\newcommand{\hlchangem}[1]{\colorbox{greybg}{$#1$}}
\newcommand{\hlchange}[1]{\colorbox{greybg}{#1}}

We now rewrite the $\ident{replaceMinBody}$ function so that we can
apply $\ident{feedback}$ to it. We have \hlchange{highlighted} the
changes from the previous definition.
\begin{displaymath}
  \begin{array}{@{}l}
    \defn{replaceMinBody} :: \tyname{Tree} \to \hlchangem{\forall \kappa.}\hlchangem{\delay\kappa}\tyname{Integer} \to (\hlchangem{\delay\kappa}\tyname{Tree},\tyname{Integer}) \\
    \hspace{0cm} \defn{replaceMinBody}\ (\cons{Leaf}\ \ident{x})\ \ident{m} = (\hlchangem{\kw{pure}}\cons{Leaf}\ \hlchangem{\circledast}\ident{m}, \ident{x}) \\
    \hspace{0cm} \defn{replaceMinBody}\ (\cons{Br}\ \ident{l}\ \ident{r})\ \ident{m} = \\
    \hspace{2cm} \kw{let}\ (\ident{l$'$}, \ident{m$_l$}) = \ident{replaceMinBody}\ \ident{l}\ \ident{m} \\
    \hspace{2cm} \textcolor{white}{\kw{let}\ }(\ident{r$'$}, \ident{m$_r$}) = \ident{replaceMinBody}\ \ident{r}\ \ident{m} \\
    \hspace{2cm} \kw{in}\ (\hlchangem{\kw{pure}}\cons{Br}\ \hlchangem{\circledast}\ident{l$'$}\ \hlchangem{\circledast}\ident{r$'$},\ \ident{min}\ \ident{m$_l$}\ \ident{m$_r$})
  \end{array}
\end{displaymath}
The changes required are minor: we must change the type, and use the
applicative functor structure of $\delay\kappa$ to indicate when
computations are taking place ``tomorrow''.

Applying $\ident{feedback}$ to $\ident{replaceMinBody}$, and using
$\kw{force}$ to remove the now redundant occurrence of $\delay\kappa$,
we can define the full $\ident{replaceMin}$ function in our system:
\begin{displaymath}
  \begin{array}{l}
    \defn{replaceMin} :: \tyname{Tree} \to \tyname{Tree} \\
    \defn{replaceMin}\ \ident{t} = \kw{force}\ (\Lambda \kappa. \ident{feedback}[\kappa]\ (\ident{replaceMinBody}[\kappa])\ \ident{t})
  \end{array}
\end{displaymath}
By the soundness property of our system that we prove in
\autoref{sec:semantics}, we are assured that we have defined a total
function from trees to trees. The standard Haskell type system does
not provide this guarantee.

\subsection{Models of guarded recursion}

To substantiate the claims we have made in this introduction, in
\autoref{sec:semantics} we construct a multiply-step-indexed model of
the type system we present in the next section. The multiple
step-indexes are required for the multiple clock variables in our
system, each representing separate time streams. Step-indexed models
were introduced by Appel and McAllester~\cite{appel01indexed} to prove
properties of recursive types. The use of step-indexing serves to
break the circularity inherent in recursive types. Dreyer
\etal~\cite{dreyer11logical} exposed the connection between Nakano's
guarded recursion and step indexed models. More recently, Birkedal
\etal~\cite{birkedal12first,birkedal13intensional} have elaborated
this connection, particularly in the direction of dependent types.

Alternative semantics for guarded recursion have involved ultrametric
spaces, for example Birkedal \etal~\cite{birkedal10metric} and
Krishnaswami and Benton \cite{krishnaswami11ultrametric}. Hutton and
Jaskelioff \cite{hutton11representing} have also used an approach
based on metric spaces for identifying productive stream generating
functions.

Finally, we mention the syntactic approach of Severi and de Vries
\cite{severi12pure}, who define the notion of infinitary strong
normalisation to prove productivity of dependent type systems with
guarded recursion.

%%% Local Variables:
%%% TeX-master: "paper"
%%% End:

\section{A type system with clocks and guards}
\label{sec:type-system}

In the introduction, we presented our motivating examples in terms of
an Haskell-like language informally extended with a Nakano-style guard
modality and clock quantification. To be able to formally state the
properties of our combination of clock quantification, clock-indexed
guard modalities and inductive types, in this section we formally
define an extension of the simply-typed $\lambda$-calculus with these
features. In the next section, we define a semantic model for our
system in which we can formally state the results that we claimed in
the introduction.

% FIXME: rewrite this introductory paragraph to flow better

% In this section we present the type system with guarded recursion and
% clock variables that we presented informally in the
% \hyperref[sec:introduction]{introduction}. This type system
% incorporates a general notion of strictly positive inductive types,
% including nested fixpoints, and a guardedness type operator which
% allows for the expression of condinductive types. In
% \autoref{sec:examples} we present several examples of the use of our
% system, including presentations of the informal examples from the
% \hyperref[sec:introduction]{introduction}. In \autoref{sec:semantics}
% we provide a semantics for this system in terms of an untyped
% domain-theoretic model that establishes the soundness properties of
% our typing discipline.

% FIXME: justify switch from :: to : for typing

% FIXME: highlight forwards refs
% FIXME: stop this intro being so formal, continue narrative
% FIXME: highlight the interesting bits

\subsection{Well-formed types with clock variables}
\label{sec:types}

\begin{figure}[t]
  \centering
  \begin{mathpar}
    \inferrule*
    {X \in \Theta}
    {\Delta; \Theta \vdash X : \sortType}
    
    \inferrule*
    { }
    {\Delta; \Theta \vdash 1 : \sortType}
    
    \inferrule*
    {\Delta; \Theta \vdash A : \sortType \\ \Delta; \Theta \vdash B : \sortType}
    {\Delta; \Theta \vdash A \times B : \sortType}
    
    \inferrule*
    {\Delta; \Theta \vdash A : \sortType \\ \Delta; \Theta \vdash B : \sortType}
    {\Delta; \Theta \vdash A + B : \sortType}
    
    \inferrule*
    {\Delta; - \vdash A : \sortType \\ \Delta; \Theta \vdash B : \sortType}
    {\Delta; \Theta \vdash A \to B : \sortType}
    
    \inferrule*
    {\Delta; \Theta, X \vdash A : \sortType}
    {\Delta; \Theta \vdash \mu X. A : \sortType}

    \inferrule*
    {\Delta, \kappa; \Theta \vdash A : \sortType}
    {\Delta; \Theta \vdash \forall \kappa. A : \sortType}

    \inferrule*
    {\Delta; \Theta \vdash A : \sortType \\ \kappa \in \Delta}
    {\Delta; \Theta \vdash \delay\kappa A : \sortType}
  \end{mathpar}
  \caption{Well-formed types and type operators}
  \label{fig:types}
\end{figure}

The types of our system include two kinds of variable that may occur
free and bound: clock variables in the clock quantifier and the
clock-indexed guard modality, as well as type variables occuring in
strictly positive positions for the inductive types. We therefore
formally define when a type is well-formed with respect to contexts of
clock and type variables, using the rules displayed in
\autoref{fig:types}.

The well-formedness judgement for types ($\Delta; \Theta \vdash A :
\sortType$) is defined with respect to a \emph{clock context} $\Delta$
and a \emph{type variable context} $\Theta$. Clock contexts are lists
of clock variables, $\Delta = \kappa_1, ..., \kappa_n$, and type
variable contexts are lists of type variable names, $\Theta = X_1,
..., X_n$. Type variables are only used to manage the scope of the
nested least fixpoint type operator $\mu$; there is no type
polymorphism in the type system we present here. We generally use
roman letters $A$, $B$, $C$ to stand for types, but we also
occasionally use $F$ and $G$ when we wish to emphasise that types with
free type variables are strictly positive type operators. For any type
$A$, we define $\mathit{fc}(A)$ to be the set of free clock variables
that appear in $A$.

Most of the type well-formedness rules are standard: we have a rule
for forming a type from a type variable that is in scope and rules for
forming the unit type $1$, product types $A \times B$, sum types $A +
B$, function types $A \to B$ and least fixpoint types $\mu X. A$. We
enforce the restriction to strictly positive type operators by
disallowing occurences of free type variables in the domain component
of function types. In \autoref{sec:typed-terms}, we will see that each
of the standard type constructors will have the usual corresponding
term-level constructs; our system subsumes the simply typed
$\lambda$-calculus with products, sums and least fixpoint types.

The remaining two rules are for forming clock quantification $\forall
\kappa.A$ types and the clock-indexed Nakano-style guard modality
$\delay\kappa A$. The type formation rule for clock quantification is
very similar to universal type quantification from standard
polymorphic type theories like System F. 

% However, clock quantification
% has a very different flavour to type quantification. The type equality
% rules in \autoref{sec:type-equality} demonstrate that clock
% quantification is ``computationally irrelevant'' and so commutes with
% almost all type operators. Moreover, clock instantiation must be
% linear in order to prevent multiple ``time-streams'' becoming
% conflated (see the statement of the clock substitution lemma below
% \fixme{more detail}).

As we described in the introduction in \autoref{sec:clock-vars}, we
have augmented the Nakano-style delay modality with a clock
variable. This allows us to distinguish between guardedness with
respect to multiple clocks, and hence to be able to close over all the
steps guarded by a particular clock.

\subsection{Type Equality}
\label{sec:type-equality}

The examples we presented in the introduction made use of several type
equalities allowing us to move clock quantification through types. For
instance, in the definition of the $\elim{deStreamCons}$ stream
deconstructor in \autoref{sec:clock-vars}, we made use of a type
equality to treat a value of type $\forall \kappa. \tyname{Integer}$
as having the type $\tyname{Integer}$. Intuitively, this is valid
because a value of type $\tyname{Integer}$ exists independently of any
clock, therefore we can remove the clock quantification. In general,
clock quantification commutes with almost all type formers, except for
guards indexed by the same clock variable. \fixme{Why not force? same reason as for not using type equalities for $\mu$}

The following rules are intended to be read as defining a judgement
$\Delta; \Theta \vdash A \equiv B : \sortType$, indicating that
the well-formed types $\Delta; \Theta \vdash A : \sortType$ and
$\Delta; \Theta \vdash B : \sortType$ are to be considered equal. In
addition to these rules, we also assume that the relation $\Delta
\vdash A \equiv B : \sortType$ is reflexive, symmetric, transitive and
a congruence.
\begin{displaymath}
  \begin{array}{r@{\hspace{0.3em}}c@{\hspace{0.3em}}ll}
    \forall \kappa. A & \equiv & A & (\kappa \not\in \mathit{fc}(A)) \\
    \forall \kappa. A + B & \equiv & (\forall \kappa. A) + (\forall \kappa. B) \\
    \forall \kappa.  A \times B & \equiv & (\forall \kappa. A) \times (\forall \kappa. B) \\
    \forall \kappa. A \to B & \equiv & A \to \forall \kappa. B & (\kappa \not\in \mathit{fc}(A)) \\
    \forall \kappa. \forall \kappa'. A & \equiv & \forall \kappa'. \forall \kappa. A & (\kappa \not= \kappa') \\
    \forall \kappa. \delay\kappa' A & \equiv & \delay\kappa'\forall \kappa. A & (\kappa \not= \kappa') \\
    \forall \kappa. \mu X. F[A,X] & \equiv & \mu X. F[\forall \kappa. A,X] & (\mathit{fc}(F) = \emptyset)
  \end{array}
\end{displaymath}
In the last rule, $F$ must be strictly positive in $A$ as well as $X$.

After we define the terms of our system in the next sub-section, it
will become apparent that several of the type equality rules could be
removed, and their use replaced by terms witnessing the conversions
back and forth. In the case of product types, we could define the
following pair of terms (using the syntax and typing rules we define
below):
\begin{displaymath}
  \begin{array}{l}
    \lambda x. (\Lambda \kappa. \mathbf{fst}~(x[\kappa]), \Lambda \kappa. \kw{snd}~(x[\kappa])) \\
    \hspace{3cm} : (\forall \kappa. A \times B) \to (\forall \kappa. A) \times (\forall \kappa. B)
  \end{array}
\end{displaymath}
and
\begin{displaymath}
  \begin{array}{l}
    \lambda x. \Lambda \kappa. (\kw{fst}~x~[\kappa], \kw{snd}~x~[\kappa]) \\
    \hspace{3cm} : (\forall \kappa. A) \times (\forall \kappa. B) \to \forall \kappa. A \times B
  \end{array}
\end{displaymath}
Indeed, the denotational semantics we present in
\autoref{sec:semantics} will interpret both of these functions as the
identity. However, not all the type equality rules are expressible in
terms of clock abstraction and application, for instance, the $\forall
\kappa. A \equiv A$ rule and the rule for sum types. Moreover, our
definition of $\elim{deCons}$ in \autoref{sec:main-results-intro} is
greatly simplified by having clock quantification commute with all
type formers uniformly (recall that we made crucial use of the
equality $\forall \kappa. F[-] \equiv F[\forall
\kappa. -]$). Therefore, we elect to include type equalities for
commuting clock quantification with all type formers uniformly.

\subsection{Well-typed Terms}\label{sec:typed-terms}

\begin{figure*}[t]
  \centering
  \textbf{1. Simply-typed $\lambda$-calculus with products and sums}
  \begin{mathpar}
    \inferrule* [right=Var]
    {\ident{x} : A \in \Gamma}
    {\Delta; \Gamma \vdash \ident{x} : A}

    \inferrule* [right=Unit]
    { }
    {\Delta; \Gamma \vdash * : 1}

    \inferrule* [right=Abs]
    {\Delta; \Gamma, \ident{x} : A \vdash e : B}
    {\Delta; \Gamma \vdash \lambda \ident{x}.\ e : A \to B}

    \inferrule* [right=App]
    {\Delta; \Gamma \vdash f : A \to B \\
      \Delta; \Gamma \vdash e : A}
    {\Delta; \Gamma \vdash f e : B}

    \inferrule* [right=Pair]
    {\Delta; \Gamma \vdash e_1 : A \\
      \Delta; \Gamma \vdash e_2 : B}
    {\Delta; \Gamma \vdash (e_1, e_2) : A \times B}

    \inferrule* [right=Fst]
    {\Delta; \Gamma \vdash e : A \times B}
    {\Delta; \Gamma \vdash \kw{fst}\ e : A}

    \inferrule* [right=Snd]
    {\Delta; \Gamma \vdash e : A \times B}
    {\Delta; \Gamma \vdash \kw{snd}\ e : B}

    \inferrule* [right=Inl]
    {\Delta; \Gamma \vdash e : A}
    {\Delta; \Gamma \vdash \kw{inl}\ e : A + B}

    \inferrule* [right=Inr]
    {\Delta; \Gamma \vdash e : B}
    {\Delta; \Gamma \vdash \kw{inr}\ e : A + B}

    \inferrule* [right=Case]
    {\Delta; \Gamma \vdash e : A + B \\
      \Delta; \Gamma, \ident{x} : A \vdash f : C \\
      \Delta; \Gamma, \ident{y} : B \vdash g : C}
    {\Delta; \Gamma \vdash
      \kw{case}\ e\ \kw{of}\ \kw{inl}\ \ident{x}. f;
      \kw{inr}\ \ident{y}. g : C}
  \end{mathpar}

  \bigskip
  
  \textbf{2. Least Fixpoint Types}
  \begin{mathpar}
    \inferrule* [right=Cons]
    {\Delta; \Gamma \vdash e : F[\mu X. F]}
    {\Delta; \Gamma \vdash \cons{Cons$_F$}\ e : \mu X. F}

    \inferrule* [right=PrimRec]
    {\Delta; \Gamma \vdash e : F[(\mu X. F) \times A] \to A}
    {\Delta; \Gamma \vdash \elim{primRec$_F$}\ e : \mu X. F \to A}
  \end{mathpar}

  \bigskip

  \textbf{3. Clock Abstraction and Application, and Type Equality}
  \begin{mathpar}
    \inferrule* [right=$\kappa$-Abs]
    {\Delta, \kappa; \Gamma \vdash e : A \\ \kappa \not\in \mathit{fv}(\Gamma)}
    {\Delta; \Gamma \vdash \Lambda \kappa. e : \forall \kappa. A}

    \inferrule* [right=$\kappa$-App]
    {\Delta; \Gamma \vdash e : \forall \kappa. A \\ \kappa' \in \Delta \\ \kappa' \not\in \mathit{fc}(A)}
    {\Delta; \Gamma \vdash e[\kappa'] : A[\kappa \mapsto \kappa']}
      
    \inferrule* [right=TyEq]
    {\Delta; \Gamma \vdash e : A \\ \Delta \vdash A \equiv B}
    {\Delta; \Gamma \vdash e : B}
  \end{mathpar}

  \bigskip

  \textbf{4. Applicative Functor Structure for $\delay\kappa-$}
  \begin{mathpar}
    \inferrule* [right=DePure]
    {\Delta; \Gamma \vdash e : A \\ \kappa \in \Delta}
    {\Delta; \Gamma \vdash \kw{pure}\ e : \delay\kappa A}
    
    \inferrule* [right=DeApp]
    {\Delta; \Gamma \vdash f : \delay\kappa (A \to B) \\
      \Delta; \Gamma \vdash e : \delay\kappa A}
    {\Delta; \Gamma \vdash f \circledast e : \delay\kappa B}
  \end{mathpar}

  \bigskip
  \textbf{5. Fix and Force}
  \begin{mathpar}
    \inferrule* [right=Fix]
    {\Delta; \Gamma \vdash f : \delay\kappa A \to A}
    {\Delta; \Gamma \vdash \kw{fix}\ f : A}

    \inferrule* [right=Force]
    {\Delta; \Gamma \vdash e : \forall \kappa. \delay\kappa A}
    {\Delta; \Gamma \vdash \kw{force}\ e : \forall \kappa. A}
  \end{mathpar}
  \caption{Well-typed terms}
  \label{fig:terms}
\end{figure*}

The well-typed terms of our system are defined by the typing judgement
$\Delta; \Gamma \vdash e : A$, given by the rules presented in
\autoref{fig:terms}. Terms are judged to be well-typed with respect to
a clock variable context $\Delta$, and a typing context $\Gamma$
consisting of a list of variable : type pairs: $x_1 : A_1, ..., x_n :
A_n$. We only consider typing contexts where each type is well-formed
with respect to the clock variable context $\Delta$, and the
\emph{empty} type variable context. Similarly, the result type $A$ in
the typing judgement must be well-formed with respect to $\Delta$ and
the empty type variable context.

We have split the typing rules in \autoref{fig:terms} into five
groups. The first group includes the standard typing rules for the
simply-typed $\lambda$-calculus with unit, product and sum
types. Apart from the additional clock variable contexts $\Delta$,
these rules are entirely standard. The second group contains the rules
for the inductive types $\mu X. F$. Again, apart from the appearance
of clock variable contexts, these rules are the standard constructor
and primitive recursion constructs for inductive types.

The third group of rules cover clock abstraction and application, and
the integration of the type equality rules we defined in the previous
section. In the \TirName{$\kappa$-App} rule, we have used the notation
$A[\kappa \mapsto \kappa']$ to indicate the substitution of the clock
variable $\kappa'$ for $\kappa$. These rules are as one might expect
for System F-style quantification with non-trivial type equality,
albeit that in the \TirName{$\kappa$-App} rule the $\kappa'$ clock
variable cannot already appear in the type. This disallows us from
using the same clock variable twice, preventing multiple
``time-streams'' becoming conflated (see the statement of the clock
substitution lemma below \fixme{more detail}).

The fourth group of rules state that the clock-indexed guard modality
supports the $\kw{pure}$ and apply $(\mathord\circledast)$ operations
of an applicative functor. When we define a denotational semantics of
our calculus in the next section, these operations will be interpreted
simply as the identity function and normal function application
respectively, meaning that the applicative functor laws hold
trivially.

Finally, the fifth group of rules presents the typing for Nakano's
guarded $\kw{fix}$ combinator, now indexed by a clock variable, and
our $\kw{force}$ combinator for removing the guard modality when it is
protected by a clock quantification.

\subsection{Type operators are functorial}
\label{sec:functorial}

Define $\ident{fmap}_F$ for all $F$.

\subsection{Programming with guarded types and clocks}
\label{sec:examples}

\paragraph{Lists and Trees, Finite and Infinite} Using the least
fixpoint type operator $\mu X. F$, we can reconstruct many of the
usual inductive data types used in functional programming. For
example, the OCaml declaration:
\begin{displaymath}
  \kw{type}~\tyname{list} = \cons{NilList} \mathrel| \cons{ConsList}~\kw{of}~A\times\tyname{list}
\end{displaymath}
of finite lists of values of type $A$, can be expressed as the type
$\mu X. 1 + A \times X$ in our system, where we have replaced the two
constructors $\cons{NilList}$ and $\cons{ConsList}$ with a use of the
sum type former. Likewise, the type of binary trees labelled with $A$s
that we used in \autoref{sec:repmin} can be written as $\mu X. A + X
\times X$.

A similar data type declaration in Haskell has a different
interpretation. If we make the following declaration in Haskell:
\begin{displaymath}
  \kw{data}~\tyname{List} = \cons{NilList} \mathrel| \cons{ConsList}~A~\tyname{List}
\end{displaymath}
then the type $\tyname{List}$ includes both finite lists of $A$s, and
infinite lists of $A$s (a similar effect can be achieved in OCaml by
use of the $\kw{lazy}$ type constructor). Haskell's type system is too
weak to distinguish the definitely finite from the possibly infinite
case, and it is often left to the documentation to warn the programmer
that some functions will be non-productive on infinite lists (for
example, the $\ident{reverse}$ function).

Making use of Nakano's guard modality $\delay\kappa-$, we are able to
express Haskell-style possibly infinite lists as the guarded inductive
type $\mu X. 1 + A \times \delay\kappa A$. As we saw in the
introduction, infinite lists can be constructed using the guarded
$\kw{fix}$ operator. For example, the infinite repetition of a fixed
value can be written in our system as:
\begin{displaymath}
  \lambda a. \kw{fix}~(\l. \cons{Cons}~(\kw{inr}~(a,l)))
\end{displaymath}
If we restrict ourselves to only one clock variable, and always use
guarded recursive types, then we obtain a programming language similar
to a monomorphic Haskell, except with a guarantee that all functions
are productive.

\paragraph{Co-inductive Streams} As we saw in
\autoref{sec:infinite-to-finite}, programming with guarded types is
productive, but they are fundamentally incompatible with normal
inductive types. Recall that if we define infinite streams of $A$s as
$\tyname{Stream} = \mu X. A \times \delay\kappa X$, then there is no
way of defining a function $\ident{tail} : \tyname{Stream} \to
\tyname{Stream}$; the best we can do is $\ident{tail} :
\tyname{Stream} \to \delay\kappa\tyname{Stream}$. The rest of the
stream only exists ``tomorrow''.

Clock quantification fixes this problem. We define:
\begin{displaymath}
  \tyname{Stream} = \forall \kappa. \mu X. A \times \delay\kappa A
\end{displaymath}
Now we can define the two deconstructors for making observations on
streams:
\begin{displaymath}
  \begin{array}{lcl}
    \ident{head} &:& \tyname{Stream} \to A \\
    \ident{head} &=& \lambda s. \Lambda \kappa. \elim{primrec}~(\lambda x. \kw{fst}~x)~(s[\kappa]) \\
    \\
    \ident{tail} &:& \tyname{Stream} \to \tyname{Stream} \\
    \ident{tail} &=& \lambda s. \Lambda \kappa. \elim{primrec}~(\lambda x. \kw{snd}~x)~(s[\kappa])
  \end{array}
\end{displaymath}
In the definition of $\ident{head}$ we use the fact that $\kappa$
cannot appear in $A$ to apply the first type equality rule from
\autoref{sec:type-equality}.

\paragraph{Stream Processing}

Ghani, Hancock and Pattinson \cite{streameating} define a type of
representations of continuous functions on streams using a least
fixpoint type nested within a greatest fixpoint. A continuous function
on streams is a function that only requires a finite amount of input
for each output. Ghani \etal's type is expressed in our system as
follows:
\begin{displaymath}
  \tyname{SP}~I~O = \forall \kappa. \mu X. \mu Y. (I \to Y) + (O \times \delay\kappa X)
\end{displaymath}
where $\tyname{SP}$ stands for ``Stream Processor'' and $\tyname{I}$
and $\tyname{O}$ stand for input and output respectively. In this
type, the outer fixpoint permits the production of infinitely many
$O$s, due to the guarded occurence of $X$, while the inner fixpoint
only permits a finite amount of $I$s to be read from the input stream.

Defining the application of an $\tyname{SP}~I~O$ to a stream of $Is$
serves as an interesting example of programming with guards and
clocks. We make use of a helper function for unfolding values of type
$\tyname{SP}~I~O$ that have already been instantiated with some clock:
\begin{displaymath}
  \ident{step} : \ident{SP}_\kappa~I~O \to \mu Y. (I \to Y) \times (O \times \delay\kappa(\ident{SP}_\kappa~I~O))
\end{displaymath}
where $\ident{SP}_\kappa~I~O$ is $\ident{SP}~I~O$ instantiated with
the clock $\kappa$.

The definition of stream processor application goes as follows, where
we permit ourselves a little pattern matching syntactic sugar for
pairs:
\begin{displaymath}
  \begin{array}{@{}l}
    \ident{apply} : \tyname{SP}~I~O \to \tyname{Stream}~I \to \tyname{Stream}~O \\
    \ident{apply} = \\
    \ \begin{array}[t]{@{}l@{}l}
      \lambda \ident{sp}~\ident{s}.\Lambda \kappa. \kw{fix}(\lambda&\ident{rec}~\ident{sp}~\ident{s}. \\
      & \elim{primrec}~(\lambda x~s. 
      \begin{array}[t]{@{}l}
        \kw{case}~x~\kw{of} \\
        \quad \kw{inl}~\ident{f}.~\ident{f}~(\ident{head}~\ident{s}) (\ident{tail}~\ident{s}) \\
        \quad \kw{inr}~(\ident{o},\ident{sp'}).\\
        \quad \quad \cons{Cons}(o,\ident{rec} \circledast \ident{sp'} \circledast \kw{pure}~s))
      \end{array} \\
      & \quad (\ident{step}~\ident{sp})) \\
      & (\ident{sp}~[\kappa])~s
    \end{array}
  \end{array}
\end{displaymath}
This function consists of two nested loops. The outer loop, expressed
using $\kw{fix}$, generates the output stream. This loop is running on
the clock $\kappa$, introduced by the $\Lambda \kappa$. We have
instantiated the stream processor with the same clock; this causes the
steps of the output stream to be in lockstep with the output steps of
the stream processor, exactly as we might expect. The inner loop,
expressed using $\elim{primrec}$, recurses over the finite list of
input instructions from the stream processor, pulling items from the
input stream as needed. The input stream is \emph{not} instantiated
with the same clock as the stream processor and the output stream: we
need to access an arbitrary number of elements from the input stream
for each step of the stream processor, so their clocks cannot run in
lockstep.

\paragraph{The Partiality Monad}

$T A = \forall \kappa. \mu X. A + \delay\kappa X$

Escardo? Danielsson

%%% Local Variables:
%%% TeX-master: "paper"
%%% End:

\section{A denotational semantics for clocks and guards}
\label{sec:semantics}

We have claimed that the typing discipline from the previous section
guarantees that programs will be productive. We now substantiate this
claim by defining a domain-theoretic model of our system, and showing
that the denotation of a well-typed closed term is never $\bot$. We
accomplish this by using a multiply-step-indexed interpretation of
types, where the multiple step indexes correspond to the multiple
clock variables that may be in scope.

\subsection{Semantics of terms}
\label{sec:semantics-of-programs}

We interpret terms in (a mild extension of) the standard domain
theoretic model of the untyped lazy $\lambda$-calculus, described by
Pitts \cite{PittsAM:compavm}. A feature of this semantics is the
erasure of anything to do with the guardedness type constructor
$\delay\kappa-$ and the clock quantification. The $\kw{fix}$ operator
is interpreted as the domain-theoretic fixpoint, i.e., exactly the
usual interpretation of general recursion.

We assume a directed complete partial order with bottom
($\mathrm{DCPO}_\bot$), $D$, satisfying the following recursive domain
equation:
\begin{displaymath}
  D \cong (D \to D)_\bot \oplus (D \times D)_\bot \oplus D_\bot \oplus D_\bot \oplus 1_\bot
\end{displaymath}
where $\oplus$ represents the coalesced sum that identifies the $\bot$
element of all the components. We are guaranteed to be able to obtain
such a $D$ by standard results about the category $\mathrm{DCPO}_\bot$
(see, e.g., Smyth and Plotkin \cite{smyth82category}). The five
components of the right hand side of this equation will be used to
carry functions, products, the left and right injections for sum types
and the unit value respectively. We use the following symbols to
represent the corresponding continuous injective maps into $D$:
\begin{center}
  \begin{tabular}{ll}
    $\semCons{Lam} : (D \to D) \to D$ & $\semCons{Pair} : D \times D \to D$ \\
    $\semCons{Inl} : D \to D$ & $\semCons{Inr} : D \to D$ \\
    $\semCons{Unit} : 1 \to D$
  \end{tabular}  
\end{center}
We will write $\semCons{Unit}$ instead of $\semCons{Unit}\ *$.

\begin{figure}[t]
  \centering
\begin{eqnarray*}
  \sem{x}\eta & = & \eta(x) \\
  \sem{*}\eta & = & \semCons{Unit} \\
  \sem{\lambda x.\ e}\eta & = & \semCons{Lam}\ (\lambda v.\ \sem{e}(\eta[x \mapsto v])) \\
  \sem{fe}\eta & = & \left\{
    \begin{array}{ll}
      d_f\ (\sem{e}\eta) & \textrm{if}\ \sem{f}\eta = \semCons{Lam}\ d_f \\
      \bot & \textrm{otherwise}
    \end{array}
    \right. \\
  \sem{(e_1,e_2)}\eta & = & \semCons{Pair}\ (\sem{e_1}\eta, \sem{e_2}\eta) \\
  \sem{\kw{fst}\ e}\eta & = & \left\{
    \begin{array}{ll}
      d_1 & \textrm{if}\ \sem{e}\eta = \semCons{Pair}\ (d_1, d_2) \\
      \bot & \textrm{otherwise}
    \end{array}
  \right. \\
  \sem{\kw{snd}\ e}\eta & = & \left\{
    \begin{array}{ll}
      d_2 & \textrm{if}\ \sem{e}\eta = \semCons{Pair}\ (d_1, d_2) \\
      \bot & \textrm{otherwise}
    \end{array}
  \right. \\
  \sem{\kw{inl}\ e}\eta & = & \semCons{Inl}\ (\sem{e}\eta) \\
  \sem{\kw{inr}\ e}\eta & = & \semCons{Inr}\ (\sem{e}\eta) \\
  \left\llbracket
    \begin{array}{l}
      \kw{case}\ e\ \kw{of}\\
      \quad\kw{inl}\ \ident{x}.f\\
      \quad\kw{inr}\ \ident{y}.g
    \end{array}
  \right\rrbracket\eta & = &
  \left\{
    \begin{array}{ll}
      \sem{f}(\eta[x \mapsto d]) & \textrm{if }\sem{e}\eta = \semCons{Inl}\ d \\
      \sem{g}(\eta[y \mapsto d]) & \textrm{if }\sem{e}\eta = \semCons{Inr}\ d \\
      \bot & \textrm{otherwise}
    \end{array}
  \right. \\
\end{eqnarray*}  
  \caption{Semantics for the simply-typed portion of our system}
  \label{fig:semantics1}
\end{figure}

Let $V$ be the set of all possible term-level variable
names. Environments $\eta$ are modelled as maps from $V$ to elements
of $D$. Terms are interpreted as continuous maps from environments to
elements of $D$. The clauses for defining the interpretation of parts
of our system that are just the simply-typed $\lambda$-calculus are
standard, and displayed in \autoref{fig:semantics1}. It can be easily
checked that this collection of equations defines a continuous
function $\sem{-} : (V \to D) \to D$, since everything is built from
standard parts.

The applicative functor structure for the guard modality is
interpreted just as the identity function and normal application, as
we promised in \autoref{sec:typed-terms}:
\begin{eqnarray*}
  \sem{\kw{pure}\ e}\eta & = & \sem{e}\eta \\
  \sem{f \circledast e}\eta & = & \left\{
    \begin{array}{ll}
      d_f\ (\sem{e}\eta) & \textrm{if}\ \sem{f}\eta = \semCons{Lam}\ d_f \\
      \bot & \textrm{otherwise}
    \end{array}
  \right.
\end{eqnarray*}
Our interpretation simply translates the $\kw{fix}$ operator as the
usual fixpoint operator $\mathrm{fix}\ f = \bigsqcup_n f^n(\bot)$ in
$\mathrm{DCPO}_\bot$. The $\kw{force}$ operator is interpreted as a
no-operation.
\begin{eqnarray*}
  \sem{\kw{fix}\ f}\eta & = &
  \left\{
    \begin{array}{ll}
      \mathrm{fix}\ d_f & \textrm{if }\sem{f}\eta = \semCons{Lam}\ d_f \\
      \bot & \textrm{otherwise}
    \end{array}
  \right.\\
  \sem{\kw{force}\ e}\eta & = & \sem{e}\eta
\end{eqnarray*}

Finally, we interpret the constructor $\cons{Cons$_F$}$ and primitive
recursion eliminator $\elim{primRec$_F$}$ for least fixpoint
types. The interpretation of construction is straightforward: the
constructor itself disappears at runtime, so the interpretation simply
ignores it:
\begin{eqnarray*}
  \sem{\cons{Cons$_F$}\ e}\eta & = & \sem{e}\eta
\end{eqnarray*}

To define the semantic counterpart of the $\elim{primRec$_F$}$
operator, we first define a higher-order continuous function
$\mathrm{primrec} : (D \to D \to D) \to (D \to D) \to D$ in the
model. This is very similar to how one would define a generic
$\elim{primRec$_F$}$ in Haskell using general recursion, albeit here
in an untyped fashion.
\begin{displaymath}
  \begin{array}{l}
    \mathrm{primrec}\ \mathit{fmap}\ f = \\
    \quad\quad\semCons{Lam}\ (\mathrm{fix}\ (\lambda g x.\ f\ (\mathit{fmap}\ (\semCons{Lam}\ (\lambda x. \semCons{Pair}\ (x, g\ x))))\ x))
  \end{array}
\end{displaymath}
The first argument to $\mathrm{primrec}$ is intended to specify a
functorial mapping function. For each of the syntactic types $\Delta;
\Theta \vdash F : \sortType$ defined in \autoref{fig:types}, we define
a suitable $\mathrm{fmap_F} : D^{|\Theta|} \to D \to D$, where
$|\Theta|$ is the number of type variables in $\Theta$. This
definition is presented in \autoref{fig:semantic-fmap}.

\begin{figure}[t]
  \begin{eqnarray*}
    \mathrm{fmap_X}\ \vec{f}\ x & = & d_f(x)\quad (f_X = \semCons{Lam}\ d_f) \\
    \mathrm{fmap_1}\ \vec{f}\ x & = & \semCons{Unit} \\
    \mathrm{fmap_{F \times G}}\ \vec{f}\ (\semCons{Pair}\ (x,y)) & = & \semCons{Pair}\ (\mathrm{fmap_F}\ \vec{f}\ x, \mathrm{fmap_G}\ \vec{f}\ y) \\
    \mathrm{fmap_{F + G}}\ \vec{f}\ (\semCons{Inl}\ x) & = & \semCons{Inl}\ (\mathrm{fmap_F}\ \vec{f}\ x) \\
    \mathrm{fmap_{F + G}}\ \vec{f}\ (\semCons{Inr}\ x) & = & \semCons{Inr}\ (\mathrm{fmap_G}\ \vec{f}\ x) \\
    \mathrm{fmap_{A \to F}}\ \vec{f}\ (\semCons{Lam}\ g) & = & \semCons{Lam}\ (\lambda v. \mathrm{fmap_F}\ \vec{f}\ (g\ v)) \\
    \mathrm{fmap_{\delayX\kappa F}}\ \vec{f}\ x & = & \mathrm{fmap_F}\ \vec{f}\ x \\
    \mathrm{fmap_{\forall\kappa.F}}\ \vec{f}\ x & = & \mathrm{fmap_F}\ \vec{f}\ x \\
    \mathrm{fmap_{\mu X. F}}\ \vec{f}\ x & = &
    \begin{array}[t]{@{}l}
      \mathrm{primrec}\ (\mathrm{fmap_F}\ \vec{f})\ \\
      \quad\quad\quad\quad(\mathrm{fmap_F}\ \vec{f}\ \mathrm{snd})\ x
    \end{array}
  \end{eqnarray*}
  where
  \begin{displaymath}
    \mathrm{snd}\ (\semCons{Pair}\ (x,y)) = y
  \end{displaymath}
  All unhandled cases are defined to be equal to $\bot$.
  \caption{Semantic $\mathrm{fmap_F}$ for $\Delta; \Theta \vdash F : \sortType$}
  \label{fig:semantic-fmap}
\end{figure}

With these definitions we can define the interpretation of the
syntactic $\elim{primRec$_F$}$ operator. Note that the syntactic type
constructor $F$ here always has exactly one free type variable, so
$\mathrm{fmap_F}$ has type $D \to D \to D$, as required by
$\mathrm{primrec}$.
\begin{displaymath}
  \begin{array}{l}
    \sem{\elim{primRec$_F$}\ f}\eta = \\
    \quad\quad\left\{
      \begin{array}{ll}
        \mathrm{primrec}\ \mathrm{fmap_F}\ d_f & \textrm{if }\sem{f}\eta = \semCons{Lam}\ d_f \\
        \bot & \textrm{otherwise}
      \end{array}
    \right.
  \end{array}
\end{displaymath}

This completes the interpretation of the terms of our programming
language in our untyped model. We now turn to the semantic meaning of
types, with the aim of showing, in \autoref{sec:semantic-soundness},
that each well-typed term's interpretation is semantically well-typed.

\subsection{Interpretation of clock variables}
\label{sec:clock-envs}

For a clock context $\Delta$, we define $\clkSem{\Delta}$ to be the
set of clock environments: mappings from the clock variables in
$\Delta$ to the natural numbers. We use $\delta, \delta'$ etc. to
range over clock environments. For $\delta$ and $\delta'$ in
$\clkSem{\Delta}$, we say that $\delta \sqsubseteq \delta'$ (in
$\clkSem\Delta$) if for all $\kappa \in \Delta$, $\delta(\kappa) \leq
\delta'(\kappa)$. The intended interpretation of some $\delta \in
\clkSem\Delta$ is that $\delta$ maps each clock $\kappa$ in $\Delta$
to a natural number stating how much time that clock has left to
run. The ordering $\delta \sqsubseteq \delta'$ indicates that the
clocks in $\delta$ have at most as much time left as the clocks in
$\delta'$. We use the notation $\delta[\kappa \mapsto n]$ to denote
the clock environment mapping $\kappa$ to $n$ and $\kappa'$ to
$\delta(\kappa')$ when $\kappa \not= \kappa'$.

\subsection{Semantic types}
\label{sec:semantic-types}

We interpret types as $\clkSem\Delta$-indexed families of partial
equivalence relations (PERs) over the semantic domain $D$. Recall that
a PER on $D$ is a binary relation on $D$ that is symmetric and
transitive, but not necessarily reflexive. Since PERs are binary
relations, we will treat them as special subsets of the cartesian
product $D \times D$. We write $\PER(D)$ for the set of all partial
equivalence relations on $D$, and $\top_D$ for the PER $D \times D$.
We require that our $\clkSem\Delta$-indexed families of PERs
contravariantly respect the partial ordering on elements of
$\clkSem\Delta$. This ensures that, as the time left to run on clocks
increases, the semantic type becomes ever more precise. When we
interpret clock quantification as intersection over all
approximations, we capture the common core of all the approximations.

Formally, a semantic type for a clock context $\Delta$ is a function
$A : \clkSem\Delta \to \PER(D)$, satisfying contravariant Kripke
monotonicity: for all $\delta' \sqsubseteq \delta$, $A\delta \subseteq
A\delta'$. We write $\ClkPER(\Delta)$ for the collection of all
semantic types for the clock context $\Delta$. In
\autoref{sec:fixpoint-types}, we will formally consider morphisms
between semantic types, and so turn each $\ClkPER(\Delta)$ into a
category. Note that we do \emph{not} require that any of our PERs are
admissible. Admissible PERs always include the pair $(\bot,\bot)$,
precisely the values we wish to exclude.

We now define semantic counterparts for each of the syntactic type
constructors we presented in \autoref{sec:types}.

\paragraph{Unit, Product, Coproduct and Function Types}

The constructions for unit, product and coproduct types are
straightforward. The unit type will be interpreted by a constant
family of PERs:
\begin{displaymath}
  1\delta = \{(\semCons{Unit}, \semCons{Unit})\}
\end{displaymath}
This trivially satisfies Kripke monotonicity.

Given semantic types $A$ and $B$, their product is defined to be
\begin{displaymath}
  (A \times B)\delta =
  \left\{
    \begin{array}{l}
      (\semCons{Pair}(x,y),\semCons{Pair}(x',y')) \\
      \quad\quad\sepbar (x,x') \in \sem{A}\delta \land (y,y') \in \sem{B}\delta 
    \end{array}
  \right\}
\end{displaymath}
and their coproduct is
\begin{displaymath}
  (A + B)\delta =
  \begin{array}{l}
    \{ (\semCons{Inl}(x), \semCons{Inl}(x')) \mathrel| (x,x') \in \sem{A}\delta \} \\
    \quad \cup \\
    \{ (\semCons{Inr}(y),\semCons{Inr}(y')) \mathrel| (y,y') \in \sem{B}\delta \}
  \end{array}
\end{displaymath}
Since $A$ and $B$ are assumed to be semantic types, it immediately
follows that $A \times B$ and $A + B$ are semantic types.

To interpret function types, we use the usual definition for
Kripke-indexed logical relations, by quantifying over all smaller clock
environments. Given semantic types $A$ and $B$, we define:
\begin{displaymath}
  (A \to B)\delta = \left\{
    \begin{array}{l}
      (\semCons{Lam}(f),\semCons{Lam}(f')) \\
      \quad \sepbar \forall \delta' \sqsubseteq \delta, (x,x') \in A\delta'.\ (fx,f'x') \in B\delta'
    \end{array}
  \right\}
\end{displaymath}
It follows by the standard argument for Kripke logical relations that
this family of PERs is contravariant in clock environments.

\paragraph{Guarded Modality} Given a semantic type $A$, we define the
semantic guard modality $\delay\kappa$ as follows, where $\kappa$ is a
member of the clock context $\Delta$:
\begin{displaymath}
  (\delay\kappa A)\delta = \left\{
    \begin{array}{ll}
      \top_D & \textrm{if }\delta(\kappa) = 0 \\
      A(\delta[\kappa \mapsto n]) & \textrm{if }\delta(\kappa) = n + 1
    \end{array}
  \right.
\end{displaymath}
The semantic type operator $\delay\kappa$ acts differently
depending on the time remaining on the clock $\kappa$ in the current
clock environment $\delta$. When the clock has run to zero,
$\delay\kappa A$ becomes completely uninformative, equating all pairs
of elements in the semantic domain $D$. If there is time remaining,
then $\delay\kappa A$ equates a pair iff $A$ would with one fewer
steps remaining. For Kripke monotonicity, we want to prove that
$(d,d') \in (\delay\kappa A)\delta$ implies $(d,d') \in (\delay\kappa
A)\delta'$ when $\delta' \sqsubseteq \delta$. If $\delta'(\kappa) = 0$
then $(\delay\kappa A)\delta' = D \times D \ni (d,d')$. If
$\delta'(\kappa) = n' + 1$ then $\delta(\kappa) = n + 1$ with $n' \leq
n$. So $(d,d') \in A(\delta[\kappa \mapsto n])$, and so by $A$'s Kripke
monotonicity, $(d,d') \in A(\delta'[\kappa \mapsto n']) =
(\delay\kappa A)\delta'$.

\paragraph{Clock Quantification}
The semantic counterpart of clock quantification takes us from
semantic types in $\ClkPER(\Delta, \kappa)$ to semantic types in
$\ClkPER(\Delta)$. Given a semantic type $A$ in $\ClkPER(\Delta,
\kappa)$, we define
\begin{displaymath}
  \forall_\kappa A = \bigcap_{n \in \mathbb{N}} A(\delta[\kappa \mapsto n])
\end{displaymath}
For Kripke monotonicity, if $(d,d') \in (\forall\kappa. A)\delta$ then
$\forall n.\ (d,d') \in A(\delta[\kappa \mapsto n])$. Since $A$ is a
semantic type, $\forall n.\ (d,d') \in A(\delta'[\kappa \mapsto n])$,
hence $(d,d') \in (\forall\kappa. A)\delta'$.

\paragraph{Complete Lattice Structure} In order to define the semantic
counterpart of the least fixpoint type operator, we make use of the
lattice structure of $\ClkPER(\Delta)$. Given $A, B \in
\ClkPER(\Delta)$, we define a partial order: $A \subseteq B$ if
$\forall \delta. A\delta \subseteq B\delta$. It is easy to see that
$\ClkPER(\Delta)$ is closed under arbitrary intersections, and so is a
complete lattice.

Each of the semantic type constructors above is monotonic with respect
to the partial order on $\ClkPER(\Delta)$ (with the obvious proviso
that $A \to B$ is only monotonic in its second argument).

\paragraph{Least Fixpoint Types}
We make use of the Knaster-Tarski theorem \cite{tarski55lattice} to
produce the least fixpoint of a monotone function on
$\ClkPER(\Delta)$. See Loader \cite{loader97equational} for an similar
usage in a setting without guarded recursion. Given a monotone
function $F : \ClkPER(\Delta) \to \ClkPER(\Delta)$, we define:
\begin{displaymath}
  (\mu F) = \bigcap \{ A \in \ClkPER(\Delta) \sepbar FA \subseteq A \}
\end{displaymath}
For any monotone $F$, $\mu F$ is immediately a semantic type by
construction, since semantic types are closed under intersection. As
an initial step towards semantic type soundness for our calculus, we
demonstrate a semantic well-typedness result for the
$\mathrm{primrec}$ function we defined in
\autoref{sec:semantics-of-programs}.

\begin{lemma}\label{lem:primrec-well-typed}
  Let $F : \ClkPER(\Delta) \to \ClkPER(\Delta)$ be a monotone
  function.  Let $\mathrm{fmap} : D \to D \to D$ be such that for all
  $\delta \in \sem{\Delta}$, for all $A,B \in \ClkPER(\Delta)$, and
  for all $(g,g') \in (A \to B)\delta$ and $(x,x') \in FA\delta$, we
  have $(\mathrm{fmap}\ g\ x, \mathrm{fmap}\ g'\ x') \in FB\delta$.
  Then, for all $\delta \in \sem{\Delta}$ and for all $(f,f') \in
  (F(\mu F \times C) \to C)\delta$, where $C$ is a semantic type, we
  have
  \begin{displaymath}
    (\mathrm{primrec}\ \mathrm{fmap}\ f, \mathrm{primrec}\ \mathrm{fmap}\ f') \in (\mu F \to C)\delta
  \end{displaymath}
\end{lemma}

\begin{proof}
  (Sketch) By induction on the least fixpoint $F(\mu F) = \mu F$,
  using the Knaster-Tarski theorem.
\end{proof}

This lemma only states that our semantic $\mathrm{primrec}$ recursion
operator is semantically well-typed. We will show in
\thmref{thm:initial-f-algebra} that $\mathrm{primrec}$ also witnesses
$\mu F$ as the carrier of the initial $F$-algebra, as we claimed in
\autoref{sec:main-results-intro}.

\subsection{Interpretation of syntactic types}\label{sec:type-interp}

\begin{figure}[t]
  \centering
  \begin{eqnarray*}
    \tySem{1}\theta & = & 1 \\
    \tySem{X}\theta & = & \theta(X) \\
    \tySem{A \times B}\theta & = & \tySem{A}\theta \times \tySem{B}\theta \\
    \tySem{A + B}\theta & = & \tySem{A}\theta + \tySem{B}\theta \\
    \tySem{A \to B}\theta & = & \tySem{A}\emptyset \to \tySem{B}\theta \\
    \tySem{\delay\kappa A}\theta & = & \delay\kappa (\tySem{A}\theta) \\
    \tySem{\forall \kappa. A}\theta & = & \forall_\kappa (\tySem{A}(\theta\mathord\uparrow_\kappa)) \\
    \tySem{\mu X. F}\theta & = & \mu (\lambda X. \tySem{F}(\theta, X))
  \end{eqnarray*}
  \caption{Interpretation of well-formed types}
  \label{fig:type-interp}
\end{figure}

A well-formed type $\Delta; \Theta \vdash A : \sortType$ is
interpreted as monotonic function $\tySem{A} :
\ClkPER(\Delta)^{|\Theta|} \to \ClkPER(\Delta)$, where $|\Theta|$
denotes the number of type variables in $\Theta$. The interpretation
is defined in the clauses in \autoref{fig:type-interp}. These clauses
make use of the constructions of semantic types defined in the
previous subsection. In the clause for $\forall \kappa$, we make use
of the ``clock weakening'' operator $-\mathord\uparrow_\kappa$ which
takes a collection of semantic types in $\ClkPER(\Delta)^{|\Theta|}$
to semantic types in $\ClkPER(\Delta,\kappa)^{|\Theta|}$ by pointwise
restriction of clock environments in $\sem{\Delta, \kappa}$ to clock
environments in $\sem{\Delta}$.

The next lemma states that syntactic substitution and semantic
substitution commute. The proof is a straightforward induction on the
well-formed type $A$. Note the side condition that $\kappa \not\in
\mathit{fc}(A)$, matching the side condition on the
\TirName{$\kappa$-App} typing rule.

\begin{lemma}\label{lem:substitution-lemma}
  Assume that $\Delta; \Theta \vdash A : \sortType$, $\kappa, \kappa'
  \in \Delta$ and $\kappa' \not\in \mathit{fc}(A)$. Then for all
  $\delta \in \sem{\Delta[\kappa\mapsto\kappa']}$ and $\theta \in
  \ClkPER(\Delta)^{|\Theta|}$,
  \begin{displaymath}
    \sem{A}\theta (\delta[\kappa \mapsto \delta(\kappa')]) = \sem{A[\kappa\mapsto\kappa']}(\theta[\kappa\mapsto\kappa'])\delta.
  \end{displaymath}
\end{lemma}

The syntactic type equality judgement $\Delta; \Theta \vdash A \equiv
B : \sortType$ that we defined in \autoref{sec:type-equality} is
interpreted as equality of semantic types. The following lemma states
that this interpretation is sound.

\begin{lemma}\label{lem:type-equality}
  If $\Delta; \Theta \vdash A \equiv B : \sortType$ then for all
  $\theta \in \ClkPER(\Delta)^{|\Theta|}$, $\sem{A}\theta =
  \sem{B}\theta$.
\end{lemma}

The next lemma states that we may use the semantic $\mathrm{fmap_F}$
functions defined in \autoref{fig:semantic-fmap} with the
$\mathrm{primrec}$ operator, by showing that $\mathrm{fmap_F}$ always
satisfies the hypotheses of \lemref{lem:primrec-well-typed}.

\begin{lemma}\label{lem:sem-fmap-well-typed}
  For all $\Delta; \Theta \vdash F : \sortType$, the $\mathrm{fmap}_F$
  defined in \autoref{fig:semantic-fmap} are all semantically
  well-typed: For all $\delta \in \sem{\Delta}$, for all $\overrightarrow{A,B}
  \in \ClkPER(\Delta)$, and for all $\overrightarrow{(g,g') \in (A \to B)\delta}$
  and $(x,x') \in FA\delta$, we have $(\mathrm{fmap_F}\ \overrightarrow{g}\ x,
  \mathrm{fmap_F}\ \overrightarrow{g'}\ x') \in FB\delta$.
\end{lemma}

It may seem that we could prove this lemma by using the syntactic
definition of $\ident{fmap}$ from \autoref{fig:fmap-syn} and then
applying \thmref{thm:semantic-soundness}. However, the semantic
well-typedness of our interpretation of $\elim{primRec}$ depends on
the semantic well-typedness of $\mathrm{fmap}$, so we must prove this
lemma directly in the model to break the circularity.

\subsection{Semantic type soundness}
\label{sec:semantic-soundness}

We now state our first main result: the semantic type soundness
property of our system. To state this result, we define the semantic
interpretation of contexts as a clock-environment indexed collection
of PERs over environments:
\begin{displaymath}
  \sem{\Gamma}\delta = \{(\eta, \eta') \mathrel| \forall (\ident{x} : A) \in \Gamma.\ (\eta(\ident{x}), \eta'(\ident{x})) \in \sem{A}\delta \}
\end{displaymath}

\begin{theorem}\label{thm:semantic-soundness}
  If $\Delta; \Gamma \vdash e : A$, then for all $\delta \in
  \sem{\Delta}$ and $(\eta,\eta') \in \sem{\Gamma}\delta$,
  $(\sem{e}\eta,\sem{e}\eta') \in \sem{A}\delta$.
\end{theorem}

\begin{proof}
  (Sketch) By induction on the derivation of $\Delta; \Gamma \vdash e
  : A$. The most interesting cases are for $\kw{fix}$ and
  $\elim{primRec}$. In essence, the case for $\kw{fix}$ goes through
  by induction on the value of the counter assigned to the clock
  variable $\kappa$ in the current clock environment. The case for
  $\elim{primRec}$ is given by \lemref{lem:primrec-well-typed} and
  \lemref{lem:sem-fmap-well-typed}.
\end{proof}

\begin{corollary}
  If $-; - \vdash e : A$ then for all $\eta$, $\sem{e}\eta \not=
  \bot$.
\end{corollary}
By this corollary, the denotation of a closed program is never $\bot$,
so well-typed programs always yield a proper value.  When the result
type $A$ is the stream type $\forall \kappa. \mu X. B \times
\delay\kappa X$, we can deduce that the denotation of $e$ will always
be infinite streams of elements of $B$. We further elaborate this
point in the next section, showing that this type is the carrier of
the final $(B \times -)$-coalgebra.

%%% Local Variables:
%%% TeX-master: "paper"
%%% End:

\section{Properties of fixpoint types}
\label{sec:fixpoint-types}

Using the denotational model of the previous section, we are now in a
position to formally state and sketch the proofs of the properties of
fixpoint types that we claimed in
\autoref{sec:main-results-intro}. Our claimed results are statements
in category theoretic language about the initiality and finality of
various (co)algebras. Therefore, we first construct suitable
categories to work in.

\begin{definition}
  For each clock variable context $\Delta$, the category
  $\ClkPER(\Delta)$ has as objects semantic types over
  $\Delta$. Morphisms $f : A \to B$ are continuous functions $f : D
  \to D$ such that for all $\delta \in \sem{\Delta}$ and $(a,a') \in
  A\delta$, $(fa, fa') \in B\delta$. Two morphisms $f$,$f'$ are
  considered equivalent if for all $\delta \in \sem{\Delta}$ and
  $(a,a') \in A\delta$, $(fa, f'a') \in B\delta$.
\end{definition}

Each well-typed term $\Delta; x : A \vdash e : B$ defines a morphism
$\sem{e} : \sem{A} \to \sem{B}$ in $\ClkPER(\Delta)$. We can define
equality between terms in our syntactic type system in terms of the
equality on morphisms in this definition. Moreover, each well-formed
type operator $\Delta; X \vdash F[X] : \sortType$ defines a (strong)
endofunctor on $\ClkPER(\Delta)$, using the semantic $\mathrm{fmap}_F$
defined in \autoref{fig:semantic-fmap} to define the action on
morphisms. We have already checked (\lemref{lem:sem-fmap-well-typed})
that this is well-defined, and it is straightforward to check that the
usual functor laws hold. It what follows, we will use
$\mathit{fmap}_F$ to refer to the action of a functor $F$ on
morphisms.

\paragraph{Initial Algebras} Recall that, for any functor $F$, an
$F$-algebra is a pair of an object $A$ and a morphism $k : FA \to
A$. A homomorphism $h$ between two $F$-algebras $(A,k_A)$ and
$(B,k_B)$ is a morphism $h : A \to B$ such that $h \circ k_A = k_B
\circ \mathit{fmap}_F~h$. An \emph{initial} $F$-algebra is an
$F$-algebra $(A,k_A)$ such that for any other $F$-algebra $(B,k_B)$,
there is a unique homomorphism $h : (A,k_A) \to (B,k_B)$.

\begin{theorem}\label{thm:initial-f-algebra}
  If $F$ is an endofunctor on $\ClkPER(\Delta)$, then $\mu F$ is the
  carrier of an initial $F$-algebra.
\end{theorem}

\begin{proof}
  Since $\mu F$ is a fixpoint of $F$, the morphism $F(\mu F) \to \mu
  F$ is simply realised by the identity map. Given any other
  $F$-algebra $(B,k_B)$, define a morphism $\mu F \to B$ using the
  $\mathrm{primrec}$ operator from
  \lemref{lem:primrec-well-typed}. Checking that this gives an
  $F$-algebra homomorphism is straightforward, proving uniqueness uses
  induction on elements of $\mu F$, by the Knaster-Tarski theorem.
\end{proof}

\paragraph{Guarded Final Co-Algebras} \thmref{thm:initial-f-algebra}
applies to all functors $F$, and in particular functors of the form
$F(\delay\kappa-)$ on the category $\ClkPER(\Delta, \kappa)$. For
these functors, as well as $\mu (F(\delay\kappa -))$ being the carrier
of an initial algebra, it is also the carrier of a final
$F(\delay\kappa-)$-coalgebra.

Coalgebras are the formal dual of algebras: for an endofunctor $F$, an
$F$-coalgebra is a pair of an object $A$ and a morphism $k_A : A \to
FA$. A homomorphism $h : (A,k_A) \to (B,k_B)$ of coalgebras is a
morphism $h : A \to B$ such that $\mathit{fmap}_F~h \circ k_A =
k_B\circ h$. A \emph{final} $F$-coalgebra is an $F$-coalgebra
$(B,k_B)$ such that for any other $F$-coalgebra $(A,k_A)$, there is a
unique $F$-coalgebra homomorphism $\mathrm{unfold}~k_A : (A,k_A) \to
(B,k_B)$.

\begin{theorem}\label{thm:final-f-de-coalgebra}
  If $F$ is an endofunctor on $\ClkPER(\Delta)$, then $\mu
  (F(\delay\kappa-))$ is the carrier of a final
  $F(\delay\kappa-)$-coalgebra in $\ClkPER(\Delta,\kappa)$.
\end{theorem}

\begin{proof} (Sketch) As for \thmref{thm:initial-f-algebra}, since
  $\mu (F(\delay\kappa-))$ is a fixpoint of $F(\delay\kappa-)$, the
  morphism $F(\mu F) \to \mu F$ is simply realised by the identity
  map. Given any other $F$-algebra $(A,k_A)$, define a morphism
  $\mathrm{unfold}~k_A : A \to \mu(F(\delay\kappa-))$ as
  \fixme{FIX}. It is straightforward to prove that this is an
  $F$-coalgebra homomorphism. Uniqueness is proved for each possible
  clock environment $\delta$ by induction on $\delta(\kappa)$.
\end{proof}

The syntactic counterpart of the construction we used in this proof is
exactly the term we used in \autoref{sec:main-results-intro} for the
definition of $\ident{unfold}$. It is also easy to check that the term
$\ident{deCons}$ we defined there is semantically equivalent to the
identity. Therefore, \thmref{thm:final-f-de-coalgebra} substantiates
the claim we made in \autoref{sec:main-results-intro} that $\mu
X. F[\delay\kappa-]$ is the syntactic description of the carrier of a
final $F[\delay-]$-coalgebra.

\paragraph{Final Co-Algebras} \thmref{thm:final-f-de-coalgebra} yields
final coalgebras in the categories $\ClkPER(\Delta, \kappa)$, where we
have a spare clock variable. By using clock quantification, we can
close over this spare clock variable, and so obtain the final
$F$-coalgebra, not just the final $F(\delay\kappa-)$-coalgebra.

\begin{theorem}\label{thm:final-f-coalgebra}
  For an endofunctor $F$ on $\ClkPER(\Delta)$, $\forall_\kappa. \mu
  (F(\delay\kappa-))$ is the carrier of a final $F$-coalgebra in
  $\ClkPER(\Delta)$.
\end{theorem}

\begin{proof}
  Almost identical to the proof for \thmref{thm:final-f-de-coalgebra}.
\end{proof}

This final theorem, along with the examples we presented in
\autoref{sec:examples}, substantiates our claim that the combination
of clocks and guards that we have presented in this paper is a viable
and comfortable setting in which to productively coprogram.

%%% Local Variables:
%%% TeX-master: "paper"
%%% End:

\section{Conclusions and Further Work}
\label{sec:conclusions}

We have presented a semantics for a small total calculus with
primitive recursion for inductive data and a compositional treatment
of corecursion, ensuring causality via the applicative structure of a
\emph{local} notion of time. In effect, we use time-based typing to
grow a given total language, where all computation terminates within
one `day', into a larger total language, where additional recursion is
justified clearly by the advancing clock. Functions from clocked
inputs to clocked outputs enforce precise producer-consumer
contracts---today's output must be computed only from today's
input---documenting their utility as components of productive
processes. Quantifying clock variables localises the time stream to a
particular construction whose clients can then use it `in the moment'.
The method, made local, can be iterated, with inner clocks justifying
the totality of computations within one `day' of an outer clock.

At present, however, we have used local time only to justify
productive corecursion, with only primitive recursion for inductive
types. It seems pressing to ask whether local time might similarly
liberalise termination checking, with a local clock measuring time into
the past and ensuring that recursive calls receive old enough inputs
that their outputs are ready when we need them. We are actively
seeking a semantics for such a system, but it currently seems more
difficult to pin down.

In due course, we should like to grow this experimental calculus to a
full blown dependent type theory where (co)recursive constructions are
checked to be total within nested local time streams, then exported to their
clients without clocking. At least we have now established what local
time streams are and how to extract productive processes from them.


%%% Local Variables:
%%% TeX-master: "paper"
%%% End:


\paragraph{Acknowledgements} We would like to thank Lars Birkedal and
Rasmus M{\o}gelberg for extremely useful discussions.

\begin{thebibliography}{24}
\providecommand{\natexlab}[1]{#1}
\providecommand{\url}[1]{\texttt{#1}}
\expandafter\ifx\csname urlstyle\endcsname\relax
  \providecommand{\doi}[1]{doi: #1}\else
  \providecommand{\doi}{doi: \begingroup \urlstyle{rm}\Url}\fi

\bibitem[Abel(2004)]{abel04termination}
A.~Abel.
\newblock Termination checking with types.
\newblock \emph{ITA}, 38\penalty0 (4):\penalty0 277--319, 2004.

\bibitem[Appel and McAllester(2001)]{appel01indexed}
A.~W. Appel and D.~A. McAllester.
\newblock An indexed model of recursive types for foundational proof-carrying
  code.
\newblock \emph{ACM Trans. Program. Lang. Syst.}, 23\penalty0 (5):\penalty0
  657--683, 2001.

\bibitem[Bird(1984)]{bird84circular}
R.~S. Bird.
\newblock {Using Circular Programs to Eliminate Multiple Traversals of Data}.
\newblock \emph{Acta Informatica}, 21:\penalty0 239--250, 1984.

\bibitem[Birkedal and M{\o}gelberg(2013)]{birkedal13intensional}
L.~Birkedal and R.~E. M{\o}gelberg.
\newblock Intensional type theory with guarded recursive types qua fixed points
  on universes.
\newblock In \emph{ACM/IEEE Symposium on Logic in Computer Science (LICS
  2013)}, 2013.

\bibitem[Birkedal et~al.(2010)Birkedal, Schwinghammer, and
  St{\o}vring]{birkedal10metric}
L.~Birkedal, J.~Schwinghammer, and K.~St{\o}vring.
\newblock A metric model of guarded recursion.
\newblock In \emph{Presented at FICS 2010}, 2010.

\bibitem[Birkedal et~al.(2012)Birkedal, M{\o}gelberg, Schwinghammer, and
  St{\o}vring]{birkedal12first}
L.~Birkedal, R.~E. M{\o}gelberg, J.~Schwinghammer, and K.~St{\o}vring.
\newblock First steps in synthetic guarded domain theory: step-indexing in the
  topos of trees.
\newblock \emph{Log. Meth. in Computer Science}, 8\penalty0 (4), 2012.

\bibitem[Capretta(2005)]{capretta05general}
V.~Capretta.
\newblock General recursion via coinductive types.
\newblock \emph{Log. Meth. in Computer Science}, 1\penalty0 (2), 2005.

\bibitem[Danielsson(2010)]{danielsson10beating}
N.~A. Danielsson.
\newblock Beating the productivity checker using embedded languages.
\newblock In \emph{Workshop on Partiality and Recursion in Interactive Theorem
  Provers (PAR 2010)}, volume~43 of \emph{EPTCS}, pages 29--48, 2010.

\bibitem[Danielsson and Altenkirch(2009)]{danielsson09mixing}
N.~A. Danielsson and T.~Altenkirch.
\newblock Mixing induction and coinduction.
\newblock Draft, 2009.

\bibitem[Dreyer et~al.(2011)Dreyer, Ahmed, and Birkedal]{dreyer11logical}
D.~Dreyer, A.~Ahmed, and L.~Birkedal.
\newblock Logical step-indexed logical relations.
\newblock \emph{Log. Meth. in Computer Science}, 7\penalty0 (2), 2011.

\bibitem[Ghani et~al.(2009)Ghani, Hancock, and
  Pattinson]{ghani09representations}
N.~Ghani, P.~Hancock, and D.~Pattinson.
\newblock Representations of stream processors using nested fixed points.
\newblock \emph{Log. Meth. in Computer Science}, 5\penalty0 (3), 2009.

\bibitem[Gim{\'e}nez(1994)]{gimenez94codifying}
E.~Gim{\'e}nez.
\newblock Codifying guarded definitions with recursive schemes.
\newblock In \emph{Types for Proofs and Programs, International Workshop
  TYPES'94}, volume 996 of \emph{Lecture Notes in Computer Science}, pages
  39--59, 1994.

\bibitem[Hutton and Jaskelioff(2011)]{hutton11representing}
G.~Hutton and M.~Jaskelioff.
\newblock Representing contractive functions on streams.
\newblock Submitted, 2011.

\bibitem[Krishnaswami and Benton(2011{\natexlab{a}})]{krishnaswami11semantic}
N.~R. Krishnaswami and N.~Benton.
\newblock A semantic model for graphical user interfaces.
\newblock In \emph{ACM SIGPLAN international conference on Functional
  Programming, ICFP 2011}, pages 45--57, 2011{\natexlab{a}}.

\bibitem[Krishnaswami and
  Benton(2011{\natexlab{b}})]{krishnaswami11ultrametric}
N.~R. Krishnaswami and N.~Benton.
\newblock Ultrametric semantics of reactive programs.
\newblock In \emph{IEEE Symposium on Logic in Computer Science, LICS 2011},
  pages 257--266, 2011{\natexlab{b}}.

\bibitem[Launchbury and {Peyton Jones}(1994)]{launchbury94lazy}
J.~Launchbury and S.~L. {Peyton Jones}.
\newblock Lazy functional state threads.
\newblock In \emph{Proceedings of the ACM SIGPLAN'94 Conference on Programming
  Language Design and Implementation (PLDI)}, pages 24--35, 1994.

\bibitem[Loader(1997)]{loader97equational}
R.~Loader.
\newblock {Equational Theories for Inductive Types}.
\newblock \emph{Annals of Pure and Applied Logic}, 84\penalty0 (2):\penalty0
  175--217, 1997.

\bibitem[McBride and Paterson(2008)]{mcbride08applicative}
C.~McBride and R.~Paterson.
\newblock Applicative programming with effects.
\newblock \emph{J. Funct. Prog.}, 18\penalty0 (1):\penalty0 1--13, 2008.

\bibitem[Moggi and Sabry(2001)]{DBLP:journals/jfp/MoggiS01}
E.~Moggi and A.~Sabry.
\newblock Monadic encapsulation of effects: a revised approach (extended
  version).
\newblock \emph{J. Funct. Prog.}, 11\penalty0 (6):\penalty0 591--627, 2001.

\bibitem[Nakano(2000)]{nakano00modality}
H.~Nakano.
\newblock A modality for recursion.
\newblock In \emph{IEEE Symposium on Logic in Computer Science (LICS 2000)},
  pages 255--266, 2000.

\bibitem[Pitts(1994)]{PittsAM:compavm}
A.~M. Pitts.
\newblock Computational adequacy via `mixed' inductive definitions.
\newblock In \emph{Mathematical Foundations of Programming Semantics, Proc.\
  9th Int.\ Conf.}, volume 802 of \emph{Lecture Notes in Computer Science},
  pages 72--82. Springer-Verlag, Berlin, 1994.

\bibitem[Severi and {de Vries}(2012)]{severi12pure}
P.~Severi and F.-J. {de Vries}.
\newblock Pure type systems with corecursion on streams: from finite to
  infinitary normalisation.
\newblock In \emph{ACM SIGPLAN International Conference on Functional
  Programming, ICFP'12}, 2012.

\bibitem[Smyth and Plotkin(1982)]{smyth82category}
M.~B. Smyth and G.~D. Plotkin.
\newblock The category-theoretic solution of recursive domain equations.
\newblock \emph{SIAM J. Comput.}, 11\penalty0 (4):\penalty0 761--783, 1982.

\bibitem[Tarski(1955)]{tarski55lattice}
A.~Tarski.
\newblock A lattice-theoretical fixpoint theorem and its applications.
\newblock \emph{Pacific Journal of Mathematics}, 5\penalty0 (2):\penalty0
  285--309, 1955.

\end{thebibliography}


% \bibliographystyle{plainnat}
% \bibliography{paper}

%\appendix
%\section{Appendix}
\label{sec:appendix}

\lemref{lem:primrec-well-typed}:

\begin{proof}
  We apply the Knaster-Tarski theorem to do induction on the least
  fixpoint $\mu F = F(\mu F)$.

  Let $h = \mathrm{fix}\ (\lambda g x. f\ (\mathrm{fmap}\
  (\semCons{Lam}\ (\lambda x. \semCons{Pair}\ (x, g\ x)))))$ and $h =
  \mathrm{fix}\ (\lambda g x. f\ (\mathrm{fmap}\ (\semCons{Lam}\
  (\lambda x. \semCons{Pair}\ (x, g\ x)))))$, so that
  $\mathrm{primrec}\ \mathrm{fmap}\ f = \semCons{Lam}\ h$ and likewise
  for $h'$. By the definition of the semantic function type
  constructor it suffices to show that for all $(x,x') \in \mu F
  \delta$, $(h x, h' x') \in C\delta$. Let $P \in \ClkPER(\Delta)$ be
  defined as:
  \begin{displaymath}
    P\delta = \{ (x,x') \in (\mu F)\delta \sepbar (h x, h' x') \in C\delta \},
  \end{displaymath}
  If we can show that $P = \mu F$ then the lemma is proved. We first
  need to check that $P$ is actually a semantic type, which follows
  from the facts that $\mu F$ and $C$ are semantic types, and the
  continuity of $h$ and $h'$. To apply the Knaster-Tarski theorem to
  show that $P = \mu F$ we must show that $FP \subseteq P$. So for any
  $(y,y') \in FP\delta$, we must show that $(h y, h' y') \in
  C\delta$. By unfolding the applications of $\mathrm{fix}$ in the
  definitions of $h$ and $h'$ this amounts to showing that
  \begin{displaymath}
    \left(
      \begin{array}{l}
        f\ (\mathrm{fmap}\ (\semCons{Lam}\ (\lambda x. \semCons{Pair}\ (x,h\ x)))\ y),\\
        f'\ (\mathrm{fmap}\ (\semCons{Lam}\ (\lambda x. \semCons{Pair}\ (x,h'\ x)))\ y')
      \end{array}
    \right)\in C\delta
  \end{displaymath}
  By the assumption that $(f,f') \in (F(\mu F \times C) \to C)\delta$
  it suffices to show that
  \begin{equation}
    \label{eq:primrec-welltyped-1}
    \left(
      \begin{array}{l}
        \mathrm{fmap}\ (\semCons{Lam}\ (\lambda x. \semCons{Pair}\ (x,h\ x)))\ y,\\
        \mathrm{fmap}\ (\semCons{Lam}\ (\lambda x. \semCons{Pair}\ (x,h'\ x)))\ y'
      \end{array}
    \right)\in F(\mu F \times C)\delta
  \end{equation}
  Now, since $(\semCons{Lam}\ (\lambda x. \semCons{Pair}\ (x,h\ x)),
  \semCons{Lam}\ (\lambda x. \semCons{Pair}\ (x,h'\ x))) \in (P \to
  \mu F \times C)\delta$, we can use the assumption on $\mathrm{fmap}$
  to show \autoref{eq:primrec-welltyped-1}, and we are done.
\end{proof}

\lemref{lem:substitution-lemma}:

\begin{proof}
  By induction on the structure of $A$. The interesting cases go as
  follows:

  When $A = \delay{\kappa''}A'$, we have two cases depending on
  whether $\kappa'' = \kappa$ or not. If $\kappa'' = \kappa$, then
  $A[\kappa \mapsto \kappa'] = (\delay\kappa A')[\kappa \mapsto
  \kappa'] = \delay{\kappa'} (A[\kappa \mapsto \kappa'])$. The LHS of
  the equation to prove is:
  \begin{displaymath}
    \begin{array}{cl}
      & \sem{A}\theta (\delta[\kappa \mapsto \delta(\kappa')]) \\
      =& \sem{\delay{\kappa}A'} \theta (\delta[\kappa \mapsto \delta(\kappa')]) \\
      =& \left\{
        \begin{array}{ll}
          \sem{A'}\theta(\delta[\kappa \mapsto n]) & \textrm{if }\delta(\kappa') = n + 1 \\
          \top_D & \textrm{otherwise}
        \end{array}
      \right.
    \end{array}
  \end{displaymath}
  While the RHS of the equation is:
  \begin{displaymath}
    \begin{array}{cl}
      & \sem{A[\kappa \mapsto \kappa']}(\theta[\kappa\mapsto\kappa'])\delta \\
      =&\sem{\delay{\kappa'}(A'[\kappa\mapsto\kappa'])}(\theta[\kappa\mapsto\kappa'])\delta \\
      =&\left\{
        \begin{array}{ll}
          \sem{A'[\kappa\mapsto\kappa']}(\theta[\kappa\mapsto\kappa'])(\delta[\kappa' \mapsto n]) & \textrm{if }\delta(\kappa') = n + 1 \\
          \top_D & \textrm{otherwise}
        \end{array}
      \right.
    \end{array}
  \end{displaymath}
  Thus if $\delta(\kappa') = 0$, then we are done. If $\delta(\kappa')
  = n + 1$ we have $\sem{A'}\theta(\delta[\kappa\mapsto n]) =
  \sem{A'}\theta(\delta[\kappa'\mapsto n][\kappa \mapsto
  \delta(\kappa')])$, since $\kappa' \not\in \mathit{fv}(A)$, which is
  equal to
  $\sem{A'[\kappa\mapsto\kappa']}(\theta[\kappa\mapsto\kappa'])(\delta[\kappa'\mapsto
  n])$ by the induction hypothesis, as required.

  If $\kappa'' \not\in \kappa$, then we have $A[\kappa \mapsto
  \kappa'] = (\delay{\kappa''} A')[\kappa \mapsto \kappa'] =
  \delay{\kappa''} (A'[\kappa \mapsto \kappa'])$. The LHS of the
  equation to prove is:
  \begin{displaymath}
    \begin{array}{cl}
      & \sem{A}\theta (\delta[\kappa \mapsto \delta(\kappa')]) \\
      =& \sem{\delta{\kappa''}A'} \theta (\delta[\kappa \mapsto \delta(\kappa')]) \\
      =&\left\{
        \begin{array}{ll}
          \sem{A'}\theta(\delta[\kappa'' \mapsto n, \kappa \mapsto \delta(\kappa')]) & \textrm{if }\delta(\kappa'') = n + 1 \\
          \top_D & \textrm{otherwise}
        \end{array}
      \right.
    \end{array}
  \end{displaymath}
  While the RHS of the equation is:
  \begin{displaymath}
    \begin{array}{@{}c@{\hspace{0.3em}}l}
      & \sem{A[\kappa \mapsto \kappa']}(\theta[\kappa\mapsto\kappa'])\delta \\
      =&\sem{\delay{\kappa''}(A'[\kappa\mapsto\kappa'])}(\theta[\kappa\mapsto\kappa'])\delta \\
      =&\left\{
        \begin{array}{ll}
          \sem{A'[\kappa\mapsto\kappa']}(\theta[\kappa\mapsto\kappa'])(\delta[\kappa'' \mapsto n]) & \textrm{if }\delta(\kappa'') = n + 1\\
          \top_D & \textrm{otherwise}
        \end{array}
      \right.
    \end{array}
  \end{displaymath}
  The required equation now follows directly from the induction
  hypothesis.
\end{proof}

\lemref{lem:type-equality}:

\begin{proof}
  By induction on the derivation of $\Delta \vdash A \equiv
  B$. Reflexivity, symmetry, transitivity and congruence are all
  trivial. 

  When the equation is $\forall \kappa. A \equiv A$: we have $(d,d')
  \in \sem{\forall \kappa. A}\delta$ iff $\forall n. (d,d') \in
  \sem{A}(\delta[\kappa\mapsto n])$ iff (since $\kappa \not\in
  \mathit{fv}(A)$) $(d,d') \in \sem{A}\delta$.  

  When the equation is $\forall \kappa. A + B \equiv (\forall
  \kappa. A) + (\forall \kappa. B)$: If $(d,d') \in \sem{\forall
    \kappa. A + B}\delta$ then $(d,d') \in \sem{A + B}(\delta[\kappa
  \mapsto n])$ for all $n$. By case analysis on $d$ and $d'$, either
  $d = \mathsf{Inl}\ x$, $d' = \mathsf{Inr}\ x'$ and $(x,x') \in
  \sem{A}(\delta[\kappa \mapsto n])$ for all $n$, or $d =
  \mathsf{Inr}\ y$, $d' = \mathsf{Inr}\ y'$ and $(y,y') \in
  \sem{B}(\delta[\kappa\mapsto n])$ for all $n$. In either case we are
  done. The opposite direction is similar.

  When the equation is $\forall \kappa. A \times B \equiv (\forall
  \kappa. A) \times (\forall \kappa. B)$: $(d,d') \in \sem{\forall
    \kappa. A \times B}\delta$ iff $(d,d') \in \sem{A \times
    B}(\delta[\kappa \mapsto n])$ for all $n$. This is exactly when $d
  = \mathsf{Pair}(x,y)$, $d' = \mathsf{Pair}(x',y')$ and $\forall n.\
  (x,x') \in \sem{A}(\delta[\kappa \mapsto n])$ and $\forall n.\
  (y,y') \in \sem{A}(\delta[\kappa \mapsto n])$, as required.

  When the equation is $\forall \kappa. A \to B \equiv A \to \forall
  \kappa. B$: if $(d,d') \in \sem{\forall \kappa.A \to B}\delta$ then
  $d = \mathsf{Lam}\ f$ and $d' = \mathsf{Lam}\ f'$, for some $f$,$f'$
  such that $\forall n. \forall \delta' \sqsubseteq \delta[\kappa
  \mapsto n], (x,x') \in \sem{A}\delta'. (f(x),f'(x')) \in
  \sem{B}\delta'$. Given $\delta'' \sqsubseteq \delta$ and $(x,x') \in
  \sem{A}\delta''$ then for all $n$, $(x,x') \in
  \sem{A}(\delta''[\kappa \mapsto n])$ (since $\kappa \not\in
  \mathit{fv}(A)$), and $\delta''[\kappa \mapsto n] = \delta[\kappa
  \mapsto n]$. Therefore $(f(x),f'(x')) \in \sem{B}(\delta''[\kappa
  \mapsto n])$, and so $(d,d') = (\mathsf{Lam}\ f, \mathsf{Lam}\ f')
  \in \sem{A \to \forall \kappa. B}\delta$. In the opposite direction,
  if $(d,d') = (\mathsf{Lam}\ f, \mathsf{Lam}\ f') \in \sem{A \to
    \forall \kappa.B}\delta$ then $\forall \delta' \sqsubseteq \delta,
  (x,x') \in \sem{A}\delta'.\ \forall n. (f(x),f'(x')) \in
  \sem{B}(\delta'[\kappa \mapsto n])$. Given $n$, $\delta''
  \sqsubseteq \delta[\kappa \mapsto n]$ and $(x,x') \in
  \sem{A}\delta''$, then there exists a $\delta'$ and $n' \leq n$ with
  $\delta'' = \delta'[\kappa \mapsto n'$. Since $\kappa \not\in
  \mathit{fv}(A)$, $(x,x') \in \sem{A}\delta'$, and $(f(x),f'(x')) \in
  \sem{B}(\delta'[\kappa \mapsto n']) = \sem{B}\delta''$, as
  required.

  When the equation is $\forall \kappa. \forall \kappa'. A \equiv
  \forall \kappa'. \forall \kappa. A$: $(d,d') \in
  \sem{A}(\delta[\kappa \mapsto n, \kappa' \mapsto n'])$ for all
  $n,n'$ if and only if $(d,d') \in \sem{A}(\delta[\kappa' \mapsto n',
  \kappa \mapsto n])$ for all $n',n$. This is just trivial interchange
  of universal quantification.

  FIXME: write up the last two equations.
\end{proof}

\thmref{thm:semantic-soundness}:

\begin{proof}
  By induction on the derivation of $\Delta; \Gamma \vdash e : A$. For
  \TirName{Var}, by assumption $(\eta(\ident{x}),\eta'(\ident{x})) \in
  \sem{A}\delta$ so we are done.

  For \TirName{Unit}, from the interpretation of terms, we know that
  for all $\eta$ and $\eta'$, $(\sem{*}\eta, \sem{*}\eta') =
  (\semCons{Unit}, \semCons{Unit}) \in \sem{1}\delta$.

  For \TirName{Abs}, by the induction hypothesis, we know that for all
  $\delta \in \sem{\Delta}$, $(\eta,\eta') \in \sem{\Gamma}\delta$ and
  $(v,v') \in \sem{A}\delta$, $(\sem{e}(\eta[\ident{x} \mapsto v]),
  \sem{e}(\eta[\ident{x} \mapsto v']) \in \sem{B}\delta$. So for any
  $\delta \in \sem{\Delta}$, we need to prove that $(\eta,\eta') \in
  \sem{\Gamma}\delta$ implies that $(\semCons{Lam}\ (\lambda v.\
  \sem{e}(\eta[\ident{x} \mapsto v])), \semCons{Lam}\ (\lambda v.\
  \sem{e}(\eta'[\ident{x} \mapsto v]))) \in \sem{A \to
    B}\delta$. Given $\delta' \sqsubseteq \delta$ and $(v,v') \in
  \sem{A}\delta'$, we also know that $(\eta,\eta') \in
  \sem{\Gamma}\delta'$ by Kripke monotonicity, so we can apply the
  inductive hypothesis to show that $(\sem{e}(\eta[\ident{x} \mapsto
  v]), \sem{e}(\eta'[\ident{x} \mapsto v'])) = ((\lambda
  v. \sem{e}(\eta[\ident{x} \mapsto v]))v, (\lambda v.\
  \sem{e}(\eta'[\ident{x} \mapsto v]))v') \in \sem{B}\delta'$, as
  required.

  For \TirName{App}, by the induction hypothesis we know that for all
  $\delta \in \sem{\Delta}$, and $(\eta, \eta') \in
  \sem{\Gamma}\delta$, $(\sem{f}\eta, \sem{f}\eta') \in \sem{A \to
    B}\delta$ and $(\sem{e}\eta, \sem{e}\eta') \in
  \sem{A}\delta$. Thus we can conclude that there exist $d_f, d'_f$
  such that $\sem{f}\eta = \semCons{Lam}\ d_f$ and $\sem{f}\eta' =
  \semCons{Lam}\ d'_f$ and $(d_f(\sem{e}\eta), d'_f(\sem{e}\eta')) \in
  \sem{B}\delta$, as required.

  For \TirName{Pair}, the induction hypothesis says that
  $(\sem{e_1}\eta, \sem{e_1}\eta') \in \sem{A}\delta$ and
  $(\sem{e_2}\eta, \sem{e_2}\eta') \in \sem{B}\delta$, so we can conclude
  that $(\semCons{Pair}(\sem{e_1}\eta, \sem{e_2}\eta),
  \semCons{Pair}(\sem{e_1}\eta', \sem{e_2}\eta')) \in \sem{A \times
    B}\delta$.

  For \TirName{Fst} and \TirName{Snd}, by the induction hypothesis, we
  know that $(\sem{e}\eta, \sem{e}\eta') \in \sem{A \times B}\delta$,
  therefore there exist $d_1,d_2,d'_1,d'_2$ such that $\sem{e}\eta =
  \semCons{Pair}(d_1, d_2)$, $\sem{e}\eta' =
  \semCons{Pair}(d'_1,d'_2)$ and $(d_1,d'_1) \in \sem{A}\delta$ and
  $(d_2,d'_2) \in \sem{B}\delta$. Thus in both cases we are done.

  For \TirName{Inl}, the induction hypothesis says that $(\sem{e}\eta,
  \sem{e}\eta') \in \sem{A}\delta$, so it follows immediately that
  $(\sem{\kw{inl}\ e}\eta, \sem{\kw{inr}\ e}\eta') = (\semCons{Inl}\
  (\sem{e}\eta), \semCons{Inr}\ (\sem{e}\eta')) \in \sem{A +
    B}\delta$. The \TirName{Inr} case is similar.

  For \TirName{Case}, by the induction hypothesis, we know that
  $(\sem{e}\eta, \sem{e}\eta') \in \sem{A + B}\delta$. So either there
  are $(d,d') \in \sem{A}\delta$ such that $\sem{e}\eta =
  \semCons{Inl}\ d$ and $\sem{e}\eta' = \semCons{Inl}\ d'$ or there
  are $(d,d') \in \sem{B}\delta$ such that $\sem{e}\eta =
  \semCons{Inr}\ d$ and $\sem{e}\eta' = \semCons{Inr}\ d'$. In the
  first case, by the induction hypothesis, we know that
  $(\sem{f}(\eta[\ident{x} \mapsto d]), \sem{f}(\eta'[\ident{x}
  \mapsto d'])) \in \sem{C}\delta$, as required. The $\mathsf{Inr}$
  case is similar.

  For \TirName{DePure}, we know that $(\sem{e}\eta, \sem{e}\eta') \in
  \sem{A}\delta$. Since $\sem{A}\delta \subseteq \sem{\delay\kappa
    A}\delta$ by Kripke monotonicity and the definition of the
  semantic type operator $\delay\kappa-$, we have that
  $(\sem{\kw{pure}\ e}\eta, \sem{\kw{pure}\ e}\eta') \in
  \sem{\delay\kappa A}\delta$.

  For \TirName{DeApp}, we know that $(\sem{f}\eta,\sem{f}\eta') \in
  \sem{\delay\kappa(A \to B)}\delta$ and $(\sem{a}\eta,\sem{a}\eta')
  \in \sem{\delay\kappa A}\delta$. If $\delta(\kappa) = 0$ then it
  doesn't matter what $\sem{f}\eta$ or $\sem{f}\eta'$ are, since
  $\sem{\delay\kappa B}(\delta[\kappa\mapsto 0]) = \top_D$. If
  $\delta(\kappa) = n + 1$, then $(\sem{f}\eta, \sem{f}\eta') \in
  \sem{A \to B}(\delta[\kappa \mapsto n])$ and $(\sem{e}\eta,
  \sem{e}\eta') \in \sem{A}(\delta[\kappa \mapsto n])$. Then, by the
  same argument above for the \TirName{App} case, $(\sem{f \circledast
    e}\eta, \sem{f \circledast e}\eta') \in \sem{A}(\delta[\kappa
  \mapsto n]) = \sem{\delay\kappa A}\delta$, as required.

  For \TirName{$\kappa$-Abs}, we know that for all $\delta \in
  \sem{\Delta, \kappa}$ and $(\eta,\eta') \in \sem{\Gamma}\delta$,
  $(\sem{e}\eta, \sem{e}\eta') \in \sem{A}\delta$. So if we are given
  $\delta' \in \sem{\Delta}$, and $(\eta, \eta') \in
  \sem{\Gamma}\delta'$, then for all $n \in \mathbb{N}$,
  $\delta'[\kappa \mapsto n] \in \sem{\Delta, \kappa}$, and
  $(\eta,\eta') \in \sem{\Gamma}(\delta'[\kappa \mapsto n])$ (since
  $\kappa \not\in \mathit{fv}(\Gamma)$) and therefore $(\sem{e}\eta,
  \sem{e}\eta') \in \sem{A}(\delta[\kappa \mapsto n])$. So
  $(\sem{\Lambda \kappa. e}\eta, \sem{\Lambda \kappa. e}\eta') =
  (\sem{e}\eta, \sem{e}\eta') \in \sem{\forall \kappa.A}\delta$, as
  required.

  For \TirName{$\kappa$-App}, we know that for all $n \in \mathbb{N}$,
  $(\sem{e}\eta, \sem{e}\eta') \in \sem{A}(\delta[\kappa \mapsto n])$,
  so certainly $(\sem{e}\eta, \sem{e}\eta') \in \sem{A}(\delta[\kappa
  \mapsto \delta(\kappa')])$. Hence, by
  \lemref{lem:substitution-lemma}, $(\sem{e}\eta, \sem{e}\eta') \in
  \sem{A[\kappa\mapsto\kappa']}\delta$, as required.

  For \TirName{TyEq}, we know that $(\sem{e}\eta, \sem{e}\eta') \in
  \sem{A}\delta$, but by \lemref{lem:type-equality}, $\Delta \vdash A
  \equiv B$ implies that $\sem{A}\delta = \sem{B}\delta$, so
  $(\sem{e}\eta, \sem{e}\eta') \in \sem{B}\delta$, as required.

  For \TirName{Fix}, we need to show that if $(\semCons{Lam}\ f,
  \semCons{Lam}\ f') \in \sem{\delay\kappa A \to A}\delta$, then
  $(\mathrm{fix}\ f, \mathrm{fix}\ f') \in \sem{A}\delta$. Let $m =
  \delta(\kappa)$. We know that for all $n \leq m$, $(x,x') \in
  \sem{\delay\kappa A}(\delta[\kappa \mapsto n])$ implies $(f(x),
  f'(x')) \in \sem{A}(\delta[\kappa \mapsto n])$. We show that
  $(\mathrm{fix}\ f, \mathrm{fix}\ f') \in \sem{A}(\delta[\kappa
  \mapsto n])$ for all $n \leq m$ by induction on $n$, which implies
  our desired result. When $n = 0$, we know that $(\mathrm{fix}\ f,
  \mathrm{fix}\ f') \in \top_D = \sem{\delay\kappa A}(\delta[\kappa
  \mapsto 0])$, so $(f\ (\mathrm{fix}\ f), f'\ (\mathrm{fix}\ f')) \in
  \sem{A}(\delta[\kappa \mapsto 0])$. Since $f\ (\mathrm{fix}\ f) =
  \mathrm{fix}\ f$, and likewise for $f'$, we have $(\mathrm{fix}\ f,
  \mathrm{fix}\ f') \in \sem{A}(\delta[\kappa \mapsto 0])$, as
  required. When $n = n' + 1$ (and $n \leq m$), we know that
  $(\mathrm{fix}\ f, \mathrm{fix}\ f') \in \sem{A}(\delta[\kappa
  \mapsto n'])$ by the induction hypothesis, which implies that
  $(\mathrm{fix}\ f, \mathrm{fix}\ f') \in \sem{\delay\kappa
    A}(\delta[\kappa \mapsto n])$. Thus $(f\ (\mathrm{fix}\ f), f'\
  (\mathrm{fix}\ f')) \in \sem{A}(\delta[\kappa \mapsto n]$, by the
  fact that $f$ and $f'$ are related. Hence $(\mathrm{fix}\ f,
  \mathrm{fix}\ f') \in \sem{A}(\delta[\kappa \mapsto n])$, as
  required.

  For \TirName{Force}, we know that for all $\delta \in \sem{\Delta}$
  and $(\eta,\eta') \in \sem{\Gamma}\delta$, $\forall n.\
  (\sem{e}\eta,\sem{e}\eta') \in \sem{\delay\kappa A}(\delta[\kappa
  \mapsto n])$. So for any $n$, $(\sem{e}\eta,\sem{e}\eta') \in
  \sem{A}(\delta[\kappa \mapsto n])$ since $\sem{\delay\kappa
    A}(\delta[\kappa\mapsto n+1]) = \sem{A}(\delta[\kappa\mapsto
  n])$. Hence $(\sem{\kw{force}\ e}\eta, \sem{\kw{force}\ e}\eta') \in
  \sem{\forall \kappa. A}\delta$.

  For \TirName{Cons}, we use the fact that $\mu F = F(\mu F)$, since
  it is a fixpoint. By the induction hypothesis we have $(\sem{e}\eta,
  \sem{e}\eta') \in \sem{F[\mu X. F]}\delta$. Since $\sem{F[\mu X. F]}
  = \sem{F}(\mu \sem{F}) = \mu \sem{F} = \sem{\mu X. F}$, and
  $(\sem{\cons{Cons}\ e}\eta, \sem{\cons{Cons}\ e}\eta') =
  (\sem{e}\eta, \sem{e}\eta')$, we are done. For \TirName{PrimRec}, we
  use \lemref{lem:primrec-well-typed} and
  \lemref{lem:sem-fmap-well-typed}.
\end{proof}


%%% Local Variables:
%%% TeX-master: "paper"
%%% End:



\end{document}
