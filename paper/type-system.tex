\section{A type system with clocks and guards}
\label{sec:type-system}

In the introduction, we presented our motivating examples in terms of
a Haskell-like language informally extended with a Nakano-style guard
modality and clock quantification. To formally state the properties of
our combination of clock quantification, clock-indexed guard
modalities and inductive types, in this section we define an extension
of the simply-typed $\lambda$-calculus with these features. In the
next section, we define a semantics for our system in which we can
formally state the results that we claimed in the introduction.

\subsection{Well-formed types with clock variables}
\label{sec:types}

\begin{figure}[t]
  \centering
  \begin{mathpar}
    \inferrule*
    {X \in \Theta}
    {\Delta; \Theta \vdash X : \sortType}
    
    \inferrule*
    { }
    {\Delta; \Theta \vdash 1 : \sortType}
    
    \inferrule*
    {\Delta; \Theta \vdash A : \sortType \\ \Delta; \Theta \vdash B : \sortType}
    {\Delta; \Theta \vdash A \times B : \sortType}
    
    \inferrule*
    {\Delta; \Theta \vdash A : \sortType \\ \Delta; \Theta \vdash B : \sortType}
    {\Delta; \Theta \vdash A + B : \sortType}
    
    \inferrule*
    {\Delta; - \vdash A : \sortType \\ \Delta; \Theta \vdash B : \sortType}
    {\Delta; \Theta \vdash A \to B : \sortType}
    
    \inferrule*
    {\Delta; \Theta, X \vdash A : \sortType}
    {\Delta; \Theta \vdash \mu X. A : \sortType}

    \inferrule*
    {\Delta, \kappa; \Theta \vdash A : \sortType}
    {\Delta; \Theta \vdash \forall \kappa. A : \sortType}

    \inferrule*
    {\Delta; \Theta \vdash A : \sortType \\ \kappa \in \Delta}
    {\Delta; \Theta \vdash \delay\kappa A : \sortType}
  \end{mathpar}
  \caption{Well-formed types and type operators}
  \label{fig:types}
\end{figure}

The types of our system include two kinds of variable that may occur
free and bound: clock variables in the clock quantifier and the
clock-indexed guard modality, as well as type variables occuring in
strictly positive positions for the inductive types. We therefore
formally define when a type is well-formed with respect to contexts of
clock and type variables, using the rules displayed in
\autoref{fig:types}.

The well-formedness judgement for types ($\Delta; \Theta \vdash A :
\sortType$) is defined with respect to a \emph{clock context} $\Delta$
and a \emph{type variable context} $\Theta$. Clock contexts are lists
of clock variables, $\Delta = \kappa_1, ..., \kappa_n$, and type
variable contexts are lists of type variable names, $\Theta = X_1,
..., X_n$. Type variables are only used to manage the scope of the
nested least fixpoint type operator $\mu$; there is no type
polymorphism in the type system we present here. We generally use
Roman letters $A$, $B$, $C$ to stand for types, but we also
occasionally use $F$ and $G$ when we wish to emphasise that types with
free type variables are strictly positive type operators. For any type
$A$, we define $\mathit{fc}(A)$ to be the set of free clock variables
that appear in $A$.

Most of the type well-formedness rules are standard: we have a rule
for forming a type from a type variable that is in scope and rules for
forming the unit type $1$, product types $A \times B$, sum types $A +
B$, function types $A \to B$ and least fixpoint types $\mu X. A$. We
enforce the restriction to strictly positive type operators by
disallowing occurences of free type variables in the domain component
of function types. In \autoref{sec:typed-terms}, we will see that each
of the standard type constructors will have the usual corresponding
term-level constructs; our system subsumes the simply typed
$\lambda$-calculus with products, sums and least fixpoint types.

The remaining two rules are for forming clock quantification $\forall
\kappa.A$ types and the clock-indexed Nakano-style guard modality
$\delay\kappa A$. The type formation rule for clock quantification is
very similar to universal type quantification from standard
polymorphic type theories like System F. 

As we described in the introduction in \autoref{sec:clock-vars}, we
have augmented the Nakano-style delay modality with a clock
variable. This allows us to distinguish between guardedness with
respect to multiple clocks, and hence to be able to close over all the
steps guarded by a particular clock.

\subsection{Type Equality}
\label{sec:type-equality}

The examples we presented in the introduction made use of several type
equalities allowing us to move clock quantification through types. For
instance, in the definition of the $\elim{deStreamCons}$ stream
deconstructor in \autoref{sec:clock-vars}, we made use of a type
equality to treat a value of type $\forall \kappa. \tyname{Integer}$
as having the type $\tyname{Integer}$. Intuitively, this is valid
because a value of type $\tyname{Integer}$ exists independently of any
clock, therefore we can remove the clock quantification. In general,
clock quantification commutes with almost all strictly positive type
formers, except for guards indexed by the same clock variable. For
those, we must use $\kw{force}$.

The following rules are intended to be read as defining a judgement
$\Delta; \Theta \vdash A \equiv B : \sortType$, indicating that the
well-formed types $\Delta; \Theta \vdash A : \sortType$ and $\Delta;
\Theta \vdash B : \sortType$ are to be considered equal. In addition
to these rules, the relation $\Delta \vdash A \equiv B : \sortType$ is
reflexive, symmetric, transitive and a congruence.
\begin{displaymath}
  \begin{array}{r@{\hspace{0.3em}}c@{\hspace{0.3em}}ll}
    \forall \kappa. A & \equiv & A & (\kappa \not\in \mathit{fc}(A)) \\
    \forall \kappa. A + B & \equiv & (\forall \kappa. A) + (\forall \kappa. B) \\
    \forall \kappa.  A \times B & \equiv & (\forall \kappa. A) \times (\forall \kappa. B) \\
    \forall \kappa. A \to B & \equiv & A \to \forall \kappa. B & (\kappa \not\in \mathit{fc}(A)) \\
    \forall \kappa. \forall \kappa'. A & \equiv & \forall \kappa'. \forall \kappa. A & (\kappa \not= \kappa') \\
    \forall \kappa. \delay\kappa' A & \equiv & \delay\kappa'\forall \kappa. A & (\kappa \not= \kappa') \\
    \forall \kappa. \mu X. F[A,X] & \equiv & \mu X. F[\forall \kappa. A,X] & (\mathit{fc}(F) = \emptyset)
  \end{array}
\end{displaymath}
In the last rule, $F$ must be strictly positive in $A$ as well as $X$.

After we define the terms of our system in the next sub-section, it
will become apparent that several of the type equality rules could be
removed, and their use replaced by terms witnessing the conversions
back and forth. In the case of product types, we could define the
following pair of terms (using the syntax and typing rules we define
below):
\begin{displaymath}
  \begin{array}{l}
    \lambda x. (\Lambda \kappa. \mathbf{fst}~(x[\kappa]), \Lambda \kappa. \kw{snd}~(x[\kappa])) \\
    \hspace{3cm} : (\forall \kappa. A \times B) \to (\forall \kappa. A) \times (\forall \kappa. B)
  \end{array}
\end{displaymath}
and
\begin{displaymath}
  \begin{array}{l}
    \lambda x. \Lambda \kappa. (\kw{fst}~x~[\kappa], \kw{snd}~x~[\kappa]) \\
    \hspace{3cm} : (\forall \kappa. A) \times (\forall \kappa. B) \to \forall \kappa. A \times B
  \end{array}
\end{displaymath}
Indeed, the denotational semantics we present in
\autoref{sec:semantics} will interpret both of these functions as the
identity. However, not all the type equality rules are expressible in
terms of clock abstraction and application, for instance, the $\forall
\kappa. A \equiv A$ rule and the rule for sum types. Moreover, our
definition of $\elim{deCons}$ in \autoref{sec:main-results-intro} is
greatly simplified by having clock quantification commute with all
type formers uniformly (recall that we made crucial use of the
equality $\forall \kappa. F[-] \equiv F[\forall
\kappa. -]$). Therefore, we elect to include type equalities for
commuting clock quantification with all type formers uniformly.

\subsection{Well-typed Terms}
\label{sec:typed-terms}

\begin{figure*}[t]
  \centering
  \textbf{1. Simply-typed $\lambda$-calculus with products and sums}
  \begin{mathpar}
    \inferrule* [right=Var]
    {\ident{x} : A \in \Gamma}
    {\Delta; \Gamma \vdash \ident{x} : A}

    \inferrule* [right=Unit]
    { }
    {\Delta; \Gamma \vdash * : 1}

    \inferrule* [right=Abs]
    {\Delta; \Gamma, \ident{x} : A \vdash e : B}
    {\Delta; \Gamma \vdash \lambda \ident{x}.\ e : A \to B}

    \inferrule* [right=App]
    {\Delta; \Gamma \vdash f : A \to B \\
      \Delta; \Gamma \vdash e : A}
    {\Delta; \Gamma \vdash f e : B}

    \inferrule* [right=Pair]
    {\Delta; \Gamma \vdash e_1 : A \\
      \Delta; \Gamma \vdash e_2 : B}
    {\Delta; \Gamma \vdash (e_1, e_2) : A \times B}

    \inferrule* [right=Fst]
    {\Delta; \Gamma \vdash e : A \times B}
    {\Delta; \Gamma \vdash \kw{fst}\ e : A}

    \inferrule* [right=Snd]
    {\Delta; \Gamma \vdash e : A \times B}
    {\Delta; \Gamma \vdash \kw{snd}\ e : B}

    \inferrule* [right=Inl]
    {\Delta; \Gamma \vdash e : A}
    {\Delta; \Gamma \vdash \kw{inl}\ e : A + B}

    \inferrule* [right=Inr]
    {\Delta; \Gamma \vdash e : B}
    {\Delta; \Gamma \vdash \kw{inr}\ e : A + B}

    \inferrule* [right=Case]
    {\Delta; \Gamma \vdash e : A + B \\
      \Delta; \Gamma, \ident{x} : A \vdash f : C \\
      \Delta; \Gamma, \ident{y} : B \vdash g : C}
    {\Delta; \Gamma \vdash
      \kw{case}\ e\ \kw{of}\ \kw{inl}\ \ident{x}. f;
      \kw{inr}\ \ident{y}. g : C}
  \end{mathpar}

  \bigskip
  
  \textbf{2. Least Fixpoint Types}
  \begin{mathpar}
    \inferrule* [right=Cons]
    {\Delta; \Gamma \vdash e : F[\mu X. F[X]]}
    {\Delta; \Gamma \vdash \cons{Cons$_F$}\ e : \mu X. F[X]}

    \inferrule* [right=PrimRec]
    {\Delta; \Gamma \vdash e : F[(\mu X. F[X]) \times A] \to A}
    {\Delta; \Gamma \vdash \elim{primRec$_F$}\ e : \mu X. F[X] \to A}
  \end{mathpar}

  \bigskip

  \textbf{3. Clock Abstraction and Application, and Type Equality}
  \begin{mathpar}
    \inferrule* [right=$\kappa$-Abs]
    {\Delta, \kappa; \Gamma \vdash e : A \\ \kappa \not\in \mathit{fc}(\Gamma)}
    {\Delta; \Gamma \vdash \Lambda \kappa. e : \forall \kappa. A}

    \inferrule* [right=$\kappa$-App]
    {\Delta; \Gamma \vdash e : \forall \kappa. A \\ \kappa' \in \Delta \\ \kappa' \not\in \mathit{fc}(A)}
    {\Delta; \Gamma \vdash e[\kappa'] : A[\kappa \mapsto \kappa']}
      
    \inferrule* [right=TyEq]
    {\Delta; \Gamma \vdash e : A \\ \Delta \vdash A \equiv B}
    {\Delta; \Gamma \vdash e : B}
  \end{mathpar}

  \bigskip

  \textbf{4. Applicative Functor Structure for $\delay\kappa-$}
  \begin{mathpar}
    \inferrule* [right=DePure]
    {\Delta; \Gamma \vdash e : A \\ \kappa \in \Delta}
    {\Delta; \Gamma \vdash \kw{pure}\ e : \delay\kappa A}
    
    \inferrule* [right=DeApp]
    {\Delta; \Gamma \vdash f : \delay\kappa (A \to B) \\
      \Delta; \Gamma \vdash e : \delay\kappa A}
    {\Delta; \Gamma \vdash f \circledast e : \delay\kappa B}
  \end{mathpar}

  \bigskip
  \textbf{5. Fix and Force}
  \begin{mathpar}
    \inferrule* [right=Fix]
    {\Delta; \Gamma \vdash f : \delay\kappa A \to A}
    {\Delta; \Gamma \vdash \kw{fix}\ f : A}

    \inferrule* [right=Force]
    {\Delta; \Gamma \vdash e : \forall \kappa. \delay\kappa A}
    {\Delta; \Gamma \vdash \kw{force}\ e : \forall \kappa. A}
  \end{mathpar}
  \caption{Well-typed terms}
  \label{fig:terms}
\end{figure*}

The terms of our system are defined by the following grammar:
\begin{displaymath}
  \begin{array}{l@{\hspace{0.3em}}r@{\hspace{0.3em}}l}
    e,f,g &::=& x \sepbar * \sepbar \lambda x.~e \sepbar f e \sepbar (e_1,e_2) \sepbar \kw{fst}~e \sepbar \kw{snd}~e \\
    & |& \kw{inl}~e \sepbar \kw{inr}~e \sepbar \kw{case}~e~\kw{of}~\kw{inl}~x.f;\kw{inr}~y.g \\
    & |& \cons{Cons}_F~e \sepbar \elim{primRec}_F~e \sepbar \Lambda \kappa.e \sepbar e[\kappa'] \\
    & |& \kw{pure}~x \sepbar f \circledast e \sepbar \kw{fix}~f \sepbar \kw{force}~e
  \end{array}
\end{displaymath}

The \emph{well-typed} terms of our system are defined by the typing judgement
$\Delta; \Gamma \vdash e : A$, given by the rules presented in
\autoref{fig:terms}. Terms are judged to be well-typed with respect to
a clock variable context $\Delta$, and a typing context $\Gamma$
consisting of a list of variable : type pairs: $x_1 : A_1, ..., x_n :
A_n$. We only consider typing contexts where each type is well-formed
with respect to the clock variable context $\Delta$, and the
\emph{empty} type variable context. The result type $A$ in the typing
judgement must also be well-formed with respect to $\Delta$ and the
empty type variable context.

We have split the typing rules in \autoref{fig:terms} into five
groups. The first group includes the standard typing rules for the
simply-typed $\lambda$-calculus with unit, product and sum
types. Apart from the additional clock variable contexts $\Delta$,
these rules are entirely standard. The second group contains the rules
for the inductive types $\mu X. F$. Again, apart from the appearance
of clock variable contexts, these rules are the standard constructor
and primitive recursion constructs for inductive types.

The third group of rules cover clock abstraction and application, and
the integration of the type equality rules we defined in the previous
section. In the \TirName{$\kappa$-App} rule, we have used the notation
$A[\kappa \mapsto \kappa']$ to indicate the substitution of the clock
variable $\kappa'$ for $\kappa$. These rules are as one might expect
for System F-style quantification with non-trivial type equality,
albeit that in the \TirName{$\kappa$-App} rule the $\kappa'$ clock
variable cannot already appear in the type. This disallows us from
using the same clock variable twice, preventing multiple
``time-streams'' becoming conflated.

The fourth group of rules state that the clock-indexed guard modality
supports the $\kw{pure}$ and apply $(\mathord\circledast)$ operations
of an applicative functor. When we define a denotational semantics of
our calculus in the next section, these operations will be interpreted
simply as the identity function and normal function application
respectively, meaning that the applicative functor laws hold
trivially.

Finally, the fifth group of rules presents the typing for Nakano's
guarded $\kw{fix}$ combinator, now indexed by a clock variable, and
our $\kw{force}$ combinator for removing the guard modality when it is
protected by a clock quantification.

\subsection{Type operators are functorial}
\label{sec:functorial}

\newcommand{\ora}[1]{\overrightarrow{#1}}

\begin{figure}[t]
  \centering
\begin{displaymath}
  \begin{array}{l@{}c@{}c@{}c@{}c@{}l}
    \ident{fmap}_F : \overrightarrow{(A \to B)} \to F[\overrightarrow{A}] \to F[\overrightarrow{B}] \\
    \begin{array}{@{}l@{}l@{}l@{\hspace{0.3em}}c@{\hspace{0.3em}}l}
    \ident{fmap}_{X_i}&\vec{f}&x & = & f_i~x \\
    \ident{fmap}_1    &\vec{f}&x & = & * \\
    \ident{fmap}_{F \times G}&\vec{f}&x & = & (\ident{fmap}_F~\vec{f}~(\kw{fst}~x), \ident{fmap}_G~\vec{f}~(\kw{snd}~x)) \\
    \ident{fmap}_{F + G}&\vec{f}&x & = & \kw{case}~x~\kw{of}~
    \begin{array}[t]{@{}l@{}l}
      \kw{inl}~y. & \kw{inl}~(\ident{fmap}_F~\vec{f}~y) \\
      \kw{inr}~z. & \kw{inr}~(\ident{fmap}_G~\vec{f}~z)
    \end{array}\\
    \ident{fmap}_{A\to F}&\vec{f}&x&=&\lambda a.~\ident{fmap}_F~\vec{f}~(x~a) \\
    \ident{fmap}_{\delayX\kappa F}&\vec{f}&x&=&\kw{pure}~(\ident{fmap}_F~\vec{f}) \circledast x \\
    \ident{fmap}_{\forall\kappa.F}&\vec{f}&x&=&\Lambda \kappa. \ident{fmap}_F~\vec{f}~(x[\kappa]) \\
    \ident{fmap}_{\mu X. F}&\vec{f}&x&=&\elim{primRec}~(\lambda x. \cons{Cons}(\ident{fmap}_F~\vec{f}~(\lambda x.~\kw{snd}~x)~x))~x      
    \end{array}
  \end{array}
\end{displaymath}  
  \caption{$\ident{fmap}_F$ for all type operators $F$}
  \label{fig:fmap-syn}
\end{figure}

In \autoref{sec:main-results-intro} we used a functorial mapping
function $\ident{fmap}_F : (A \to B) \to F[A] \to F[B]$, which we
claimed was defined for each strictly positive type operator $F$. We
define this operator by structural recursion on the derivation of $F$,
using the clauses in \autoref{fig:fmap-syn}. Due to the presence of
nested least fixpoint types in our calculus, we handle $n$-ary
strictly positive type operators.

\subsection{Programming with guarded types and clocks}
\label{sec:examples}

\paragraph{Lists and Trees, Finite and Infinite} Using the least
fixpoint type operator $\mu X. F$, we can reconstruct many of the
usual inductive data types used in functional programming. For
example, the OCaml declaration:
\begin{displaymath}
  \kw{type}~\tyname{list} = \cons{NilList} \mathrel| \cons{ConsList}~\kw{of}~A\times\tyname{list}
\end{displaymath}
of finite lists of values of type $A$, can be expressed as the type
$\mu X. 1 + A \times X$ in our system, where we have replaced the two
constructors $\cons{NilList}$ and $\cons{ConsList}$ with a use of the
sum type former. Likewise, the type of binary trees labelled with $A$s
that we used in \autoref{sec:repmin} can be written as $\mu X. A + X
\times X$.

A similar data type declaration in Haskell has a different
interpretation. If we make the following declaration in Haskell:
\begin{displaymath}
  \kw{data}~\tyname{List} = \cons{NilList} \mathrel| \cons{ConsList}~A~\tyname{List}
\end{displaymath}
then the type $\tyname{List}$ includes both finite lists of $A$s, and
infinite lists of $A$s (a similar effect can be achieved in OCaml by
use of the $\kw{lazy}$ type constructor). Haskell's type system is too
weak to distinguish the definitely finite from the possibly infinite
case, and it is often left to the documentation to warn the programmer
that some functions will be non-productive on infinite lists (for
example, the $\ident{reverse}$ function).

Making use of Nakano's guard modality $\delay\kappa-$, we are able to
express Haskell-style possibly infinite lists as the guarded inductive
type $\mu X. 1 + A \times \delay\kappa A$. As we saw in the
introduction, infinite lists can be constructed using the guarded
$\kw{fix}$ operator. For example, the infinite repetition of a fixed
value can be written in our system as:
\begin{displaymath}
  \lambda a. \kw{fix}~(\lambda l. \cons{Cons}~(\kw{inr}~(a,l)))
\end{displaymath}
If we restrict ourselves to only one clock variable, and always use
guarded recursive types, then we obtain a programming language similar
to a monomorphic Haskell, except with a guarantee that all functions
are productive.

\paragraph{Co-inductive Streams} As we saw in
\autoref{sec:infinite-to-finite}, programming with guarded types is
productive, but they are fundamentally incompatible with normal
inductive types. Recall that if we define infinite streams of $A$s as
$\tyname{Stream}^\kappa~A = \mu X. A \times \delay\kappa X$, then
there is no way of defining a function $\ident{tail} :
\tyname{Stream}^\kappa~A \to \tyname{Stream}^\kappa~A$; the best we
can do is $\ident{tail} : \tyname{Stream}^\kappa~A \to
\delay\kappa\tyname{Stream}^\kappa~A$. The rest of the stream only
exists ``tomorrow''.

Clock quantification fixes this problem. We define:
\begin{displaymath}
  \tyname{Stream}~A = \forall \kappa. \mu X. A \times \delay\kappa A
\end{displaymath}
Now we can define the two deconstructors for making observations on
streams, with the right types:
\begin{displaymath}
  \begin{array}{@{}l}
    \begin{array}{@{}l@{\hspace{0.3em}}c@{\hspace{0.3em}}l}
      \defn{head} &:& \tyname{Stream}~A \to A \\
      \defn{head} &=& \lambda s. \Lambda \kappa. \elim{primRec}~(\lambda x. \kw{fst}~x)~(s[\kappa]) \\
    \end{array}\\
    \\
    \begin{array}{@{}l@{\hspace{0.3em}}c@{\hspace{0.3em}}l}
      \defn{tail} &:& \tyname{Stream}~A \to \tyname{Stream}~A \\
      \defn{tail} &=& \lambda s. (\kw{force}~(\Lambda \kappa. \elim{primRec}~
      \begin{array}[t]{@{}l}
        (\lambda x. \kw{pure}(\lambda x. \kw{fst}~x) \circledast (\kw{snd}~x)) \\
        (s[\kappa])))
      \end{array}
    \end{array}
  \end{array}
\end{displaymath}
In the definition of $\ident{head}$ we use the fact that $\kappa$
cannot appear in $A$ to apply the first type equality rule from
\autoref{sec:type-equality}.

\paragraph{Stream Processing}

Ghani, Hancock and Pattinson \cite{ghani09representations} define a
type of representations of continuous functions on streams using a
least fixpoint type nested within a greatest fixpoint. A continuous
function on streams is a function that only requires a finite amount
of input for each element of the output. Ghani \etal's type is
expressed in our system as follows:
\begin{displaymath}
  \tyname{SP}~I~O = \forall \kappa. \mu X. \mu Y. (I \to Y) + (O \times \delay\kappa X)
\end{displaymath}
where $\tyname{SP}$ stands for ``Stream Processor'' and $\tyname{I}$
and $\tyname{O}$ stand for input and output respectively. In this
type, the outer fixpoint permits the production of infinitely many
$O$s, due to the guarded occurence of $X$, while the inner fixpoint
only permits a finite amount of $I$s to be read from the input
stream. This kind of nested fixpoint is not expressible within Coq,
but is possible with Agda's support for coinductive types
\cite{danielsson09mixing}.

Defining the application of an $\tyname{SP}~I~O$ to a stream of $Is$
serves as an interesting example of programming with guards and
clocks. We make use of a helper function for unfolding values of type
$\tyname{SP}~I~O$ that have already been instantiated with some clock:
\begin{displaymath}
  \defn{step} : \ident{SP}^\kappa~I~O \to \mu Y. (I \to Y) \times (O \times \delay\kappa(\ident{SP}^\kappa~I~O))
\end{displaymath}
where $\ident{SP}^\kappa~I~O$ is $\ident{SP}~I~O$ instantiated with
the clock $\kappa$.

The definition of stream processor application goes as follows, where
we permit ourselves a little pattern matching syntactic sugar for
pairs:
\begin{displaymath}
  \begin{array}{@{}l}
    \defn{apply} : \tyname{SP}~I~O \to \tyname{Stream}~I \to \tyname{Stream}~O \\
    \defn{apply} = \\
    \ \begin{array}[t]{@{}l@{}l}
      \lambda \ident{sp}~\ident{s}.\Lambda \kappa. \kw{fix}(\lambda&\ident{rec}~\ident{sp}~\ident{s}. \\
      & \elim{primRec}~(\lambda x~s.
      \begin{array}[t]{@{}l}
        \kw{case}~x~\kw{of} \\
        \quad \kw{inl}~\ident{f}.~\ident{f}~(\ident{head}~\ident{s}) (\ident{tail}~\ident{s}) \\
        \quad \kw{inr}~(\ident{o},\ident{sp'}).\\
        \quad \quad \cons{Cons}(o,\ident{rec} \circledast \ident{sp'} \circledast \kw{pure}~s))
      \end{array} \\
      & \quad (\ident{step}~\ident{sp})) \\
      & (\ident{sp}~[\kappa])~s
    \end{array}
  \end{array}
\end{displaymath}
This function consists of two nested loops. The outer loop, expressed
using $\kw{fix}$, generates the output stream. This loop is running on
the clock $\kappa$, introduced by the $\Lambda \kappa$. We have
instantiated the stream processor with the same clock; this causes the
steps of the output stream to be in lockstep with the output steps of
the stream processor, exactly as we might expect. The inner loop,
expressed using $\elim{primRec}$, recurses over the finite list of
input instructions from the stream processor, pulling items from the
input stream as needed. The input stream is \emph{not} instantiated
with the same clock as the stream processor and the output stream: we
need to access an arbitrary number of elements from the input stream
for each step of the stream processor, so their clocks cannot run in
lockstep.

%\paragraph{Producer-Consumer Contracts} We are grateful to Danielsson
%for the following example of troubled productivity. In Haskell, we
%might accidentally write
%\begin{verbatim}
%nats :: [Integer]
%nats = 0 : maap (1+) nats
%
%maap :: (a -> b) -> [a] -> [b]
%maap f (a : a' : as) = f a : f a' : maap f as
%\end{verbatim}
%This \texttt{maap} is perfectly productive and coincides with
%\texttt{map} on all well defined infinite lists, yet it is too strict 
%to be used productively in \texttt{nats}. We can see that
%\texttt{maap} fails the stricter requirement that only one element
%need be consumed for each element produced. In our setting, the
%definition
%\begin{displaymath}
%  \begin{array}{l}
%    \defn{nats} : \tyname{Stream}~\tyname{Natural} \\
%    \defn{nats} = \Lambda \kappa. \kw{fix}~(\lambda \ident{rec}.
%      \cons{Cons}(\cons{Zero},\kw{pure}~(\defn{map}~[\kappa]~\cons{Suc})
%      \circledast \ident{rec}))
%  \end{array}
%\end{displaymath}
%requires
%\begin{displaymath}
%    \defn{map} : \forall \kappa.(A\to B)\to\tyname{Stream}^\kappa~A\to\tyname{Stream}^\kappa~B
%\end{displaymath}
%which enforces the producer-consumer contract precisely: we cannot
%wait for tomorrow's input to deliver today's output. The weaker
%\begin{displaymath}
%   (A\to B)\to\tyname{Stream}~A\to\tyname{Stream}~B
%\end{displaymath}
%makes no such promise and is thus insufficient to document the
%safe use of $\defn{map}$ as a \emph{component} of productive processes.


\paragraph{The Partiality Monad} The partiality monad is a well-known
coinductively defined monad that allows the definition of functions
via general recursion, even in a total language. The partiality monad
is defined as $\tyname{Partial}~A = \nu X. A + X$. Hence, a value of
type $\tyname{Partial}~A$ consists of a stream of $\kw{inr}$s, either
continuing forever, or terminating in an $\kw{inl}$ with a value of
type $A$. Using the partiality monad, possibly non-terminating
computations can be represented, with non-termination represented by
an infinite stream that never returns a value. Capretta
\cite{capretta05general} gives a detailed description.

To express the partiality monad in our system, we decompose it into a
guarded least fixpoint and clock quantification, just as we did for
streams and stream processors above:
\begin{displaymath}
  \begin{array}{l}
    \tyname{Partial}^\kappa~A = \mu X. A + \delay\kappa X \\
    \tyname{Partial}~A = \forall \kappa. \tyname{Partial}^\kappa~A
  \end{array}
\end{displaymath}
For convenience, we define two constructors for
$\tyname{Partial}^\kappa~A$, corresponding to the two components of
the sum type:
\begin{displaymath}
  \begin{array}{l}
    \defn{now}^\kappa : A \to \tyname{Partial}^\kappa~A \\
    \defn{now}^\kappa = \lambda a.~\cons{Cons}~(\kw{inl}~a) \\
    \\
    \defn{later}^\kappa : \delay\kappa (\tyname{Partial}^\kappa~A) \to \tyname{Partial}^\kappa~A \\
    \defn{later}^\kappa = \lambda p.~\cons{Cons}~(\kw{inr}~p)
  \end{array}
\end{displaymath}
It is straightforward to use guarded recursion and clock
quantification to define the $\ident{return}$ and bind functions that
demonstrate that $\tyname{Partial}$ is indeed a monad. For an example
of a possibly non-terminating function that can be expressed using the
partiality monad, we define the following function
$\ident{collatz}$. For each natural number $n$, this function only
terminates if the Collatz sequence starting from $n$ reaches
$1$. Whether or not this sequence reaches $1$ for all $n$ is a famous
unsolved question in number theory.
\begin{displaymath}
  \begin{array}{@{}l}
    \defn{collatz} : \tyname{Natural} \to \tyname{Partial}~1 \\
    \defn{collatz} = \lambda n. \Lambda \kappa. \kw{fix}~(\lambda \ident{rec}~\ident{n}.
    \begin{array}[t]{@{}l}
      \kw{if}~n = 1~\kw{then}~\ident{now}^\kappa~(*) \\
      \kw{else}~\kw{if}~n~\ident{mod}~2 = 0~\kw{then} \\
      \quad \ident{later}^\kappa~(\ident{rec} \circledast (\kw{pure}~(n/2))) \\
      \kw{else} \\
      \quad \ident{later}^\kappa~(\ident{rec} \circledast (\kw{pure}~(3*n + 1))))~n
    \end{array}
  \end{array}
\end{displaymath}

The partiality monad is not a drop-in replacement for general
recursion. The type $\tyname{Partial}~A$ reveals information about the
number of steps that it takes to compute a value of type
$A$. Therefore, it is possible to write a timeout function that runs a
partial computation for a fixed number of steps before giving up:
\begin{displaymath}
  \defn{timeout} : \tyname{Natural} \to \tyname{Partial}~A \to A + 1
\end{displaymath}
The partiality monad allows us to write possibly non-terminating
functions, but also allows us to make more observations on them.

%%% Local Variables:
%%% TeX-master: "paper"
%%% End:
