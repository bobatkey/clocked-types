% -*- TeX-engine: xetex -*-

\documentclass[xetex,serif,mathserif]{beamer}

\usepackage{stmaryrd}
%\usepackage[all]{xy}
\usepackage{soul}

\usepackage{euler}
\usepackage[cm-default]{fontspec}
\usepackage{xunicode}
\usepackage{xltxtra}

%\usepackage{mathpartir}
\usepackage{cancel}

\newcommand{\delay}[1]{\mathord{\rhd\kern-0.4em^{#1}}}
\newcommand{\sem}[1]{\llbracket #1 \rrbracket}

\setmainfont{Linux Libertine O}

\setbeamertemplate{navigation symbols}{}
\usecolortheme[rgb={0.8,0,0}]{structure}
\usefonttheme{serif}
\usefonttheme{structurebold}
\setbeamercolor{description item}{fg=black}

\definecolor{titlered}{rgb}{0.8,0.0,0.0}
\definecolor{white}{rgb}{1.0,1.0,1.0}

\newenvironment{slide}[1]{\begin{frame}\frametitle{#1}}{\end{frame}}
\newenvironment{verbslide}[1]{\begin{frame}[containsverbatim]\frametitle{#1}}{\end{frame}}
\newcommand{\titlecard}[1]{\begin{frame}\begin{center}\usebeamercolor[fg]{frametitle}\usebeamerfont{frametitle}#1\end{center}\end{frame}}

\def\greyuntil<#1>#2{{\temporal<#1>{\color{black!0}}{\color{black}}{\color{black}} #2}}
\def\greyfrom<#1>#2{{\temporal<#1>{\color{black}}{\color{black!40}}{\color{black!40}} #2}}

\definecolor{highlight}{rgb}{0.8,0.1,0.1}
\definecolor{cons}{rgb}{0.8,0.1,0.2}
\definecolor{elim}{rgb}{0.1,0.5,0.35}
\definecolor{defnd}{rgb}{0.1,0.2,0.8}
\newcommand{\highlight}[1]{\textcolor{highlight}{#1}}
\newcommand{\cons}[1]{\textcolor{cons}{\textsf{#1}}}
\newcommand{\elim}[1]{\textcolor{elim}{\textsf{#1}}}
\newcommand{\defnd}[1]{\mathit{#1}}
\newcommand{\unitSet}{1}
\newcommand{\hlchange}[1]{\colorbox{black!20}{$#1$}}
\newcommand{\hlchangenull}[1]{\colorbox{white}{$#1$}}

\newcommand{\Fam}{\mathrm{Fam}}
\newcommand{\Set}{\mathrm{Set}}
\newcommand{\Alg}{\mathrm{Alg}}
\newcommand{\inn}{\mathit{in}}
\newcommand{\fold}[1]{\llparenthesis #1 \rrparenthesis}
%\newcommand{\tyname}[1]{\texttt{#1}}
\newcommand{\sepbar}{\mathrel|}
\newcommand{\altdiff}[3]{\alt<-#1>{\hlchangenull{#2}}{\hlchange{#3}}}
\newcommand{\altdiffX}[3]{\alt<-#1>{\hlchangenull{\textcolor{white}{#3}}}{\hlchange{#3}}}

\newcommand{\adjunction}[4]{{#1} \ar@/_/[r]_{#4}
    \ar@{}[r]|{\perp}& \ar@/_/[l]_{#3} {#2}}

\newcommand{\pullbackcorner}[1][dr]{\save*!/#1-1.2pc/#1:(-1,1)@^{|-}\restore}

\title{Productive Coprogramming \\ \emph{with} \\ Guarded Recursion}
\author{Robert Atkey \\ {\small \textcolor{black!60}{\textit{@bentnib}}}\\ \ \\ Conor McBride \\ {\small \textcolor{black!60}{\textit{@pigworker}}}}
\date{26th September 2013}

\newcommand{\kw}[1]{\textbf{#1}}
\newcommand{\tyname}[1]{\textit{#1}}
\newcommand{\ident}[1]{\textrm{#1}}
\newcommand{\defn}[1]{\textcolor{defnd}{\textrm{#1}}}

\begin{document}

\frame{\titlepage}

%\titlecard{Playing in the Stream}

\begin{slide}{}
  \only<2,3,4>{\special{pdf: put @thispage <</Trans<</S /Fade /D 0.2>>>>}}

  \begin{displaymath}
    \kw{\altdiffX{1}{}{\kw{co}}data}\ \tyname{Stream} = \cons{StreamCons}\ \tyname{Integer}\ \tyname{Stream}
  \end{displaymath}

  \medskip

  \only<-2>{\ }
  \only<3->{\textcolor{titlered}{\emph{Productivity}}: Each element is produced in finite time}

  \medskip

  \only<-3>{\ }
  \only<4->{\textcolor{titlered}{\emph{Guardedness}}: Each recursive call is “guarded” by a constructor}

  \bigskip

  \begin{displaymath}
    \begin{array}{@{}l@{}}
      \left.
        \begin{array}{@{}l@{}}
          \defn{ones} :: \tyname{Stream} \\
          \defn{ones} = \cons{StreamCons}\ 1\ \ident{ones} \\
          \\
          \defn{map} :: (\tyname{Integer} \to \tyname{Integer}) \to \tyname{Stream} \to \tyname{Stream} \\
          \defn{map}\ \ident{f}\ (\cons{StreamCons}\ \ident{x}\ \ident{xs}) = \cons{StreamCons}\ (\ident{f}\ \ident{x})\ (\ident{map}\ \ident{f}\ \ident{xs}) \\
          % \\
          % \defn{merge} :: \tyname{Stream} \to \tyname{Stream} \to \tyname{Stream} \\
          % \defn{merge}\ (\cons{StreamCons}\ \ident{x}\ \ident{xs})\ (\cons{StreamCons}\ \ident{y}\ \ident{ys}) = \\
          % \quad\quad \cons{StreamCons}\ \ident{x}\ (\cons{StreamCons}\ \ident{y}\ (\ident{merge}\ \ident{xs}\ \ident{ys}))
        \end{array}
        \only<-4>{\right. \hspace{1.538cm}}
      \only<5->{\right\} \textit{guarded}}
      \\
      \\
      \\
      \\
      \left.
        \begin{array}{@{}l}
          \only<6->{\defn{filter} :: (\tyname{Integer} \to \tyname{Bool}) \to \tyname{Stream} \to \tyname{Stream}} \\
          \only<6->{\defn{filter}\ \ident{f}\ (\cons{StreamCons}\ \ident{z}\ \ident{s}) =} \\
          \only<6->{\hspace{0.5cm}\kw{if }\ident{f}\ \ident{z}\kw{ then }\cons{StreamCons}\ \ident{z}\ (\ident{filter}\ \ident{f}\ \ident{s})\kw{ else }\ident{filter}\ \ident{f}\ \ident{s}}
        \end{array}
        \only<-6>{\right.}
        \only<7->{\right\} \textit{unguarded}}
    \end{array}
  \end{displaymath}
\end{slide}

\titlecard{So guardedness $\approx$ productivity?}

\begin{slide}{}
    \begin{displaymath}
    \begin{array}{@{}l}
      \begin{array}{@{}l@{\hspace{0.4em}}l}
        \defn{mergef} ::
        &(\tyname{Integer} \to \tyname{Integer} \to \tyname{Stream} \to \tyname{Stream}) \to \\
        &\tyname{Stream} \to \\
        &\tyname{Stream} \to \\
        &\tyname{Stream} \\
      \end{array} \\
      \defn{mergef}\ \ident{f}\ (\cons{StreamCons}\ \ident{x}\ \ident{xs})\ (\cons{StreamCons}\ \ident{y}\ \ident{ys}) = \\
      \quad\quad \ident{f}\ \ident{x}\ \ident{y}\ (\ident{mergef}\ \ident{f}\ \ident{xs}\ \ident{ys})
    \end{array}
  \end{displaymath}

  \bigskip

  \begin{minipage}[t][4cm]{1.0\linewidth}    
\only<2>{
  A bad choice of argument yields non-productive definitions:

  \begin{displaymath}
    \begin{array}{l}
      \defn{badf} :: \tyname{Integer} \to \tyname{Integer} \to \tyname{Stream} \to \tyname{Stream} \\
      \defn{badf}\ \ident{x}\ \ident{y}\ \ident{s} = \ident{s}
    \end{array}
  \end{displaymath}
}
\only<3>{
  Coq reports:

  \medskip

  \hspace*{1cm}\emph{Sub-expression “}$\ident{f}\ \ident{x}\ \ident{y}\ (\ident{mergef}\ \ident{f}\ \ident{xs}\ \ident{ys})$\emph{” not in guarded form}
}
\only<4>{
  Get around guardedness check by changing the type of $\ident{f}$?
  \begin{displaymath}
    \begin{array}{l}
      \defn{f} :: \tyname{Integer} \to \tyname{Integer} \to \tyname{Integer} \\
      \defn{f} :: \tyname{Integer} \to \tyname{Integer} \to (\tyname{Integer}, [\tyname{Integer}])
    \end{array}
  \end{displaymath}
  but what about:
  \begin{displaymath}
    \begin{array}{l}
      \defn{f}\ \ident{x}\ \ident{y}\ \ident{s} = \cons{StreamCons}\ \ident{x}\ (\ident{map}\ (+1)\ s)
    \end{array}
  \end{displaymath}
}
  \end{minipage}
\pause
\pause
\pause
\end{slide}

% Non-compositional, requires a whole program analysis/rewrite

\begin{frame}
  \only<2>{\special{pdf: put @thispage <</Trans<</S /Fade /D 0.2>>>>}}
  \begin{center}
    {\usebeamercolor[fg]{frametitle}\usebeamerfont{frametitle}Can we put guardedness information in the types?} \\
    \pause
    \textcolor{black!60}{(Nakano, 2000)}
  \end{center}
\end{frame}

\begin{slide}{}
  \only<2>{\special{pdf: put @thispage <</Trans<</S /Fade /D 0.2>>>>}}

  % \begin{displaymath}
  %   \begin{array}{l}
  %     \defn{mergef} :: \\
  %     \quad\quad(\tyname{Integer} \to \tyname{Integer} \to \tyname{Stream} \to \tyname{Stream}) \to \\
  %     \quad\quad\tyname{Stream} \to \\
  %     \quad\quad\tyname{Stream} \\
  %   \end{array}
  % \end{displaymath}

  % \bigskip

  \begin{displaymath}
    \defn{f} :: \tyname{Integer} \to \tyname{Integer} \to \altdiffX{1}{}{\rhd}\tyname{Stream} \to \tyname{Stream}
  \end{displaymath}

  \pause

  % \bigskip

  % Let us add a modality:
  % \begin{displaymath}
  %   \defn{f} :: \tyname{Integer} \to \tyname{Integer} \to \rhd\tyname{Stream} \to \tyname{Stream}
  % \end{displaymath}
% FIXME: fix the layout of this slide  
\end{slide}

\begin{slide}{}
  \only<6>{\special{pdf: put @thispage <</Trans<</S /Fade /D 0.1>>>>}}

  \textcolor{titlered}{\emph{Streams today}}:
  \begin{displaymath}
    \tyname{Stream}
  \end{displaymath}

  \bigskip
  \pause

  \textcolor{titlered}{\emph{Streams tomorrow}}:
  \begin{displaymath}
    \rhd \tyname{Stream}
  \end{displaymath}

  \bigskip
  \pause

  \textcolor{titlered}{\emph{Construction and Deconstruction}}:
  \begin{displaymath}
    \begin{array}{l}
      \cons{StreamCons}   :: \tyname{Integer} \to \rhd\tyname{Stream} \to \tyname{Stream} \\
      \elim{deStreamCons} :: \tyname{Stream} \to (\tyname{Integer}, \rhd\tyname{Stream})    
    \end{array}
  \end{displaymath}

  \bigskip
  \pause

  \textcolor{titlered}{\emph{Combinators}}:
  \begin{displaymath}
    \begin{array}{l@{\hspace{2cm}}l}
      \kw{fix}    :: (\rhd A \to A) \to A
      &
      \begin{array}{l}
        \kw{pure} :: A \to \rhd A \\
        \kw{$(\circledast)$} :: \rhd (A \to B) \to \rhd A \to \rhd B
      \end{array}
    \end{array}
  \end{displaymath}
  
  \pause
  \medskip

  \begin{displaymath}
    \begin{array}{l@{\hspace{2cm}}l}
      \defn{ones} = \kw{fix}\ (\lambda s.\ \cons{StreamCons}\ 1\ \ident{s})
      &
      \altdiff{5}{\kw{fix}\ (\lambda \ident{x}.\ \ident{x})}
      {\cancel{\kw{fix}\ (\lambda \ident{x}.\ \ident{x})}}
    \end{array}
  \end{displaymath}

  \pause
\end{slide}

\begin{slide}{}
  \begin{displaymath}
    \begin{array}{l}
      \begin{array}{@{}l@{\hspace{0.2em}}l}
        \defn{mergef} :: &
        (\tyname{Integer} \to \tyname{Integer} \to \rhd\tyname{Stream} \to \tyname{Stream}) \to \\
        &\tyname{Stream} \to \\
        &\tyname{Stream} \to \\
        &\tyname{Stream}
      \end{array} \\
      \defn{mergef}\ \ident{f} = \kw{fix}\ (\lambda \ident{g}\ \ident{xs}\ \ident{ys}. \\
      \hspace{3cm} \kw{let}\ (\ident{x}, \ident{xs’}) = \elim{deStreamCons}\ \ident{xs} \\
      \hspace{3cm} \textcolor{white}{\kw{let}\ }(\ident{y}, \ident{ys’}) = \elim{deStreamCons}\ \ident{ys} \\
      \hspace{3cm} \kw{in}\ \ident{f}\ \ident{x}\ \ident{y}\ (\ident{g} \circledast \ident{xs’} \circledast \ident{ys’}))
    \end{array}
  \end{displaymath}

  \begin{displaymath}
    g :: \rhd(\tyname{Stream} \to \tyname{Stream} \to \tyname{Stream})
  \end{displaymath}
\end{slide}

\frame{}

% \titlecard{From the Infinite to the Finite}

\begin{slide}{}

  \textcolor{titlered}{\emph{Problem}}:
  Can we write
  \begin{displaymath}
    \defn{take} :: \tyname{Natural} \to \tyname{Stream} \to [\tyname{Integer}]
  \end{displaymath}
  by structural recursion on the first argument?

  \bigskip
  \pause
  \textcolor{titlered}{\emph{Solution attempt}}:
  \begin{displaymath}
    \begin{array}{l}
      \defn{take}\ 0\ \ident{s} = [] \\
      \defn{take}\ (\ident{n}+1)\ \ident{s} = \ident{x} \mathop{\cons{:}} \ident{take}\ \ident{n}\ \alt<-2>{\colorbox{white}{\ident{s’}}}{\colorbox{red!75}{\ident{s’}}} \\
      \quad \kw{where}\ (\ident{x}, \ident{s’}) = \elim{deStreamCons}\ \ident{s}
    \end{array}
  \end{displaymath}
  \pause

  The type of $\elim{deStreamCons}$ prevents us:
  \begin{displaymath}
    \elim{deStreamCons} :: \tyname{Stream} \to (\tyname{Integer}, \rhd\tyname{Stream})    
  \end{displaymath}

  We cannot leave the “time stream” of the stream.
\end{slide}

\begin{frame}
  \begin{displaymath}
    \begin{array}{l@{\hspace{0.2em}}c@{\hspace{0.3em}}l}
      \tyname{Stream} &\only<2->\cong& \only<2->{\tyname{Integer} \times \rhd\tyname{Stream}} \\
      & \only<3->\cong & \only<3->{\tyname{Integer} \times \rhd(\tyname{Integer} \times \rhd\tyname{Stream})} \\
      & \only<4->\cong & \only<4->{\tyname{Integer} \times \rhd(\tyname{Integer} \times \rhd(\tyname{Integer} \times \rhd\tyname{Stream}))} \\
      & \only<5->\cong & \only<5->{\tyname{Integer} \times \rhd(\tyname{Integer} \times \rhd(\tyname{Integer} \times \rhd(\tyname{Integer} \times ...)))} \\
      & & \hspace{10cm}
    \end{array}
  \end{displaymath}

  \pause
  \pause
  \pause
  \pause
  \pause
  \bigskip

  \begin{center}
    \emph{How do we get the whole stream?}
  \end{center}

  \pause
  \begin{center}
    \emph{Hold infinity in the palm of your hand?}
  \end{center}
\end{frame}

\begin{frame}
  \only<2>{\special{pdf: put @thispage <</Trans<</S /Fade /D 0.2>>>>}}
  \begin{center}
    {\usebeamercolor[fg]{frametitle}\usebeamerfont{frametitle}Eternity in an Hour} \\
    \pause
    \textcolor{black!60}{\emph{(or 15 minutes, plus 5 minutes for questions)}}
  \end{center}
\end{frame}

% \begin{frame}
%   \only<2>{\special{pdf: put @thispage <</Trans<</S /Fade /D 0.2>>>>}}
%   \begin{center}
%     {\usebeamercolor[fg]{frametitle}\usebeamerfont{frametitle}\emph{A Solution}: Clock Variables} \\
%     \pause
%     \textcolor{black!60}{\emph{(and quantification over them)}}
%   \end{center}
% \end{frame}

\begin{slide}{}
  \textcolor{titlered}{\emph{Clock Variables}}, $\kappa$ \\
  \begin{itemize}
  \item A clock $\kappa$ represents a fixed amount of time remaining
  \item $\tyname{Stream}^\kappa$ is a stream for at least the time remaining in $\kappa$
  \item $\delay\kappa$ removes one unit of time remaining in $\kappa$
  \end{itemize}
  
  \bigskip
  \pause

  \textcolor{titlered}{\emph{Clocked Construction and Deconstruction}}: \\
  \begin{displaymath}
    \begin{array}{l}
      \cons{StreamCons} :: \forall \kappa.\ \tyname{Integer} \to \delay\kappa \tyname{Stream}^\kappa \to \tyname{Stream}^\kappa \\
      \elim{deStreamCons} :: \forall \kappa.\ \tyname{Stream}^\kappa \to (\tyname{Integer}, \delay\kappa\tyname{Stream}^\kappa)    
    \end{array}
  \end{displaymath}

  \bigskip
  \pause

  \textcolor{titlered}{\emph{Clocked Combinators}}: \\
  \begin{displaymath}
    \begin{array}{l}
      \kw{pure} :: \forall \kappa.\ A \to \delay\kappa A \\
      \kw{$(\circledast)$} :: \forall \kappa.\ \delay\kappa (A \to B) \to \delay\kappa A \to \delay\kappa B \\
      \kw{fix} :: \forall \kappa.\ (\delay\kappa A \to A) \to A
    \end{array}
  \end{displaymath}

\end{slide}

% In these types, we used clock quantification just to make the
% combinators work in multiple contexts. But clock quantification does
% more than that.

\begin{slide}{}
  \textcolor{titlered}{\emph{Quantification over all time}}:\\

  \medskip

  \quad\quad $(\forall \kappa.~A)$ is $A$s will all possible amounts of time remaining.

  % For type $A$ with a free clock variable $\kappa$:
  % \begin{itemize}
  % \item Form the type $\forall \kappa.\ A$
  % \item $A$ with all possible amounts of time remaining
  % \end{itemize}

  \pause
  \bigskip

  \quad\quad$(\forall \kappa.\ \tyname{Stream}^\kappa)$ is the type of streams that go on forever

  \pause
  \bigskip
  \bigskip

  \textcolor{titlered}{\emph{Forcing}}: If we quantify over all time,
  then tomorrow becomes today:
  \begin{displaymath}
    \kw{force} :: (\forall \kappa.\ \delay\kappa A) \to (\forall \kappa.\ A)
  \end{displaymath}

\end{slide}

\begin{slide}{}
  \textcolor{titlered}{\emph{Problem}}:
  Can we write
  \begin{displaymath}
    \defn{take} :: \tyname{Natural} \to (\forall \kappa.\ \tyname{Stream}^\kappa) \to [\tyname{Integer}]
  \end{displaymath}
  by structural recursion on the first argument?

  \bigskip

  \pause
  \textcolor{titlered}{\emph{Solution}}:
  \begin{displaymath}
    \begin{array}{l}
      \defn{take}\ 0\ \ident{s} = [] \\
      \defn{take}\ (\ident{n}+1)\ \ident{s} = \ident{x} \mathop{\cons{:}} \ident{take}\ \ident{n}\ \ident{s’} \\
      \quad \kw{where}\ (\ident{x}, \ident{s’}) = \elim{deStreamCons$^\nu$}\ \ident{s}
    \end{array}
  \end{displaymath}

  \pause
  \begin{displaymath}
    \begin{array}{l}
      \elim{deStreamCons$^\nu$} :: (\forall \kappa. \tyname{Stream}^\kappa) \to (\tyname{Integer}, \forall \kappa. \tyname{Stream}^\kappa) \\
      \elim{deStreamCons$^\nu$}\ \ident{x} = \\
      \quad\quad
      \begin{array}[t]{l}
        (\Lambda \kappa. \kw{fst}\ (\elim{deStreamCons}\ [\kappa]\ (\ident{x}\ [\kappa])),\\
        \quad\kw{force}\ (\Lambda \kappa. \kw{snd}\ (\elim{deStreamCons}\ [\kappa]\ (\ident{x}\ [\kappa]))))
      \end{array}
    \end{array}
  \end{displaymath}
\end{slide}

\frame{}

% \begin{frame}
% %  \only<2>{\special{pdf: put @thispage <</Trans<</S /Fade /D 0.2>>>>}}
%   \begin{center}
%     {\usebeamercolor[fg]{frametitle}\usebeamerfont{frametitle}Guarded Types} \\
    
%     \medskip
%     \pause
%     \emph{with} \\
%     \medskip
%     {\usebeamercolor[fg]{frametitle}\usebeamerfont{frametitle}Clocks} \\

%     \medskip
%     \pause
%     \emph{yields} \\
%     \medskip
%     {\usebeamercolor[fg]{frametitle}\usebeamerfont{frametitle}Productive Programming}
%   \end{center}
% \end{frame}

\begin{slide}{}
  \begin{tabular*}{1.0\textwidth}{@{\extracolsep{\fill}}lr}
    \textcolor{titlered}{\emph{A classic example}}
    &
    \textcolor{black!60}{(Bird, 1984)}
  \end{tabular*}
  
  \begin{displaymath}
    \begin{array}{l}
      \defn{replaceMin} :: \tyname{Tree} \to \tyname{Tree} \\
      \defn{replaceMin}\ \ident{t} = \\
      \hspace{1cm} \kw{let}\ (\ident{t’}, \ident{m}) = \ident{replaceMinBody}\ (\ident{t}, \ident{m})\ \kw{in}\ \ident{t’} \\
      \hspace{0.6cm} \kw{where} \\
      \hspace{1cm} \defn{replaceMinBody}\ (\cons{Leaf}\ \ident{x},\ \ident{m}) = (\cons{Leaf}\ \ident{m}, \ident{x}) \\
      \hspace{1cm} \defn{replaceMinBody}\ (\cons{Br}\ \ident{l}\ \ident{r},\ \ident{m}) = \\
      \hspace{3cm} \kw{let}\ (\ident{l’}, \ident{m$_l$}) = \ident{replaceMinBody}\ \ident{l}\ \ident{m} \\
      \hspace{3cm} \textcolor{white}{\kw{let}\ }(\ident{r’}, \ident{m$_r$}) = \ident{replaceMinBody}\ \ident{r}\ \ident{m} \\
      \hspace{3cm} \kw{in}\ (\cons{Br}\ \ident{l’}\ \ident{r’},\ \ident{min}\ \ident{m$_l$}\ \ident{m$_r$})
    \end{array}
  \end{displaymath}
\end{slide}

\begin{slide}{}
  %\only<2-5>{\special{pdf: put @thispage <</Trans<</S /Fade /D 0.2>>>>}}
  \textcolor{titlered}{\emph{Time Stratification with Clocks}}:
  \begin{displaymath}
    \begin{array}{l}
      \defn{replaceMinBody} :: \altdiffX{1}{}{\forall \kappa.\ }\tyname{Tree} \to \altdiffX{2}{}{\delay\kappa}\tyname{Integer} \to (\altdiffX{3}{}{\delay\kappa}\tyname{Tree},\ \tyname{Integer}) \\
      \hspace{0cm} \defn{replaceMinBody}\ \altdiffX{4}{}{[\kappa]}\ (\cons{Leaf}\ \ident{x})\ \ident{m} = (\altdiffX{4}{}{\ident{pure}}\cons{Leaf}\ \altdiffX{4}{}{\circledast}\ident{m}, \ident{x}) \\
      \hspace{0cm} \defn{replaceMinBody}\ \altdiffX{5}{}{[\kappa]}\ (\cons{Br}\ \ident{l}\ \ident{r}) \ident{m} = \\
      \hspace{2cm} \kw{let}\ (\ident{l’}, \ident{m$_l$}) = \ident{replaceMinBody}\ \altdiffX{5}{}{[\kappa]}\ \ident{l}\ \ident{m} \\
      \hspace{2cm} \textcolor{white}{\kw{let}\ }(\ident{r’}, \ident{m$_r$}) = \ident{replaceMinBody}\ \altdiffX{5}{}{[\kappa]}\ \ident{r}\ \ident{m} \\
      \hspace{2cm} \kw{in}\ (\altdiffX{5}{}{\ident{pure}}\cons{Br}\ \altdiffX{5}{}{\circledast}\ident{l’}\ \altdiffX{5}{}{\circledast}\ident{r’},\ \ident{min}\ \ident{m$_l$}\ \ident{m$_r$})
    \end{array}
  \end{displaymath}
  \pause
  \pause
  \pause
  \pause
  \pause
  \pause
  \textcolor{titlered}{\emph{Closing the circle, safely}}:
%   \begin{displaymath}
% %    \defn{trace} :: \forall \kappa.\ ((A[\kappa], \delay\kappa U) \to (B[\kappa], U)) \to A[\kappa] \to B[\kappa]
%   \end{displaymath}
  \begin{displaymath}
    \begin{array}{l}
      \defn{feedback} : \forall \kappa.~(\delay\kappa U \to (B[\kappa], U)) \to B[\kappa] \\
      \\
      \only<5->{\defn{replaceMin} :: \tyname{Tree} \to \tyname{Tree}} \\
      \only<5->{\defn{replaceMin}\ \ident{t} = \kw{force}\ (\Lambda \kappa.~\ident{feedback}[\kappa]\ (\ident{replaceMinBody}[\kappa])\ \ident{t})}
%\kw{force}\ (\Lambda \kappa. \ident{trace}[\kappa]\ (\ident{replaceMinBody}[\kappa])\ \ident{t})}
    \end{array}
  \end{displaymath}
\end{slide}

\begin{slide}{}
  \begin{center}
    {\usebeamercolor[fg]{frametitle}\usebeamerfont{frametitle}A Slogan} \\

    \pause
    \medskip

    \emph{Well \st{typed} clocked programs ramble on forever \\ and never get to the $\bot$ of things.}

    \pause
    \medskip

    \textcolor{black!60}{\emph{(Proved using a Kripke-indexed PER
        semantics over a domain-theoretic model of the untyped lazy
        $\lambda$-calculus)}}
  \end{center}
\end{slide}

\begin{frame}
  \textcolor{titlered}{\emph{See the paper for}}:
  \quad\quad ${\color{titlered}\left\{
    \begin{array}{l}
      \textrm{\textcolor{black}{Many more examples - \emph{not just streams!}}} \\
      \textrm{\textcolor{black}{Formal calculus}} \\
      \textrm{\textcolor{black}{Type soundness proof (no program denotes $\bot$)}} \\
      \textrm{\textcolor{black}{Representation of final coalgebras and initial-final coincidence}} \\
    \end{array}\right.}$

\pause
\vspace{1.2cm}

\textcolor{titlered}{\emph{The clocks are ticking}}...
\quad\quad ${\color{titlered}\left\{
    \begin{array}{l}
      \textrm{\textcolor{black}{Can well-founded recursion be handled with clocks?}} \\
      \textrm{\textcolor{black}{Extension to dependent types}} \\
      \textrm{\textcolor{black}{Call-by-value models/implementation}} \\
      \textrm{\textcolor{black}{Application to Okasaki's amortised lazy data structures?}}
    \end{array}\right.}$

\end{frame}

\end{document}