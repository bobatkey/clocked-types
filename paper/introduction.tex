\section{Introduction}
\label{sec:introduction}

Coprogramming refers to the practice of explicitly manipulating
codata, the non-well-founded dual of well-founded data like finite
lists and trees. Codata is useful for representing the output of an
infinite process as a never ending stream, or for representing
potentially infinite search trees of possibilities. The natural way to
express codata is as a recursively defined coprogram. However, for
recursively defined coprograms to be safely used in a total setting,
the system must ensure that all recursive definitions are
productive. The state of the art for productivity checking is
currently either awkward syntactic guardedness checkers, or more
flexible sized type systems that explicitly label everything with size
information.

In this paper, we investigate the use of \emph{guarded recursion} (due
to Hiroshi Nakano \cite{nakano00modality}) as a lightweight typing
discipline for enabling productive coprogramming. As we show in this
introduction, guarded recursion by itself is not suitable for
coprogramming, due to the strict segmentation of time inherent in
putting guardedness information in types. We therefore introduce the
concept of \emph{clock variables} to take us from the
finite-time-sliced world of guarded recursion to the infinite world of
codata.

The contributions we make in this paper are the following:
\begin{enumerate}
\item We define a core type system for comfortable productive
  coprogramming, combining three key features: initial algebras,
  Nakano's guarded recursion and our main conceptual contribution:
  quantification over clock variables. We show that by combining the
  three key features of our system, we obtain a system for programming
  with \emph{final coalgebras}, the category theoretic description of
  codata. The combination of guarded recursion and clock
  quantification allows us to dispense with the clunky syntactic
  guardedness checks, and move to a flexible, compositional and local
  type-based system for ensuring guardedness.
\item We define a domain-theoretic denotational model that effectively
  interprets our system in the untyped lazy $\lambda$-calculus. We use
  a multiply-step-indexed model of types to prove that all well-typed
  programs are either terminating or productive and to show that
  correctness of our reconstruction of final coalgebras.
\item A side benefit of the combination of guarded recursion and clock
  quantification is that, due to the careful tracking of what data is
  available when through guarded types, we are able to accurately
  type, and show safe, strange circular functional programs like
  Bird's replaceMin example.
\end{enumerate}

Despite the large amount of recent work on guarded recursion (we
provide references throughout this introduction), this is the first
paper that formally links guarded recursion to productive
coprogramming. As recently noted by Birkedal and
M{\o}gelberg~\cite{birkedal13intensional}, ``guarded recursive types
can be used to[sic] as a crutch for higher-order programming with
coinductive types''. In this paper, we show that this is indeed
possible.

\subsection{Guardedness checkers and guarded recursion}

Several systems with support for corecursion, such as Coq (as
described by Gim{\'e}nez \cite{gimenez94codifying}) and Agda (as
described by Danielsson and Altenkirch \cite{danielsson09mixing}),
make use of syntactic guardedness checks to ensure productivity of
programs defined using corecursion. Without these checks, the
soundness of these systems cannot be guaranteed. An unfortunate aspect
of these guardedness checks is their non-compositionality. We
demonsrate the problem with an example, and see how Nakano's guarded
recursion can be used to provide a compositional type-based
guardedness check.

We will use Haskell notation for each of our informal examples, since
it serves to illustrate the issues concisely. Consider the following
Haskell declaration of a type of infinite streams of integers:
\begin{displaymath}
  \kw{data}\ \tyname{Stream} = \cons{StreamCons}\ \tyname{Integer}\ \tyname{Stream}
\end{displaymath}
% FIXME: something about (co)data coincidence

An example of a $\tyname{Stream}$ is the infinite stream of $1$s:
\begin{displaymath}
  \begin{array}{l}
    \defn{ones} :: \tyname{Stream} \\
    \defn{ones} = \cons{StreamCons}\ 1\ \ident{ones}
  \end{array}
\end{displaymath}
It is easy to see that this recursive definition is \emph{guarded};
$\ident{ones}$ is only invoked recursively within an application of
the stream constructor $\cons{StreamCons}$. This definition of
$\ident{ones}$ defines an infinite data structure, but each piece of
the data structure is delivered to us in finite time. Hence,
$\ident{ones}$ is productive.

An example of a non-guarded definition is the filter function,
extended from finite lists to infinite streams:
\begin{displaymath}
  \begin{array}{l}
    \defn{filter} :: (\tyname{Integer} \to \tyname{Bool}) \to \tyname{Stream} \to \tyname{Stream} \\
    \defn{filter}\ \ident{f}\ (\cons{StreamCons}\ \ident{z}\ \ident{s}) = \\
    \hspace{0.5cm}\kw{if }\ident{f}\ \ident{z}\kw{ then }\ident{filter}\ \ident{f}\ \ident{s}\kw{ else }\cons{StreamCons}\ \ident{z}\ (\ident{filter}\ \ident{f}\ \ident{s})
  \end{array}
\end{displaymath}
This definition is not guarded: in the $\kw{then}$-case of the
conditional, the recursive call to $\ident{filter}$ is not within an
application of the stream constructor $\cons{StreamCons}$. A syntactic
guardedness checker is right to reject this definition: consider the
case when all of the elements of the stream are filtered out;
$\ident{filter}$ will never return anything, and so will be
non-productive.

Syntactic guardedness checkers are not always so helpful. The
following higher-order function defines a general merge function on
pairs of streams:
\begin{displaymath}
  \begin{array}{l}
    \defn{mergef} :: (\tyname{Integer} \to \tyname{Integer} \to \tyname{Stream} \to \tyname{Stream}) \to \\
    \textcolor{white}{\textrm{mergef} :: }\tyname{Stream} \to \tyname{Stream} \to \tyname{Stream} \\
    \defn{mergef}\ \ident{f}\ (\cons{StreamCons}\ \ident{x}\ \ident{xs})\ (\cons{StreamCons}\ \ident{y}\ \ident{ys}) = \\
    \quad\quad \ident{f}\ \ident{x}\ \ident{y}\ (\ident{mergef}\ \ident{f}\ \ident{xs}\ \ident{ys})
  \end{array}
\end{displaymath}
Any syntactic checker looking for constructors to guard recursive
calls will reject this function: there are no constructors anywhere in
the definition! This rejection is with good reason: there are
functions that we could pass to $\ident{mergef}$ that would render it
unproductive. For example, this function will cause $\ident{mergef}$
to hang on all pairs of streams:
\begin{displaymath}
  \begin{array}{l}
    \defn{badf} :: \tyname{Integer} \to \tyname{Integer} \to \tyname{Stream} \to \tyname{Stream} \\
    \defn{badf}\ \ident{x}\ \ident{y}\ \ident{s} = \ident{s}
  \end{array}
\end{displaymath}
On the other hand, there are plenty of good functions $\ident{f}$ that
we could use with $\ident{mergef}$ to obtain productive functions, but
a syntactic guardedness checker does not allow us to express this
fact.

A possible way out of this problem is to retain the syntactic
guardedness check, and work around it by changing the type of
$\ident{f}$. For instance, we could change required type of
$\ident{f}$ to
\begin{displaymath}
  \ident{f} :: \tyname{Integer} \to \tyname{Integer} \to \tyname{Integer}
\end{displaymath}
to allow for merging functions that operate element-wise. Or we could
change the type to
\begin{displaymath}
  \ident{f} :: \tyname{Integer} \to \tyname{Integer} \to (\tyname{Integer}, [\tyname{Integer}])
\end{displaymath}
which would allow for the functional argument to replace each pair of
elements from the input streams with a non-empty list of values. The
general trick is to make $\ident{f}$ return ``instructions'' on how to
transform the stream back to $\ident{mergef}$, which then executes
them in a way that the syntactic guardedness checker can see is
guarded. This technique has been elaborated in detail by Danielsson
\cite{danielsson10beating}. However, it is difficult to see whether we
could capture all the possibilities for good $\ident{f}$s in a single
type. We could give yet another type which would accommodate the
behaviour of the following possible value of $\ident{f}$:
\begin{displaymath}
  \begin{array}{l}
    \defn{f}\ \ident{x}\ \ident{y}\ \ident{s} = \cons{StreamCons}\ \ident{x}\ (\ident{map}\ (+\ident{y})\ \ident{s})
  \end{array}
\end{displaymath}
Incorporating the forms of all the possible productive definitions
into a single type seems impractical. Moreover, we complicate the
definition of $\ident{mergef}$, which is now forced to become
essentially an implementation of a virtual machine for running little
stream transformer programs returned by its functional argument,
rather than the simple one line definition we gave above.

A more promising type-based solution is provided by Nakano
\cite{nakano00modality}. Nakano introduces a guardedness type
constructor that represents values that may only be used in a guarded
way. Thus Nakano transports the guardedness checks from the syntactic
level into the type system. We write this constructor, applied to a
type $A$, as $\rhd A$ (Nakano uses the notation $\bullet A$, but the
$\rhd A$ notation has become more common, and conveys an intuitive
idea of movement in time). A useful way to think of $\rhd A$ is as a
value of $A$ that is only available ``tomorrow'', and the temporal gap
between today and tomorrow can only be bridged by the application of a
constructor such as $\cons{StreamCons}$. The reading of $\rhd A$ as a
value of $A$ tomorrow is explicitly supported in the step-indexed
semantics we present in \autoref{sec:semantics} by means of a clock
counting down to zero.

We now alter the type of $\ident{f}$ to be:
\begin{displaymath}
  \ident{f} :: \tyname{Integer} \to \tyname{Integer} \to \rhd\tyname{Stream} \to \tyname{Stream}
\end{displaymath}
This type captures the intution that the stream argument to
$\ident{f}$ must be only be used in a guarded way. To introduce
guarded types into the system, we must make some changes to the types
of our primitives. We alter the type of the constructor
$\cons{StreamCons}$ as follows, and also introduce a stream
deconstructor that we previously implicitly used via pattern matching:
\begin{displaymath}
  \begin{array}{l}
    \cons{StreamCons} :: \tyname{Integer} \to \rhd\tyname{Stream} \to \tyname{Stream} \\
    \elim{deStreamCons} :: \tyname{Stream} \to (\tyname{Integer}, \rhd\tyname{Stream})
  \end{array}
\end{displaymath}
The type of $\cons{StreamCons}$ shows that it takes a guarded stream
``tomorrow'', and produces a non-guarded stream ``today''. This is in
line with the temporal intuition we have of the type
$\rhd\tyname{Stream}$. The inverse operation $\elim{deStreamCons}$
takes a full stream and returns the integer and the \emph{guarded}
remainder of the stream.

To give the guardedness type constructor force, Nakano proposes the
following alternative type of the fixpoint operator for defining
recursive values:
\begin{displaymath}
  \kw{fix} :: (\rhd A \to A) \to A
\end{displaymath}
The standard typing of the $\kw{fix}$ operator is $(A \to A) \to
A$. Nakano's alternative typing ensures that recursive definitions
must be guarded by only allowing access to recursively generated
values ``tomorrow''.

To actually program with the guardedness constructor, we equip it with
the structure of an applicative functor \cite{mcbride08applicative}:
\begin{displaymath}
  \begin{array}{l}
    \kw{pure} :: A \to \rhd A \\
    \kw{$(\circledast)$} :: \rhd (A \to B) \to \rhd A \to \rhd B
  \end{array}
\end{displaymath}
Note that $\rhd$ is not a monad: a join operation of type $\rhd(\rhd
A) \to \rhd A$ would allow us to collapse multiple guardedness
obligations.

This method for integrating Nakano's guardedness type constructor into
a typed $\lambda$-calculus is not the only one.  Nakano uses an system
of subtyping with respect to the guardedness type constructor that has
a similar effect to assuming that $\rhd$ is an applicative functor. We
propose to treat $\rhd$ applicative functor here, due to the greater
ease with which it can be incorporated into a standard functional
programming language like Haskell. Krishnaswami and Benton
\cite{krishnaswami11semantic,krishnaswami11ultrametric} have presented
an alternative method that annotates members of the typing context
with the level of guardedness that applies to them. Severi and de
Vries \cite{severi12pure} have generalised Krishnaswami and Benton's
approach for any typed $\lambda$-calculus derived from a Pure Type
System (PTS), including dependently typed systems. These calcului have
nice proof theoretic properties, such as being able to state
productivity in terms of infinitary strong normalisation.

With the guardedness type constructor, and its applicative functor
structure, we can rewrite the $\ident{mergef}$ function to allow the
alternative typing which only allows functional arguments that will
lead to productive definitions.
\begin{displaymath}
  \begin{array}{l}
    \defn{mergef} :: (\tyname{Integer} \to \tyname{Integer} \to \rhd\tyname{Stream} \to \tyname{Stream}) \to \\
    \textcolor{white}{\textrm{mergef} :: }\ \tyname{Stream} \to \tyname{Stream} \to \tyname{Stream} \\
    \defn{mergef}\ \ident{f} = \kw{fix}\ (\lambda \ident{g}\ \ident{xs}\ \ident{ys}. \\
    \hspace{2.5cm} \kw{let}\ (\ident{x}, \ident{xs'}) = \elim{deStreamCons}\ \ident{xs} \\
    \hspace{2.5cm} \textcolor{white}{\kw{let}\ }(\ident{y}, \ident{ys'}) = \elim{deStreamCons}\ \ident{ys} \\
    \hspace{2.5cm} \kw{in}\ \ident{f}\ \ident{x}\ \ident{y}\ (\ident{g} \circledast \ident{xs'} \circledast \ident{ys'}))
  \end{array}
\end{displaymath}
Aside from the explicit uses of $\kw{fix}$ and $\elim{deStreamCons}$,
which were hidden by syntactic sugar in our previous definition, the
only substantial changes to the definition are the uses of applicative
functor apply ($\circledast$). These uses indicate operations that are
occuring under a guardedness type constructor.

\subsection{From the infinite to the finite}
\label{sec:infinite-to-finite}

The type system for guarded recursion that we described above allowed
us to remove the syntactic guardedness check and replace it with a
compositional type-based one. However, aside from proposing
applicative functor structure as a convienent way to incorporate the
guardedness type constructor into a functional language, we have not
gone much beyond Nakano's original system. We now describe a problem
with guarded recursion when attempting to combine infinite and finite
data. We propose a solution to this problem in the
\hyperref[sec:clock-vars]{next subsection}.

Consider the $\ident{take}$ function that reads a finite prefix of an
infinite stream into a list. This function will have the following type:
\begin{displaymath}
  \defn{take} :: \tyname{Natural} \to \tyname{Stream} \to [\tyname{Integer}]
\end{displaymath}
where we wish to regard the type $[\tyname{Integer}]$ as the type of
\emph{finite} lists of integers. The explicit segregation of
well-founded and non-well-founded types is important for our intended
applications of total functional programming and theorem proving.

We also assume that the type $\tyname{Natural}$ of natural numbers is
well-founded, so we may attempt to write $\ident{take}$ by structural
recursion on its first argument. However, we run into a difficulty,
which we have \colorbox{greybg}{highlighted}.
\begin{displaymath}
  \begin{array}{l}
    \defn{take} :: \tyname{Natural} \to \tyname{Stream} \to [\tyname{Integer}] \\
    \defn{take}\ 0\ \ident{s} = [] \\
    \defn{take}\ (\ident{n}+1)\ \ident{s} = \ident{x} \mathop{\cons{:}} \ident{take}\ \ident{n}\ \colorbox{greybg}{$\ident{s'}$} \\
    \quad \kw{where}\ (\ident{x}, \ident{s'}) = \elim{deStreamCons}\ \ident{s}
  \end{array}
\end{displaymath}
The problem is that the variable $\ident{s'}$, which we have obtained
from $\elim{deStreamCons}$, has type $\rhd\tyname{Stream}$. However,
to invoke the $\ident{take}$ function structurally recursively, we
need something of type $\tyname{Stream}$, without the guardedness
restriction.

We analyse this problem in terms of our intuitive reading of
$\rhd\tyname{Stream}$ as a stream that is available
``tomorrow''. Nakano's typing discipline slices the construction of
infinite data like $\tyname{Stream}$ into discrete steps. In contrast,
a well-founded data type like $[\tyname{Integer}]$ lives entirely in
the moment. While a stream is being constructed, this slicing is
required so that we do not get ahead of ourselves and attempt to build
things today from things that will only be available tomorrow. But
once a stream has been fully constructed, we ought to be able to blur
the distinctions between the days of its construction. We accomplish
this in our type system by means of \emph{clock variables}, and
quantification over them.

\subsection{Clock variables}
\label{sec:clock-vars}

We extend the type system with \emph{clock variables} $\kappa$. A
clock variable $\kappa$ represents an individual time sequence that
can be used for safe construction of infinite data like streams. In
our model, clock variables are interpreted as counters running down to
zero in the style of step-indexed models. By quantifying over all
counters we are able to accommodate all possible finite observations
on an infinite data structure.

We annotate the guardedness type constructor with a clock variable to
indicate which time stream we are considering: $\delay\kappa
A$. Infinite data types such as streams are now annotated by their
clock variable, indicating the time stream that they are being
constructed on. We have the following new types for the constructor
and deconstructor:
\begin{displaymath}
  \begin{array}{l}
    \cons{StreamCons$^\kappa$} :: \tyname{Integer} \to \delay\kappa \tyname{Stream}^\kappa \to \tyname{Stream}^\kappa \\
    \elim{deStreamCons$^\kappa$} :: \tyname{Stream}^\kappa \to (\tyname{Integer}, \delay\kappa\tyname{Stream}^\kappa)    
  \end{array}
\end{displaymath}
We now regard the type $\tyname{Stream}^\kappa$ as the type of
infinite streams in the process of construction. A finished infinite
stream is represented by \emph{quantifying} over the relevant clock
variable: $\forall \kappa. \tyname{Stream}^\kappa$. If we think of
each possible downward counting counter that $\kappa$ could represent,
then this universally quantified type allows for any counter large
enough to allow us any finite extraction of data from the stream. We
use the notation $\Lambda\kappa. e$ and $e[\kappa]$ to indicate clock
abstraction and application respectively, in the style of explicitly
typed System F type abstraction and application.

The fixpoint operator is parameterised by the clock variable:
\begin{displaymath}
  \kw{fix} :: \forall \kappa. (\delay\kappa A \to A) \to A
\end{displaymath}
as is the applicative functor structure carried by $\delay\kappa$:
\begin{displaymath}
  \begin{array}{l}
    \kw{pure} :: \forall \kappa. A \to \delay\kappa A \\
    \kw{$(\circledast)$} :: \forall \kappa. \delay\kappa (A \to B) \to \delay\kappa A \to \delay\kappa B
  \end{array}
\end{displaymath}
We also add an additional piece of structure to eliminate guarded
types that have been universally quantified:
\begin{displaymath}
  \kw{force} :: (\forall \kappa.\ \delay\kappa A) \to (\forall \kappa.\ A)
\end{displaymath}
Intuitively, if we have a value that, for any time stream, is one time
step away, we simply instantiate it with a time stream that has at
least one step left to run and extract the value. Note that the
universal quantification is required on the right hand side since
$\kappa$ may appear free in the type $A$. The $\kw{force}$ operator
has a similar flavour to the $\kw{runST} :: (\forall s. \tyname{ST}\
s\ a) \to a$ for the $\tyname{ST}$ monad
\cite{DBLP:journals/jfp/MoggiS01,launchbury94lazy}. In both cases,
rank-2 quantification is used to prevent values from ``leaking'' out
from the context in which it is safe to use them.

Using $\kw{force}$ we may write a deconstructor for completed streams
of type $\forall \kappa. \tyname{Stream}^\kappa$.
\begin{displaymath}
  \begin{array}{l}
    \elim{deStreamCons} :: (\forall \kappa. \tyname{Stream}^\kappa) \to (\tyname{Integer}, \forall \kappa. \tyname{Stream}^\kappa) \\
    \elim{deStreamCons}\ \ident{x} = \\
    \quad\quad
    \begin{array}[t]{l}
      (\Lambda \kappa. \kw{fst}\ (\elim{deStreamCons}^\kappa\ (\ident{x}[\kappa])),\\
      \quad\kw{force}\ (\Lambda \kappa. \kw{snd}\ (\elim{deStreamCons}^\kappa\ (\ident{x}[\kappa]))))
    \end{array}
  \end{array}
\end{displaymath}
In this definition we have made use of a feature of our type system
that states that the type equality $\forall \kappa. A \equiv A$ holds
whenever $\kappa$ is not free in $A$. Our system also includes other
type equalities that demonstrate how clock quantification interacts
with other type formers. These are presented in
\autoref{sec:type-equality}.

In the presence of clock variables, our definition of $\ident{take}$
is well-typed and we get the function we desire: taking a finite
observation of a piece of infinite data:
\begin{displaymath}
  \begin{array}{l}
    \defn{take} :: \tyname{Natural} \to (\forall \kappa. \tyname{Stream}^\kappa) \to [\tyname{Integer}] \\
    \defn{take}\ 0\ \ident{s} = [] \\
    \defn{take}\ (\ident{n}+1)\ \ident{s} = \ident{x} \mathop{\cons{:}} \ident{take}\ \ident{n}\ \ident{s'} \\
    \quad \kw{where}\ (\ident{x}, \ident{s'}) = \elim{deStreamCons}\ \ident{s}
  \end{array}
\end{displaymath}

By using clock quantification to ``close over'' the guardedness, our
approach thus \emph{localises} the discipline which makes the
productivity of definitions plain, without affecting the types at
which the defined objects are used. Abel's \emph{sized types} impose a
similar discipline, but globally~\cite{abel04termination}.

\subsection{Final coalgebras from initial algebras, guards and clocks}
\label{sec:main-results-intro}

In the previous subsection, we informally stated that the type
$\tyname{Stream} = \forall \kappa. \tyname{Stream}^\kappa$ represents
exactly the type of infinite streams of integers. We make this
informal claim precise by using the category theoretic notion of final
coalgebra. A key feature of our approach is that we decompose the
notion of final coalgebra into three parts: initial algebras,
Nakano-style guarded types, and clock quantification.

\paragraph{Initial algebras}
In \autoref{sec:type-system} we present a type system that includes a
notion of strictly positive type operator. We include a least fixpoint
operator $\mu X. F[X]$. For normal strictly positive type operators
$F$ (i.e. ones that do not involve the $\delay\kappa$ operator), $\mu
X. F[X]$ is exactly the normal least fixpoint of $F$. Thus we can
define the normal well-founded types of natural numbers, lists, trees
and so on. Our calculus contains constructors $\cons{Cons$_F$} ::
F[\mu X. F] \to \mu X. F$ for building values of these types, and
recursion combinators for eliminating values:
\begin{displaymath}
  \elim{primRec$_{F,A}$} :: (F[(\mu X. F) \times A] \to A) \to \mu X. F[X] \to A 
\end{displaymath}
We opt to use primitive recursion instead of a fold operator
(i.e. $\ident{fold} :: (F[A] \to A) \to \mu X. F[X] \to A$) because
it allows for constant time access to the top-level of an inductive
data structure.

We show in the proof of \thmref{thm:initial-f-algebra} that $\mu X. F$
is the carrier of the initial $F$-algebra in our model. Therefore, we
may use standard reasoning techniques about initial algebras to reason
about programs written using the $\elim{primRec}$ combinator.

\paragraph{Guarded final coalgebras}
When we consider strictly positive type operators of the form
$F[\delay\kappa X]$, where $\kappa$ does not appear in $F$ then, in
addition to $\mu X. F[\delay\kappa X]$ being the least fixpoint of the
operator $F[\delay\kappa -]$, it is also the \emph{greatest}
fixpoint. This initial-final coincidence is familiar from domain
theoretic models of recursive types (see, e.g., Smyth and Plotkin
\cite{smyth82category}), and has been previously been observed as a
feature of guarded recursive types by Birkedal
\etal~\cite{birkedal12first}.

The $\ident{unfold}$ combinator that witnesses $\mu X. F[\delay\kappa
X]$ as final can be implemented in terms of the $\kw{fix}$ operator,
and $\cons{Cons}$:
\begin{displaymath}
  \begin{array}[t]{l}
    \defn{unfold$_F$} :: \forall \kappa. (A \to F[\delay\kappa A]) \to A \to \mu X. F[\delay\kappa X] \\
    \defn{unfold$_F$} = \Lambda \kappa.\lambda \ident{f}. \kw{fix}[\kappa]\ (\lambda \ident{g}\ \ident{a}. \cons{Cons}\ (\ident{fmap}_F\ (\lambda \ident{x}. \ident{g} \circledast \ident{x})\ (\ident{f}\ \ident{a})))
  \end{array}
\end{displaymath}
where $\ident{fmap}_F :: (A \to B) \to (F[A] \to F[B])$ is the
functorial mapping combinator associated with the strictly positive
type operator $F$. Note that without the additional $\delay\kappa$ in
$F[\delay\kappa -]$, there would be no way to define $\ident{unfold}$,
due to the typing of $\kw{fix}$.

To make observations on elements, we define a $\ident{deCons$_F$}$
combinator for every $F$, using the primitive recursion
\begin{displaymath}
  \begin{array}{l}
    \defn{deCons$_F$} :: \forall \kappa. (\mu X. F[\delay\kappa X]) \to F[\delay\kappa \mu X. F[\delay\kappa X]] \\
    \defn{deCons$_F$} = \Lambda \kappa. \elim{primRec}\ (\lambda \ident{x}.\ \ident{fmap}_{F[\delay\kappa-]}\ (\lambda \ident{y}.\kw{fst}\ \ident{y})\ \ident{x})
  \end{array}
\end{displaymath}
The proof of \thmref{thm:final-f-de-coalgebra} shows that these
definitions of $\ident{unfold$_F$}$ and $\ident{deCons$_F$}$ exhibit
$\mu X. F[\delay\kappa X]$ as the final $F[\delay\kappa-]$-coalgebra,
using $\ident{deCons$_F$}$ as the structure map. This means that
$\ident{unfold$_F$}[\kappa] f$ is the unique $F[\delay\kappa-]$-coalgebra
homomorphism from $\mu X. F[\delay\kappa-]$ to $A$.

\paragraph{Final coalgebras}
\thmref{thm:final-f-de-coalgebra} only gives us final coalgebras in
the case when the recursion in the type is guarded by the type
constructor $\delay\kappa-$. So we do not yet have a dual to the least
fixpoint type constructor $\mu X. F$. However, as we hinted above in
the case of streams, we can use clock quantification to construct a
final $F$-coalgebra, where $F$ need not contain an occurence of
$\delay\kappa-$.

For the type $\forall \kappa. \mu X. F[\delay\kappa-]$ we define an
$\ident{unfold$^\nu_F$}$ operator, in a similar style to the
$\ident{unfold$_F$}$ operator above. However, this operator takes an
argument of type $(A \to F[A])$, with no mention of guardedness.
\begin{displaymath}
  \begin{array}{l}
    \defn{unfold$^\nu_F$} :: (A \to F[A]) \to A \to \forall \kappa. \mu X. F[\delay\kappa X] \\
    \defn{unfold$^\nu_F$} = \lambda \ident{f}\ \ident{a}. \Lambda \kappa. \\
    \quad\quad\kw{fix}[\kappa]\ (\lambda \ident{g}\ \ident{a}. \cons{Cons}\ (\ident{fmap}_F (\lambda \ident{x}. \ident{g} \circledast \kw{pure}\ \ident{x})\ (\ident{f}\ \ident{a})))\ \ident{a}
  \end{array}
\end{displaymath}
We can also define the corresponding deconstructor, building on the
definition of $\ident{deCons$_F$}$ above, but also using the $\kw{force}$
construct in conjunction with clock quantification. This is a
generalisation of the definition of $\elim{deStreamCons}$ above.
\begin{displaymath}
  \begin{array}{l}
    \defn{deCons$^\nu_F$} :: (\forall \kappa. \mu X. F[\delay\kappa X]) \to F[\forall \kappa. \mu X. F[\delay\kappa X]] \\
    \defn{deCons$^\nu_F$} = \ident{fmap$_F$}\ (\lambda x. \kw{force}\ x)\ (\Lambda \kappa. \ident{deCons$_F$}[\kappa]\ (x[\kappa]))
  \end{array}
\end{displaymath}
In the application of $\ident{fmap$_F$}$, we make use of the type
equalities in our calculus, which allow us to treat the types $\forall
\kappa. F[X]$ and $F[\forall \kappa. X]$ as equivalent, when $\kappa$
does not appear in $F$. We present the type equalities in
\autoref{sec:type-equality}, and prove them sound in
\autoref{sec:type-interp}.

Our last main technical result, \thmref{thm:final-f-coalgebra}, states
that, in the semantics we define \autoref{sec:semantics}, $\forall
\kappa. \mu X. F[\delay\kappa-]$ actually is the final
$F$-coalgebra. The use of clock quantification has allowed us to
remove mention of the guardedness type constructor, and given us a way
to safely program and reason about $F$-coalgebras.

\subsection{Circular traversals of trees}
\label{sec:repmin}

We now present a perhaps unexpected application of the use of clock
quantification that does not relate to ensuring productivity of
recursive programs generating infinite data. An interesting use of
laziness in functional programming languages is to peform single-pass
transformations of data structures that would appear to require at
least two passes. A classic example, due to Bird
\cite{bird84circular}, is the $\ident{replaceMin}$ function that
replaces each value in a binary tree with the minimum of all the
values in the tree, in a single pass. In normal Haskell, this function
is written as follows:
\begin{displaymath}
  \begin{array}{l}
    \defn{replaceMin} :: \tyname{Tree} \to \tyname{Tree} \\
    \defn{replaceMin}\ \ident{t} = \kw{let}\ (\ident{t'}, \ident{m}) = \ident{replaceMinBody}\ \ident{t}\ \ident{m}\ \kw{in}\ \ident{t'} \\
    \hspace{0.6cm} \kw{where} \\
    \hspace{1cm} \defn{replaceMinBody}\ (\cons{Leaf}\ \ident{x})\ \ident{m} = (\cons{Leaf}\ \ident{m}, \ident{x}) \\
    \hspace{1cm} \defn{replaceMinBody}\ (\cons{Br}\ \ident{l}\ \ident{r})\ \ident{m} = \\
    \hspace{2cm} \kw{let}\ (\ident{l'}, \ident{m$_l$}) = \ident{replaceMinBody}\ \ident{l}\ \ident{m} \\
    \hspace{2cm} \textcolor{white}{\kw{let}\ }(\ident{r'}, \ident{m$_r$}) = \ident{replaceMinBody}\ \ident{r}\ \ident{m} \\
    \hspace{2cm} \kw{in}\ (\cons{Br}\ \ident{l'}\ \ident{r'},\ \ident{min}\ \ident{m$_l$}\ \ident{m$_r$})
  \end{array}
\end{displaymath}
The interesting part of this function is the first $\kw{let}$
expression. This takes the minimum value $\ident{m}$ computed by the
traversal of the tree and passes it into the \emph{same} traversal to
build the new tree. This function is a marvel of declarative
programming: we have declared that we wish the tree $t'$ to be
labelled with the minimum value in the tree $t$ just by stating that
they be the same, without explaining at all why this definition makes
any sense. Intuitively, the reason that this works is that the overall
minimum value is never used in the computation of the minimum, only
the construction of the new tree. The computation of the minimum and
the construction of the new tree conceptually exist at different
moments in time, and it is only safe to treat them the same after both
have finished.

Using explicit clock variables we can give a version of the
$\ident{replaceMin}$ definition that demonstrates that we have
actually defined a total function from trees to trees. The use of
clock variables allows us to be explicit about the time when various
operations are taking place.

Out first step is to replace the circular use of $\kw{let}$ with a
\emph{feedback} combinator defined in terms of the $\kw{fix}$
operator:
\begin{displaymath}
  \begin{array}{l}
    \defn{feedback} : \forall \kappa.~(\delay\kappa U \to (B[\kappa], U)) \to B[\kappa] \\
    \defn{feedback} = \Lambda \kappa. \lambda f. \kw{fst}~(\kw{fix}~(\lambda x.~f~(\kw{pure}~(\lambda x. \kw{snd}~x) \circledast x)))
  \end{array}
\end{displaymath}
In the application of the $\kw{fix}$ operator, $x : \delay\kappa (A B
\times U)$.  The notation $B[\kappa]$ indicates that $\kappa$ may
appear free in the type $B$.

% We have named this definition $\ident{trace}$ due to similarity of its
% type to the trace operator of traced symmetric monoidal categories
% \cite{something}. Trace operators provide an axiomatisaion of the
% notion of feedback operator that has many applications in denotational
% semantics. The key difference with our definition is that the first
% occurence of the ``feedback type'' $U$ is guarded by the guardedness
% type constructor. This will prevent us from writing definitions that
% get ahead of themselves and access data that is not yet ready.

\newcommand{\hlchangem}[1]{\colorbox{greybg}{$#1$}}
\newcommand{\hlchange}[1]{\colorbox{greybg}{#1}}

We now rewrite the $\ident{replaceMinBody}$ function so that we can
apply $\ident{feedback}$ to it. We have \hlchange{highlighted} the
changes from the previous definition.
\begin{displaymath}
  \begin{array}{@{}l}
    \defn{replaceMinBody} :: \tyname{Tree} \to \hlchangem{\forall \kappa.}\hlchangem{\delay\kappa}\tyname{Integer} \to (\hlchangem{\delay\kappa}\tyname{Tree},\tyname{Integer}) \\
    \hspace{0cm} \defn{replaceMinBody}\ (\cons{Leaf}\ \ident{x})\ \ident{m} = (\hlchangem{\kw{pure}}\cons{Leaf}\ \hlchangem{\circledast}\ident{m}, \ident{x}) \\
    \hspace{0cm} \defn{replaceMinBody}\ (\cons{Br}\ \ident{l}\ \ident{r})\ \ident{m} = \\
    \hspace{2cm} \kw{let}\ (\ident{l'}, \ident{m$_l$}) = \ident{replaceMinBody}\ \ident{l}\ \ident{m} \\
    \hspace{2cm} \textcolor{white}{\kw{let}\ }(\ident{r'}, \ident{m$_r$}) = \ident{replaceMinBody}\ \ident{r}\ \ident{m} \\
    \hspace{2cm} \kw{in}\ (\hlchangem{\kw{pure}}\cons{Br}\ \hlchangem{\circledast}\ident{l'}\ \hlchangem{\circledast}\ident{r'},\ \ident{min}\ \ident{m$_l$}\ \ident{m$_r$})
  \end{array}
\end{displaymath}
The changes required are minor: we must change the type, and use the
applicative functor structure of $\delay\kappa$ to indicate when
computations are taking place ``tomorrow''.

Applying $\ident{feedback}$ to $\ident{replaceMinBody}$, and using
$\kw{force}$ to remove the now redundant occurence of $\delay\kappa$,
we can define the full $\ident{replaceMin}$ function in our system:
\begin{displaymath}
  \begin{array}{l}
    \defn{replaceMin} :: \tyname{Tree} \to \tyname{Tree} \\
    \defn{replaceMin}\ \ident{t} = \kw{force}\ (\Lambda \kappa. \ident{feedback}[\kappa]\ (\ident{replaceMinBody}[\kappa])\ \ident{t})
  \end{array}
\end{displaymath}
By the soundness property of our system that we prove in
\autoref{sec:semantics}, we are assured that we have defined a total
function from trees to trees. The standard Haskell type system does
not provide this guarantee.

\subsection{Models of guarded recursion}

To substantiate the claims we have made in this introduction, in
\autoref{sec:semantics} we construct a multiply-step-indexed model of
the type system we present in the next section. The multiple
step-indexes are required for the multiple clock variables in our
system, each representing separate time streams. Step-indexed models
were introduced by Appel and McAllester~\cite{appel01indexed} to prove
properties of recursive types. The use of step-indexing serves to
break the circularity inherent in recursive types. Dreyer
\etal~\cite{dreyer11logical} exposed the connection between Nakano's
guarded recursion and step indexed models. More recently, Birkedal
\etal~\cite{birkedal12first,birkedal13intensional} have elaborated
this connection, particularly in the direction of dependent types.

Alternative semantics for guarded recursion have involved ultrametric
spaces, for example Birkedal \etal~\cite{birkedal10metric} and
Krishnaswami and Benton \cite{krishnaswami11ultrametric}. Hutton and
Jaskelioff \cite{hutton11representing} have also used an approach
based on metric spaces for identifying productive stream generating
functions.

Finally, we mention the syntactic approach of Severi and de Vries
\cite{severi12pure}, who define the notion of infinitary strong
normalisation to prove productivity of dependent type systems with
guarded recursion.

%%% Local Variables:
%%% TeX-master: "paper"
%%% End:
