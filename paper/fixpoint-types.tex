\section{Properties of fixpoint types}
\label{sec:fixpoint-types}

Using the denotational model of the previous section, we are now in a
position to formally state and sketch the proofs of the properties of
fixpoint types that we claimed in
\autoref{sec:main-results-intro}. Our claimed results are statements
in category theoretic language about the initiality and finality of
various (co)algebras. Therefore, we first construct suitable
categories to work in.

\begin{definition}
  For each clock variable context $\Delta$, the category
  $\ClkPER(\Delta)$ has as objects semantic types over
  $\Delta$. Morphisms $f : A \to B$ are continuous functions $f : D
  \to D$ such that for all $\delta \in \sem{\Delta}$ and $(a,a') \in
  A\delta$, $(fa, fa') \in B\delta$. Two morphisms $f$,$f'$ are
  considered equivalent if for all $\delta \in \sem{\Delta}$ and
  $(a,a') \in A\delta$, $(fa, f'a') \in B\delta$.
\end{definition}

Each well-typed term $\Delta; x : A \vdash e : B$ defines a morphism
$\sem{e} : \sem{A} \to \sem{B}$ in $\ClkPER(\Delta)$. We can define
equality between terms in our syntactic type system in terms of the
equality on morphisms in this definition. Moreover, each well-formed
type operator $\Delta; X \vdash F[X] : \sortType$ defines a (strong)
endofunctor on $\ClkPER(\Delta)$, using the semantic $\mathrm{fmap}_F$
defined in \autoref{fig:semantic-fmap} to define the action on
morphisms. We have already checked (\lemref{lem:sem-fmap-well-typed})
that this is well-defined, and it is straightforward to check that the
usual functor laws hold. In what follows, we will use
$\mathit{fmap}_F$ to refer to the action of a functor $F$ on
morphisms.

\paragraph{Initial Algebras} Recall that, for any functor $F$, an
$F$-algebra is a pair of an object $A$ and a morphism $k : FA \to
A$. A homomorphism $h$ between two $F$-algebras $(A,k_A)$ and
$(B,k_B)$ is a morphism $h : A \to B$ such that $h \circ k_A = k_B
\circ \mathit{fmap}_F~h$. An \emph{initial} $F$-algebra is an
$F$-algebra $(A,k_A)$ such that for any other $F$-algebra $(B,k_B)$,
there is a unique homomorphism $h : (A,k_A) \to (B,k_B)$.

\begin{theorem}\label{thm:initial-f-algebra}
  If $F$ is an endofunctor on $\ClkPER(\Delta)$, then $\mu F$ is the
  carrier of an initial $F$-algebra.
\end{theorem}

\begin{proof} (Sketch) Since $\mu F$ is a fixpoint of $F$, the
  morphism $F(\mu F) \to \mu F$ is simply realised by the identity
  map. Given any other $F$-algebra $(B,k_B)$, define a morphism $\mu F
  \to B$ using the $\mathrm{primrec}$ operator from
  \lemref{lem:primrec-well-typed}. Checking that this gives an
  $F$-algebra homomorphism is straightforward, proving uniqueness uses
  induction on elements of $\mu F$, by the Knaster-Tarski theorem.
\end{proof}

\paragraph{Guarded Final Co-Algebras} \thmref{thm:initial-f-algebra}
applies to all functors $F$, and in particular functors of the form
$F(\delay\kappa-)$ on the category $\ClkPER(\Delta, \kappa)$. As well
as $\mu (F(\delay\kappa -))$ being the carrier of an initial algebra,
it is also the carrier of a final $F(\delay\kappa-)$-coalgebra.

Coalgebras are the formal dual of algebras: for an endofunctor $F$, an
$F$-coalgebra is a pair of an object $A$ and a morphism $k_A : A \to
FA$. A homomorphism $h : (A,k_A) \to (B,k_B)$ of coalgebras is a
morphism $h : A \to B$ such that $\mathit{fmap}_F~h \circ k_A =
k_B\circ h$. A \emph{final} $F$-coalgebra is an $F$-coalgebra
$(B,k_B)$ such that for any other $F$-coalgebra $(A,k_A)$, there is a
unique $F$-coalgebra homomorphism $\mathrm{unfold}~k_A : (A,k_A) \to
(B,k_B)$.

\begin{theorem}\label{thm:final-f-de-coalgebra}
  If $F$ is an endofunctor on $\ClkPER(\Delta)$, then $\mu
  (F(\delay\kappa-))$ is the carrier of a final
  $F(\delay\kappa-)$-coalgebra in $\ClkPER(\Delta,\kappa)$.
\end{theorem}

\begin{proof} (Sketch) As for \thmref{thm:initial-f-algebra}, since
  $\mu (F(\delay\kappa-))$ is a fixpoint of $F(\delay\kappa-)$, the
  morphism $\mu (F(\delay\kappa-)) \to F(\delay\kappa(\mu (F(\delay\kappa-))))$ is simply realised by the identity
  map. Given any other $F$-coalgebra $(A,k_A)$, define a morphism
  $\mathrm{unfold}~k_A : A \to \mu(F(\delay\kappa-))$ as
  $\mathrm{fix}~(\lambda g~a.~\mathrm{fmap}_F~g~(k_A~a))$. It is
  straightforward to prove that this is an $F$-coalgebra
  homomorphism. Uniqueness is proved for each possible clock
  environment $\delta$ by induction on $\delta(\kappa)$.
\end{proof}

The syntactic counterpart of the construction we used in this proof is
exactly the term we used in \autoref{sec:main-results-intro} for the
definition of $\ident{unfold}$. It is also easy to check that the term
$\ident{deCons}$ we defined there is semantically equivalent to the
identity. Therefore, \thmref{thm:final-f-de-coalgebra} substantiates
the claim we made in \autoref{sec:main-results-intro} that $\mu
X. F[\delay\kappa-]$ is the syntactic description of the carrier of a
final $F[\delay\kappa-]$-coalgebra.

\paragraph{Final Co-Algebras} \thmref{thm:final-f-de-coalgebra} gives
final coalgebras in the categories $\ClkPER(\Delta, \kappa)$, where we
have a spare clock variable. By using clock quantification, we can
close over this clock variable, and get the final
$F$-coalgebra, not just the final $F(\delay\kappa-)$-coalgebra.

\begin{theorem}\label{thm:final-f-coalgebra}
  For an endofunctor $F$ on $\ClkPER(\Delta)$, $\forall_\kappa. \mu
  (F(\delay\kappa-))$ is the carrier of a final $F$-coalgebra in
  $\ClkPER(\Delta)$.
\end{theorem}

\begin{proof}
  (Sketch) Almost identical to the proof for
  \thmref{thm:final-f-de-coalgebra}.
\end{proof}

This final theorem, along with the examples we presented in
\autoref{sec:examples}, substantiates our claim that the combination
of clocks and guards that we have presented in this paper is a viable
and comfortable setting in which to productively coprogram.

%%% Local Variables:
%%% TeX-master: "paper"
%%% End:
