\section{A denotational semantics for clocks and guards}
\label{sec:semantics}

We have claimed that the typing discipline from the previous section
guarantees that programs will be productive. We now substantiate this
claim by defining a domain-theoretic model of our system, and showing
that the denotation of a well-typed closed term is never $\bot$. We
accomplish this by using a multiply-step-indexed interpretation of
types, where the multiple step indexings correspond to the multiple
clock variables that may be in scope.

\subsection{Semantics of terms}
\label{sec:semantics-of-programs}

We interpret terms in (a mild extension of) the standard domain
theoretic model of the untyped lazy $\lambda$-calculus, described by
Pitts \cite{PittsAM:compavm}. A feature of this semantics is the
erasure of anything to do with the guardedness type constructor
$\delay\kappa-$ and the clock quantification. The $\kw{fix}$ operator
is interpreted directly as the domain-theoretic fixpoint, i.e.,
exactly the usual interpretation of recursion in functional
programming languages.

We assume a directed complete partial order with bottom
($\mathrm{DCPO}_\bot$), $D$, satisfying the following recursive domain
equation:
\begin{displaymath}
  D \cong (D \to D)_\bot \oplus (D \times D)_\bot \oplus D_\bot \oplus D_\bot \oplus 1_\bot
\end{displaymath}
where $\oplus$ represents the coalesced sum that identifies the $\bot$
element of all the components. We are guaranteed to be able to obtain
such a $D$ by standard results about the category $\mathrm{DCPO}_\bot$
(see, e.g., Smyth and Plotkin \cite{smyth82category}). The five
components of the right hand side of this equation will be used to
carry functions, products, the left and right injections for sum types
and the unit value respectively. We use the following symbols to
represent the corresponding continuous injective maps into $D$:
\begin{center}
  \begin{tabular}{ll}
    $\semCons{Lam} : (D \to D) \to D$ & $\semCons{Pair} : D \times D \to D$ \\
    $\semCons{Inl} : D \to D$ & $\semCons{Inr} : D \to D$ \\
    $\semCons{Unit} : 1 \to D$
  \end{tabular}  
\end{center}
We will write $\semCons{Unit}$ instead of $\semCons{Unit}\ *$.

\begin{figure}[t]
  \centering
\begin{eqnarray*}
  \sem{x}\eta & = & \eta(x) \\
  \sem{*}\eta & = & \semCons{Unit} \\
  \sem{\lambda x.\ e}\eta & = & \semCons{Lam}\ (\lambda v.\ \sem{e}(\eta[x \mapsto v])) \\
  \sem{fe}\eta & = & \left\{
    \begin{array}{ll}
      d_f\ (\sem{e}\eta) & \textrm{if}\ \sem{f}\eta = \semCons{Lam}\ d_f \\
      \bot & \textrm{otherwise}
    \end{array}
    \right. \\
  \sem{(e_1,e_2)}\eta & = & \semCons{Pair}\ (\sem{e_1}\eta, \sem{e_2}\eta) \\
  \sem{\kw{fst}\ e}\eta & = & \left\{
    \begin{array}{ll}
      d_1 & \textrm{if}\ \sem{e}\eta = \semCons{Pair}\ (d_1, d_2) \\
      \bot & \textrm{otherwise}
    \end{array}
  \right. \\
  \sem{\kw{snd}\ e}\eta & = & \left\{
    \begin{array}{ll}
      d_2 & \textrm{if}\ \sem{e}\eta = \semCons{Pair}\ (d_1, d_2) \\
      \bot & \textrm{otherwise}
    \end{array}
  \right. \\
  \sem{\kw{inl}\ e}\eta & = & \semCons{Inl}\ (\sem{e}\eta) \\
  \sem{\kw{inr}\ e}\eta & = & \semCons{Inr}\ (\sem{e}\eta) \\
  \left\llbracket
    \begin{array}{l}
      \kw{case}\ e\ \kw{of}\\
      \quad\kw{inl}\ \ident{x}.f\\
      \quad\kw{inr}\ \ident{y}.g
    \end{array}
  \right\rrbracket\eta & = &
  \left\{
    \begin{array}{ll}
      \sem{f}(\eta[x \mapsto d]) & \textrm{if }\sem{e}\eta = \semCons{Inl}\ d \\
      \sem{g}(\eta[y \mapsto d]) & \textrm{if }\sem{e}\eta = \semCons{Inr}\ d \\
      \bot & \textrm{otherwise}
    \end{array}
  \right. \\
\end{eqnarray*}  
  \caption{Semantics for the simply-typed portion of our system}
  \label{fig:semantics1}
\end{figure}

Let $V$ be the set of all possible term-level variable
names. Environments $\eta$ are modelled as maps from $V$ to elements
of $D$. Terms are interpreted as continuous maps from environments to
elements of $D$. The clauses for defining the interpretation of parts
of our system that are just the simply-typed $\lambda$-calculus are
standard, and displayed in \autoref{fig:semantics1}. It can be easily
checked that this collection of equations defines a continuous
function $\sem{-} : (V \to D) \to D$, since everything is built from
standard parts.

The applicative functor structure for the guard modality is
interpreted just as the identity function and normal application, as
we promised in \autoref{sec:typed-terms}:
\begin{eqnarray*}
  \sem{\kw{pure}\ e}\eta & = & \sem{e}\eta \\
  \sem{f \circledast e}\eta & = & \left\{
    \begin{array}{ll}
      d_f\ (\sem{e}\eta) & \textrm{if}\ \sem{f}\eta = \semCons{Lam}\ d_f \\
      \bot & \textrm{otherwise}
    \end{array}
  \right.
\end{eqnarray*}
Our interpretation simply translates the $\kw{fix}$ operator as the
usual fixpoint operator $\mathrm{fix}\ f = \bigsqcup_n f^n(\bot)$ in
$\mathrm{DCPO}_\bot$. The $\kw{force}$ operator is interpreted as a
no-operation.
\begin{eqnarray*}
  \sem{\kw{fix}\ f}\eta & = &
  \left\{
    \begin{array}{ll}
      \mathrm{fix}\ d_f & \textrm{if }\sem{f}\eta = \semCons{Lam}\ d_f \\
      \bot & \textrm{otherwise}
    \end{array}
  \right.\\
  \sem{\kw{force}\ e}\eta & = & \sem{e}\eta
\end{eqnarray*}

Finally, we interpret the constructor $\cons{Cons$_F$}$ and primitive
recursion eliminator $\elim{primRec$_F$}$ for least fixpoint
types. The interpretation of construction is straightforward: the
constructor itself disappears at runtime, so the interpretation simply
ignores it:
\begin{eqnarray*}
  \sem{\cons{Cons$_F$}\ e}\eta & = & \sem{e}\eta
\end{eqnarray*}

To define the semantic counterpart of the $\elim{primRec$_F$}$
operator, we first define a higher-order continuous function
$\mathrm{primrec} : (D \to D \to D) \to (D \to D) \to D$ in the
model. This is very similar to how one would define a generic
$\elim{primRec$_F$}$ in Haskell using general recursion, albeit here
in an untyped fashion.
\begin{displaymath}
  \begin{array}{l}
    \mathrm{primrec}\ \mathit{fmap}\ f = \\
    \quad\quad\semCons{Lam}\ (\mathrm{fix}\ (\lambda g x.\ f\ (\mathit{fmap}\ (\semCons{Lam}\ (\lambda x. \semCons{Pair}\ (x, g\ x))))\ x))
  \end{array}
\end{displaymath}
The first argument to $\mathrm{primrec}$ is intended to specify a
functorial mapping function. For each of the syntactic types $\Delta;
\Theta \vdash F : \sortType$ defined in \autoref{fig:types}, we define
a suitable $\mathrm{fmap_F} : D^{|\Theta|} \to D \to D$, where
$|\Theta|$ is the number of type variables in $\Theta$. This
definition is presented in \autoref{fig:semantic-fmap}.

\begin{figure}[t]
  \begin{eqnarray*}
    \mathrm{fmap_X}\ \vec{f}\ x & = & d_f(x)\quad (f_X = \semCons{Lam}\ d_f) \\
    \mathrm{fmap_1}\ \vec{f}\ x & = & \semCons{Unit} \\
    \mathrm{fmap_{F \times G}}\ \vec{f}\ (\semCons{Pair}\ (x,y)) & = & \semCons{Pair}\ (\mathrm{fmap_F}\ \vec{f}\ x, \mathrm{fmap_G}\ \vec{f}\ y) \\
    \mathrm{fmap_{F + G}}\ \vec{f}\ (\semCons{Inl}\ x) & = & \semCons{Inl}\ (\mathrm{fmap_F}\ \vec{f}\ x) \\
    \mathrm{fmap_{F + G}}\ \vec{f}\ (\semCons{Inr}\ x) & = & \semCons{Inr}\ (\mathrm{fmap_G}\ \vec{f}\ x) \\
    \mathrm{fmap_{A \to F}}\ \vec{f}\ (\semCons{Lam}\ g) & = & \semCons{Lam}\ (\lambda v. \mathrm{fmap_F}\ \vec{f}\ (g\ v)) \\
    \mathrm{fmap_{\delay\kappa F}}\ \vec{f}\ x & = & \mathrm{fmap_F}\ \vec{f}\ x \\
    \mathrm{fmap_{\forall\kappa.F}}\ \vec{f}\ x & = & \mathrm{fmap_F}\ \vec{f}\ x \\
    \mathrm{fmap_{\mu X. F}}\ \vec{f}\ x & = &
    \begin{array}[t]{@{}l}
      \mathrm{primrec}\ (\mathrm{fmap_F}\ \vec{f})\ \\
      \quad\quad\quad\quad(\mathrm{fmap_F}\ \vec{f}\ \mathrm{snd})\ x
    \end{array}
  \end{eqnarray*}
  where
  \begin{displaymath}
    \mathrm{snd}\ (\semCons{Pair}\ (x,y)) = y
  \end{displaymath}
  All unhandled cases are defined to be equal to $\bot$.
  \caption{Semantic $\mathrm{fmap_F}$ for $\Delta; \Theta \vdash F : \sortType$}
  \label{fig:semantic-fmap}
\end{figure}

With these definitions we can define the interpretation of the
syntactic $\elim{primRec$_F$}$ operator. Note that the syntactic type
constructor $F$ here always has exactly one free type variable, so
$\mathrm{fmap_F}$ has type $D \to D \to D$, as required by
$\mathrm{primrec}$.
\begin{displaymath}
  \begin{array}{l}
    \sem{\elim{primRec$_F$}\ f}\eta = \\
    \quad\quad\left\{
      \begin{array}{ll}
        \mathrm{primrec}\ \mathrm{fmap_F}\ d_f & \textrm{if }\sem{f}\eta = \semCons{Lam}\ d_f \\
        \bot & \textrm{otherwise}
      \end{array}
    \right.
  \end{array}
\end{displaymath}

This completes the interpretation of the terms of our programming
language in our untyped model. We now turn to the semantic meaning of
types, with the aim of showing, in \autoref{sec:semantic-soundness},
that each well-typed term's interpretation is semantically well-typed.

\subsection{Interpretation of clock variables}
\label{sec:clock-envs}

For a clock context $\Delta$, we define $\clkSem{\Delta}$ to be the
set of clock environments: mappings from the clock variables in
$\Delta$ to the natural numbers. We use $\delta, \delta'$ etc. to
range over clock environments. For $\delta$ and $\delta'$ in
$\clkSem{\Delta}$, we say that $\delta \sqsubseteq \delta'$ (in
$\clkSem\Delta$) if for all $\kappa \in \Delta$, $\delta(\kappa) \leq
\delta'(\kappa)$. The intended interpretation of some $\delta \in
\clkSem\Delta$ is that $\delta$ maps each clock $\kappa$ in $\Delta$
to a natural number stating how much time that clock has left to
run. The ordering $\delta \sqsubseteq \delta'$ indicates that the
clocks in $\delta$ have at most as much time left as the clocks in
$\delta'$. We use the notation $\delta[\kappa \mapsto n]$ to denote
the clock environment mapping $\kappa$ to $n$ and $\kappa'$ to
$\delta(\kappa')$ when $\kappa \not= \kappa'$.

\subsection{Semantic types}
\label{sec:semantic-types}

We interpret types as $\clkSem\Delta$-indexed families of partial
equivalence relations (PERs) over the semantic domain $D$. Recall that
a PER on $D$ is a binary relation on $D$ that is symmetric and
transitive, but not necessarily reflexive. Since PERs are binary
relations, we will treat them as special subsets of the cartesian
product $D \times D$. We write $\PER(D)$ for the set of all partial
equivalence relations on $D$, and $\top_D$ for the PER $D \times D$.
We require that our $\clkSem\Delta$-indexed families of PERs
contravariantly respect the partial ordering on elements of
$\clkSem\Delta$. This ensures that, as the time left to run on clocks
increases, the semantic type becomes ever more precise. When we
interpret universal clock quantification as intersection over all
approximations, we capture the common core of all the possible
approximations.

Formally, a semantic type for a clock context $\Delta$ is a function
$A : \clkSem\Delta \to \PER(D)$, satisfying contravariant Kripke
monotonicity: for all $\delta' \sqsubseteq \delta$, $A\delta \subseteq
A\delta'$. We write $\ClkPER(\Delta)$ for the collection of all
semantic types for the clock context $\Delta$. In
\autoref{sec:fixpoint-types}, we will formally consider morphisms
between semantic types, and so turn each $\ClkPER(\Delta)$ into a
category. Note that we do \emph{not} require that any of our PERs are
admissible. Admissible PERs always include the pair $(\bot,\bot)$,
precisely the values we wish to exclude.

We now define semantic counterparts for each of the syntactic type
constructors we presented in \autoref{sec:types}.

\paragraph{Unit, Product, Coproduct and Function Types}

The constructions for unit, product and coproduct types are
straightforward. The unit type will be interpreted by a constant
family of PERs:
\begin{displaymath}
  1\delta = \{(\semCons{Unit}, \semCons{Unit})\}
\end{displaymath}
This trivially satisfies Kripke monotonicity.

Given semantic types $A$ and $B$, their product is defined to be
\begin{displaymath}
  (A \times B)\delta =
  \left\{
    \begin{array}{l}
      (\semCons{Pair}(x,y),\semCons{Pair}(x',y')) \\
      \quad\quad\sepbar (x,x') \in \sem{A}\delta \land (y,y') \in \sem{B}\delta 
    \end{array}
  \right\}
\end{displaymath}
and their coproduct is
\begin{displaymath}
  (A + B)\delta =
  \begin{array}{l}
    \{ (\semCons{Inl}(x), \semCons{Inl}(x')) \mathrel| (x,x') \in \sem{A}\delta \} \\
    \quad \cup \\
    \{ (\semCons{Inr}(y),\semCons{Inr}(y')) \mathrel| (y,y') \in \sem{B}\delta \}
  \end{array}
\end{displaymath}
Since $A$ and $B$ are assumed to be semantic types, it immediately
follows that $A \times B$ and $A + B$ are semantic types.

To interpret function types, we use the usual definition for
Kripke-indexed logical relations, by quantifing over all smaller clock
environments. Given semantic types $A$ and $B$, we define:
\begin{displaymath}
  (A \to B)\delta = \left\{
    \begin{array}{l}
      (\semCons{Lam}(f),\semCons{Lam}(f')) \\
      \quad \sepbar \forall \delta' \sqsubseteq \delta, (x,x') \in A\delta'.\ (fx,f'x') \in B\delta'
    \end{array}
  \right\}
\end{displaymath}
It follows by the standard argument for Kripke logical relations that
this family of PERs is contravariant in clock environments.

\paragraph{Guarded Modality} Given a semantic type $A$, we define the
semantic guard modality $\delay\kappa$ as follows, where $\kappa$ is a
member of the clock context $\Delta$:
\begin{displaymath}
  (\delay\kappa A)\delta = \left\{
    \begin{array}{ll}
      \top_D & \textrm{if }\delta(\kappa) = 0 \\
      A(\delta[\kappa \mapsto n]) & \textrm{if }\delta(\kappa) = n + 1
    \end{array}
  \right.
\end{displaymath}
Thus, the semantic type operator $\delay\kappa$ acts differently
depending on the time remaining on the clock $\kappa$ in the current
clock environment $\delta$. When the clock has run to zero,
$\delay\kappa A$ becomes completely uninformative, equating all pairs
of elements in the semantic domain $D$. If there is time remaining,
then $\delay\kappa A$ equates a pair iff $A$ would with one fewer
steps remaining. For Kripke monotonicity, we want to prove that
$(d,d') \in (\delay\kappa A)\delta$ implies $(d,d') \in (\delay\kappa
A)\delta'$ when $\delta' \sqsubseteq \delta$. If $\delta'(\kappa) = 0$
then $(\delay\kappa A)\delta' = D \times D \ni (d,d')$. If
$\delta'(\kappa) = n' + 1$ then $\delta(\kappa) = n + 1$ with $n' \leq
n$. So $(d,d') \in A(\delta[\kappa \mapsto n])$, and so by the Kripke
monotonicity for $A$, $(d,d') \in A(\delta'[\kappa \mapsto n']) =
(\delay\kappa A)\delta'$.

\paragraph{Clock Quantification}
The semantic counterpart of clock quantification takes us from
semantic types in $\ClkPER(\Delta, \kappa)$ to semantic types in
$\ClkPER(\Delta)$. Given a semantic type $A$ in $\ClkPER(\Delta,
\kappa)$, we define
\begin{displaymath}
  \forall_\kappa A = \bigcap_{n \in \mathbb{N}} A(\delta[\kappa \mapsto n])
\end{displaymath}
For Kripke monotonicity, if $(d,d') \in (\forall\kappa. A)\delta$ then
$\forall n.\ (d,d') \in A(\delta[\kappa \mapsto n])$. Since $A$ is a
semantic type, $\forall n.\ (d,d') \in A(\delta'[\kappa \mapsto n])$,
hence $(d,d') \in (\forall\kappa. A)\delta'$.

\paragraph{Complete Lattice Structure} In order to define the semantic
counterpart of the least fixpoint type operator, we make use of the
lattice structure of $\ClkPER(\Delta)$. Given $A, B \in
\ClkPER(\Delta)$, we define a partial order: $A \subseteq B$ if
$\forall \delta. A\delta \subseteq B\delta$. It is easy to see that
$\ClkPER(\Delta)$ is closed under arbitrary intersections, and so is a
complete lattice.

Each of the semantic type constructors above is monotonic with respect
to the partial order on $\ClkPER(\Delta)$ (with the obvious proviso
that $A \to B$ is only monotonic in its second argument).

\paragraph{Least Fixpoint Types}
We make use of the Knaster-Tarski theorem \cite{tarski55lattice} to
produce the least fixpoint of a monotone function on
$\ClkPER(\Delta)$. See Loader \cite{loader97equational} for an similar
usage in a setting without guarded recursion. Given a monotone
function $F : \ClkPER(\Delta) \to \ClkPER(\Delta)$, we define:
\begin{displaymath}
  (\mu F) = \bigcap \{ A \in \ClkPER(\Delta) \sepbar FA \subseteq A \}
\end{displaymath}
For any monotone $F$, $\mu F$ is immediately a semantic type by
construction, since semantic types are closed under intersection. As
an initial step towards semantic type soundness for our calculus, we
demonstrate a semantic well-typedness result for the
$\mathrm{primrec}$ functon we defined in
\autoref{sec:semantics-of-programs}.

\begin{lemma}\label{lem:primrec-well-typed}
  Let $F : \ClkPER(\Delta) \to \ClkPER(\Delta)$ be a monotone
  function.  Let $\mathrm{fmap} : D \to D \to D$ be such that for all
  $\delta \in \sem{\Delta}$, for all $A,B \in \ClkPER(\Delta)$, and
  for all $(g,g') \in (A \to B)\delta$ and $(x,x') \in FA\delta$, we
  have $(\mathrm{fmap}\ g\ x, \mathrm{fmap}\ g'\ x') \in FB\delta$.
  Then, for all $\delta \in \sem{\Delta}$ and for all $(f,f') \in
  (F(\mu F \times C) \to C)\delta$, where $C$ is a semantic type, we
  have
  \begin{displaymath}
    (\mathrm{primrec}\ \mathrm{fmap}\ f, \mathrm{primrec}\ \mathrm{fmap}\ f') \in (\mu F \to C)\delta
  \end{displaymath}
\end{lemma}

\begin{proof}
  (Sketch) By induction on the least fixpoint $F(\mu F) = \mu F$,
  using the Knaster-Tarski theorem.
\end{proof}

This lemma only states that our semantic $\mathrm{primrec}$ recursion
operator is semantically well-typed. We will show in
\thmref{thm:initial-f-algebra} that $\mathrm{primrec}$ also witnesses
$\mu F$ as the carrier of the initial $F$-algebra, as we claimed in
\autoref{sec:main-results-intro}.

\subsection{Interpretation of syntactic types}\label{sec:type-interp}

\begin{figure}[t]
  \centering
  \begin{eqnarray*}
    \tySem{1}\theta & = & 1 \\
    \tySem{X}\theta & = & \theta(X) \\
    \tySem{A \times B}\theta & = & \tySem{A}\theta \times \tySem{B}\theta \\
    \tySem{A + B}\theta & = & \tySem{A}\theta + \tySem{B}\theta \\
    \tySem{A \to B}\theta & = & \tySem{A}\emptyset \to \tySem{B}\theta \\
    \tySem{\delay\kappa A}\theta & = & \delay\kappa (\tySem{A}\theta) \\
    \tySem{\forall \kappa. A}\theta & = & \forall_\kappa (\tySem{A}(\theta\mathord\uparrow_\kappa)) \\
    \tySem{\mu X. F}\theta & = & \mu (\lambda X. \tySem{F}(\theta, X))
  \end{eqnarray*}
  \caption{Interpretation of well-formed types}
  \label{fig:type-interp}
\end{figure}

A well-formed type $\Delta; \Theta \vdash A : \sortType$ is
interpreted as monotonic function $\tySem{A} :
\ClkPER(\Delta)^{|\Theta|} \to \ClkPER(\Delta)$, where $|\Theta|$
denotes the number of type variables in $\Theta$. The interpretation
is defined in the clauses in \autoref{fig:type-interp}. These clauses
make use of the constructions of semantic types defined in the
previous subsection. In the clause for $\forall \kappa$, we make use
of the ``clock weakening'' operator $-\mathord\uparrow_\kappa$ which
takes a collection of semantic types in $\ClkPER(\Delta)^{|\Theta|}$
to semantic types in $\ClkPER(\Delta,\kappa)^{|\Theta|}$ by pointwise
restriction of clock environments in $\sem{\Delta, \kappa}$ to clock
environments in $\sem{\Delta}$.

The next lemma states that syntactic substitution and semantic
substitution commute. The proof is a straightforward induction on the
well-formed type $A$. Note the side condition that $\kappa \not\in
\mathit{fc}(A)$, matching the side condition on the
\TirName{$\kappa$-App} typing rule.

\begin{lemma}\label{lem:substitution-lemma}
  Assume that $\Delta; \Theta \vdash A : \sortType$, $\kappa, \kappa'
  \in \Delta$ and $\kappa' \not\in \mathit{fc}(A)$. Then for all
  $\delta \in \sem{\Delta[\kappa\mapsto\kappa']}$ and $\theta \in
  \ClkPER(\Delta)^{|\Theta|}$,
  \begin{displaymath}
    \sem{A}\theta (\delta[\kappa \mapsto \delta(\kappa')]) = \sem{A[\kappa\mapsto\kappa']}(\theta[\kappa\mapsto\kappa'])\delta.
  \end{displaymath}
\end{lemma}

The syntactic type equality judgement $\Delta; \Theta \vdash A \equiv
B : \sortType$ that we defined in \autoref{sec:type-equality} is
interpreted as equality of semantic types. The following lemma states
that this interpretation is sound.

\begin{lemma}\label{lem:type-equality}
  If $\Delta; \Theta \vdash A \equiv B : \sortType$ then for all
  $\theta \in \ClkPER(\Delta)^{|\Theta|}$, $\sem{A}\theta =
  \sem{B}\theta$.
\end{lemma}

The next lemma states that we may use the semantic $\mathrm{fmap_F}$
functions defined in \autoref{fig:semantic-fmap} with the
$\mathrm{primrec}$ operator, by showing that $\mathrm{fmap_F}$ always
satisfies the hypotheses of \lemref{lem:primrec-well-typed}.

\begin{lemma}\label{lem:sem-fmap-well-typed}
  For all $\Delta; \Theta \vdash F : \sortType$, the $\mathrm{fmap}_F$
  defined in \autoref{fig:semantic-fmap} are all semantically
  well-typed: For all $\delta \in \sem{\Delta}$, for all $\vec{A,B}
  \in \ClkPER(\Delta)$, and for all $\vec{(g,g') \in (A \to B)\delta}$
  and $(x,x') \in FA\delta$, we have $(\mathrm{fmap_F}\ \vec{g}\ x,
  \mathrm{fmap_F}\ \vec{g'}\ x') \in FB\delta$.
\end{lemma}

It may seem that could prove this lemma by using the syntactic
definition of $\ident{fmap}$ from \autoref{fig:fmap-syn} and then
applying \thmref{thm:semantic-soundness}. However, the semantic
well-typedness of our definition of the interpretation of
$\elim{primRec}$ depends on the semantic well-typedness of
$\mathrm{fmap}$, so we must prove this lemma directly in the model to
break the circularity.

\subsection{Semantic type soundness}
\label{sec:semantic-soundness}

We now state our first main result: the semantic type soundness
property of our system. To state this result, we define the semantic
interpretation of contexts as a clock-environment indexed collection
of PERs over environments:
\begin{displaymath}
  \sem{\Gamma}\delta = \{(\eta, \eta') \mathrel| \forall (\ident{x} : A) \in \Gamma.\ (\eta(\ident{x}), \eta'(\ident{x})) \in \sem{A}\delta \}
\end{displaymath}

\begin{theorem}\label{thm:semantic-soundness}
  If $\Delta; \Gamma \vdash e : A$, then for all $\delta \in
  \sem{\Delta}$ and $(\eta,\eta') \in \sem{\Gamma}\delta$,
  $(\sem{e}\eta,\sem{e}\eta') \in \sem{A}\delta$.
\end{theorem}

\begin{proof}
  (Sketch) By induction on the derivation of $\Delta; \Gamma \vdash e
  : A$. The most interesting cases are for $\kw{fix}$ and
  $\elim{primRec}$. In essence, the case for $\kw{fix}$ goes through
  by induction on the value of the counter assigned to the clock
  variable $\kappa$ in the current clock environment. The case for
  $\elim{primRec}$ is given by \lemref{lem:primrec-well-typed} and
  \lemref{lem:sem-fmap-well-typed}.
\end{proof}

This theorem has the following corollary, stating that the denotation
of a closed program is never $\bot$: well-typed programs always yield
a proper value.
\begin{corollary}
  If $-; - \vdash e : A$ then for all $\eta$, $\sem{e}\eta \not=
  \bot$.
\end{corollary}
When the result type $A$ is the stream type $\forall \kappa. \mu X. B
\times \delay\kappa X$, we can deduce that the denotation of $e$ will
always be infinite streams of elements of $B$. We further elaborate
this point in the next section, showing that this type is the carrier
of the final $(B \times -)$-coalgebra.

%%% Local Variables:
%%% TeX-master: "paper"
%%% End:
