\section{A type system with clocks and guards}
\label{sec:type-system}

In the introduction, we presented our motivating examples in terms of
an Haskell-like language informally extended with a Nakano-style guard
modality and clock quantification. To be able to formally state the
properties of our combination of clock quantification, clock-indexed
guard modalities and inductive types, in this section we formally
define an extension of the simply-typed $\lambda$-calculus with these
features. In the next section, we define a semantic model for our
system in which we can formally state the results that we claimed in
the introduction.

% FIXME: rewrite this introductory paragraph to flow better

% In this section we present the type system with guarded recursion and
% clock variables that we presented informally in the
% \hyperref[sec:introduction]{introduction}. This type system
% incorporates a general notion of strictly positive inductive types,
% including nested fixpoints, and a guardedness type operator which
% allows for the expression of condinductive types. In
% \autoref{sec:examples} we present several examples of the use of our
% system, including presentations of the informal examples from the
% \hyperref[sec:introduction]{introduction}. In \autoref{sec:semantics}
% we provide a semantics for this system in terms of an untyped
% domain-theoretic model that establishes the soundness properties of
% our typing discipline.

% FIXME: justify switch from :: to : for typing

% FIXME: highlight forwards refs
% FIXME: stop this intro being so formal, continue narrative
% FIXME: highlight the interesting bits

\subsection{Well-formed types with clock variables}
\label{sec:types}

\begin{figure}[t]
  \centering
  \begin{mathpar}
    \inferrule*
    {X \in \Theta}
    {\Delta; \Theta \vdash X : \sortType}
    
    \inferrule*
    { }
    {\Delta; \Theta \vdash 1 : \sortType}
    
    \inferrule*
    {\Delta; \Theta \vdash A : \sortType \\ \Delta; \Theta \vdash B : \sortType}
    {\Delta; \Theta \vdash A \times B : \sortType}
    
    \inferrule*
    {\Delta; \Theta \vdash A : \sortType \\ \Delta; \Theta \vdash B : \sortType}
    {\Delta; \Theta \vdash A + B : \sortType}
    
    \inferrule*
    {\Delta; - \vdash A : \sortType \\ \Delta; \Theta \vdash B : \sortType}
    {\Delta; \Theta \vdash A \to B : \sortType}
    
    \inferrule*
    {\Delta; \Theta, X \vdash A : \sortType}
    {\Delta; \Theta \vdash \mu X. A : \sortType}

    \inferrule*
    {\Delta, \kappa; \Theta \vdash A : \sortType}
    {\Delta; \Theta \vdash \forall \kappa. A : \sortType}

    \inferrule*
    {\Delta; \Theta \vdash A : \sortType \\ \kappa \in \Delta}
    {\Delta; \Theta \vdash \delay\kappa A : \sortType}
  \end{mathpar}
  \caption{Well-formed types and type operators}
  \label{fig:types}
\end{figure}

The types of our system include two kinds of variable that may occur
free and bound: clock variables in the clock quantifier and the
clock-indexed guard modality, as well as type variables occuring in
strictly positive positions for the inductive types. We therefore
formally define when a type is well-formed with respect to contexts of
clock and type variables, using the rules displayed in
\autoref{fig:types}.

The well-formedness judgement for types ($\Delta; \Theta \vdash A :
\sortType$) is defined with respect to a \emph{clock context} $\Delta$
and a \emph{type variable context} $\Theta$. Clock contexts are lists
of clock variables, $\Delta = \kappa_1, ..., \kappa_n$, and type
variable contexts are lists of type variable names, $\Theta = X_1,
..., X_n$. Type variables are only used to manage the scope of the
nested least fixpoint type operator $\mu$; there is no type
polymorphism in the type system we present here. We generally use
roman letters $A$, $B$, $C$ to stand for types, but we also
occasionally use $F$ and $G$ when we wish to emphasise that types with
free type variables are strictly positive type operators. For any type
$A$, we define $\mathit{fc}(A)$ to be the set of free clock variables
that appear in $A$.

Most of the type well-formedness rules are standard: we have a rule
for forming a type from a type variable that is in scope and rules for
forming the unit type $1$, product types $A \times B$, sum types $A +
B$, function types $A \to B$ and least fixpoint types $\mu X. A$. We
enforce the restriction to strictly positive type operators by
disallowing occurences of free type variables in the domain component
of function types. In \autoref{sec:typed-terms}, we will see that each
of the standard type constructors will have the usual corresponding
term-level constructs; our system subsumes the simply typed
$\lambda$-calculus with products, sums and least fixpoint types.

The remaining two rules are for forming clock quantification $\forall
\kappa.A$ types and the clock-indexed Nakano-style guard modality
$\delay\kappa A$. The type formation rule for clock quantification is
very similar to universal type quantification from standard
polymorphic type theories like System F. 

% However, clock quantification
% has a very different flavour to type quantification. The type equality
% rules in \autoref{sec:type-equality} demonstrate that clock
% quantification is ``computationally irrelevant'' and so commutes with
% almost all type operators. Moreover, clock instantiation must be
% linear in order to prevent multiple ``time-streams'' becoming
% conflated (see the statement of the clock substitution lemma below
% \fixme{more detail}).

As we described in the introduction in \autoref{sec:clock-vars}, we
have augmented the Nakano-style delay modality with a clock
variable. This allows us to distinguish between guardedness with
respect to multiple clocks, and hence to be able to close over all the
steps guarded by a particular clock.

\subsection{Type Equality}
\label{sec:type-equality}

The examples we presented in the introduction made use of several type
equalities allowing us to move clock quantification through types. For
instance, in the definition of the $\elim{deStreamCons}$ stream
deconstructor in \autoref{sec:clock-vars}, we made use of a type
equality to treat a value of type $\forall \kappa. \tyname{Integer}$
as having the type $\tyname{Integer}$. Intuitively, this is valid
because a value of type $\tyname{Integer}$ exists independently of any
clock, therefore we can remove the clock quantification. In general,
clock quantification commutes with almost all type formers, except for
guards indexed by the same clock variable. \fixme{Why not force? same reason as for not using type equalities for $\mu$}

The following rules are intended to be read as defining a judgement
$\Delta; \Theta \vdash A \equiv B : \sortType$, indicating that
the well-formed types $\Delta; \Theta \vdash A : \sortType$ and
$\Delta; \Theta \vdash B : \sortType$ are to be considered equal. In
addition to these rules, we also assume that the relation $\Delta
\vdash A \equiv B : \sortType$ is reflexive, symmetric, transitive and
a congruence.
\begin{displaymath}
  \begin{array}{r@{\hspace{0.3em}}c@{\hspace{0.3em}}ll}
    \forall \kappa. A & \equiv & A & (\kappa \not\in \mathit{fc}(A)) \\
    \forall \kappa. A + B & \equiv & (\forall \kappa. A) + (\forall \kappa. B) \\
    \forall \kappa.  A \times B & \equiv & (\forall \kappa. A) \times (\forall \kappa. B) \\
    \forall \kappa. A \to B & \equiv & A \to \forall \kappa. B & (\kappa \not\in \mathit{fc}(A)) \\
    \forall \kappa. \forall \kappa'. A & \equiv & \forall \kappa'. \forall \kappa. A & (\kappa \not= \kappa') \\
    \forall \kappa. \delay\kappa' A & \equiv & \delay\kappa'\forall \kappa. A & (\kappa \not= \kappa') \\
    \forall \kappa. \mu X. F[A,X] & \equiv & \mu X. F[\forall \kappa. A,X] & (\mathit{fc}(F) = \emptyset)
  \end{array}
\end{displaymath}
In the last rule, $F$ must be strictly positive in $A$ as well as $X$.

After we define the terms of our system in the next sub-section, it
will become apparent that several of the type equality rules could be
removed, and their use replaced by terms witnessing the conversions
back and forth. In the case of product types, we could define the
following pair of terms (using the syntax and typing rules we define
below):
\begin{displaymath}
  \begin{array}{l}
    \lambda x. (\Lambda \kappa. \mathbf{fst}~(x[\kappa]), \Lambda \kappa. \kw{snd}~(x[\kappa])) \\
    \hspace{3cm} : (\forall \kappa. A \times B) \to (\forall \kappa. A) \times (\forall \kappa. B)
  \end{array}
\end{displaymath}
and
\begin{displaymath}
  \begin{array}{l}
    \lambda x. \Lambda \kappa. (\kw{fst}~x~[\kappa], \kw{snd}~x~[\kappa]) \\
    \hspace{3cm} : (\forall \kappa. A) \times (\forall \kappa. B) \to \forall \kappa. A \times B
  \end{array}
\end{displaymath}
Indeed, the denotational semantics we present in
\autoref{sec:semantics} will interpret both of these functions as the
identity. However, not all the type equality rules are expressible in
terms of clock abstraction and application, for instance, the $\forall
\kappa. A \equiv A$ rule and the rule for sum types. Moreover, our
definition of $\elim{deCons}$ in \autoref{sec:main-results-intro} is
greatly simplified by having clock quantification commute with all
type formers uniformly (recall that we made crucial use of the
equality $\forall \kappa. F[-] \equiv F[\forall
\kappa. -]$). Therefore, we elect to include type equalities for
commuting clock quantification with all type formers uniformly.

\subsection{Well-typed Terms}\label{sec:typed-terms}

\begin{figure*}[t]
  \centering
  \textbf{1. Simply-typed $\lambda$-calculus with products and sums}
  \begin{mathpar}
    \inferrule* [right=Var]
    {\ident{x} : A \in \Gamma}
    {\Delta; \Gamma \vdash \ident{x} : A}

    \inferrule* [right=Unit]
    { }
    {\Delta; \Gamma \vdash * : 1}

    \inferrule* [right=Abs]
    {\Delta; \Gamma, \ident{x} : A \vdash e : B}
    {\Delta; \Gamma \vdash \lambda \ident{x}.\ e : A \to B}

    \inferrule* [right=App]
    {\Delta; \Gamma \vdash f : A \to B \\
      \Delta; \Gamma \vdash e : A}
    {\Delta; \Gamma \vdash f e : B}

    \inferrule* [right=Pair]
    {\Delta; \Gamma \vdash e_1 : A \\
      \Delta; \Gamma \vdash e_2 : B}
    {\Delta; \Gamma \vdash (e_1, e_2) : A \times B}

    \inferrule* [right=Fst]
    {\Delta; \Gamma \vdash e : A \times B}
    {\Delta; \Gamma \vdash \kw{fst}\ e : A}

    \inferrule* [right=Snd]
    {\Delta; \Gamma \vdash e : A \times B}
    {\Delta; \Gamma \vdash \kw{snd}\ e : B}

    \inferrule* [right=Inl]
    {\Delta; \Gamma \vdash e : A}
    {\Delta; \Gamma \vdash \kw{inl}\ e : A + B}

    \inferrule* [right=Inr]
    {\Delta; \Gamma \vdash e : B}
    {\Delta; \Gamma \vdash \kw{inr}\ e : A + B}

    \inferrule* [right=Case]
    {\Delta; \Gamma \vdash e : A + B \\
      \Delta; \Gamma, \ident{x} : A \vdash f : C \\
      \Delta; \Gamma, \ident{y} : B \vdash g : C}
    {\Delta; \Gamma \vdash
      \kw{case}\ e\ \kw{of}\ \kw{inl}\ \ident{x}. f;
      \kw{inr}\ \ident{y}. g : C}
  \end{mathpar}

  \bigskip
  
  \textbf{2. Least Fixpoint Types}
  \begin{mathpar}
    \inferrule* [right=Cons]
    {\Delta; \Gamma \vdash e : F[\mu X. F]}
    {\Delta; \Gamma \vdash \cons{Cons$_F$}\ e : \mu X. F}

    \inferrule* [right=PrimRec]
    {\Delta; \Gamma \vdash e : F[(\mu X. F) \times A] \to A}
    {\Delta; \Gamma \vdash \elim{primRec$_F$}\ e : \mu X. F \to A}
  \end{mathpar}

  \bigskip

  \textbf{3. Clock Abstraction and Application, and Type Equality}
  \begin{mathpar}
    \inferrule* [right=$\kappa$-Abs]
    {\Delta, \kappa; \Gamma \vdash e : A \\ \kappa \not\in \mathit{fv}(\Gamma)}
    {\Delta; \Gamma \vdash \Lambda \kappa. e : \forall \kappa. A}

    \inferrule* [right=$\kappa$-App]
    {\Delta; \Gamma \vdash e : \forall \kappa. A \\ \kappa' \in \Delta \\ \kappa' \not\in \mathit{fc}(A)}
    {\Delta; \Gamma \vdash e[\kappa'] : A[\kappa \mapsto \kappa']}
      
    \inferrule* [right=TyEq]
    {\Delta; \Gamma \vdash e : A \\ \Delta \vdash A \equiv B}
    {\Delta; \Gamma \vdash e : B}
  \end{mathpar}

  \bigskip

  \textbf{4. Applicative Functor Structure for $\delay\kappa-$}
  \begin{mathpar}
    \inferrule* [right=DePure]
    {\Delta; \Gamma \vdash e : A \\ \kappa \in \Delta}
    {\Delta; \Gamma \vdash \kw{pure}\ e : \delay\kappa A}
    
    \inferrule* [right=DeApp]
    {\Delta; \Gamma \vdash f : \delay\kappa (A \to B) \\
      \Delta; \Gamma \vdash e : \delay\kappa A}
    {\Delta; \Gamma \vdash f \circledast e : \delay\kappa B}
  \end{mathpar}

  \bigskip
  \textbf{5. Fix and Force}
  \begin{mathpar}
    \inferrule* [right=Fix]
    {\Delta; \Gamma \vdash f : \delay\kappa A \to A}
    {\Delta; \Gamma \vdash \kw{fix}\ f : A}

    \inferrule* [right=Force]
    {\Delta; \Gamma \vdash e : \forall \kappa. \delay\kappa A}
    {\Delta; \Gamma \vdash \kw{force}\ e : \forall \kappa. A}
  \end{mathpar}
  \caption{Well-typed terms}
  \label{fig:terms}
\end{figure*}

The well-typed terms of our system are defined by the typing judgement
$\Delta; \Gamma \vdash e : A$, given by the rules presented in
\autoref{fig:terms}. Terms are judged to be well-typed with respect to
a clock variable context $\Delta$, and a typing context $\Gamma$
consisting of a list of variable : type pairs: $x_1 : A_1, ..., x_n :
A_n$. We only consider typing contexts where each type is well-formed
with respect to the clock variable context $\Delta$, and the
\emph{empty} type variable context. Similarly, the result type $A$ in
the typing judgement must be well-formed with respect to $\Delta$ and
the empty type variable context.

We have split the typing rules in \autoref{fig:terms} into five
groups. The first group includes the standard typing rules for the
simply-typed $\lambda$-calculus with unit, product and sum
types. Apart from the additional clock variable contexts $\Delta$,
these rules are entirely standard. The second group contains the rules
for the inductive types $\mu X. F$. Again, apart from the appearance
of clock variable contexts, these rules are the standard constructor
and primitive recursion constructs for inductive types.

The third group of rules cover clock abstraction and application, and
the integration of the type equality rules we defined in the previous
section. In the \TirName{$\kappa$-App} rule, we have used the notation
$A[\kappa \mapsto \kappa']$ to indicate the substitution of the clock
variable $\kappa'$ for $\kappa$. These rules are as one might expect
for System F-style quantification with non-trivial type equality,
albeit that in the \TirName{$\kappa$-App} rule the $\kappa'$ clock
variable cannot already appear in the type. This disallows us from
using the same clock variable twice, preventing multiple
``time-streams'' becoming conflated (see the statement of the clock
substitution lemma below \fixme{more detail}).

The fourth group of rules state that the clock-indexed guard modality
supports the $\kw{pure}$ and apply $(\mathord\circledast)$ operations
of an applicative functor. When we define a denotational semantics of
our calculus in the next section, these operations will be interpreted
simply as the identity function and normal function application
respectively, meaning that the applicative functor laws hold
trivially.

Finally, the fifth group of rules presents the typing for Nakano's
guarded $\kw{fix}$ combinator, now indexed by a clock variable, and
our $\kw{force}$ combinator for removing the guard modality when it is
protected by a clock quantification.

\subsection{Type operators are functorial}
\label{sec:functorial}

Define $\ident{fmap}_F$ for all $F$.

\subsection{Examples}
\label{sec:examples}

\paragraph{Lists and Trees}
$\mu X. 1 + A \times X$

$\mu X. 1 + A \times A \times X$.


\paragraph{Streams}
$\forall \kappa. \mu X. A \times \delay\kappa X$.

\paragraph{Stream Eating}

Example type: $\forall \kappa. \mu X. \mu Y. (I \to Y) + (O \times
\delay\kappa X)$. Stream eating functions that consume finite amounts
of input for every output. Should be able to define the application
and composition operations.

\paragraph{The Partiality Monad}

$T A = \forall \kappa. \mu X. A + \delay\kappa X$

Escardo?

\paragraph{Circular Programming} Refer back to the replaceMin example.

%%% Local Variables:
%%% TeX-master: "paper"
%%% End:
