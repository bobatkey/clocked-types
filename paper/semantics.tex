\section{Semantics}\label{sec:semantics}

We now present a denotational semantics for the type system we
presented in the \hyperref[sec:type-system]{previous section}. This
semantics establishes the key soundness properties: avoidance of
non-terminating functions and correct representation of final
coalgebras.

In \autoref{sec:semantics-of-programs} we define an untyped model of
the calculus we defined in the previous section. Each term of the
calculus is interpreted in a domain-theoretic model of the untyped
lazy $\lambda$-calculus. A highlight of this interpretation is that
the $\kw{fix}$ operator from our calculus is interpreted directly as
the domain-theoretic fixpoint in the model. In
\autoref{sec:clock-envs} we start to define the semantics of types, by
first defining the semantics of the clock contexts $\Delta$. Clock
contexts are interpreted as clock environments: collections of
step-indexing counters, counting down. Then in
\autoref{sec:semantic-types} we define semantic types as special kinds
of partial equivalence relations (PERs) over the untyped
domain-theoretic model, indexed by a clock environments. In
\autoref{sec:type-interp}, we show how to interpret each of the
syntactic types in our calculus as semantic types, and in
\autoref{sec:semantic-soundness} we show our first main result: that
every well-typed term is semantically well-typed. Finally, in
\autoref{sec:fixpoint-types}, we state and prove the initiality and
finality of the various definitions of fixpoint types that we stated
in \autoref{sec:main-results-intro}.

\subsection{Semantics of Programs}\label{sec:semantics-of-programs}

We interpret terms of our programming language in (a mild extension
of) the standard domain theoretic model of the untyped lazy
$\lambda$-calculus \cite{PittsAM:compavm}. The key feature of this
semantics is the erasure of anything to do with the guardedness type
constructor $\delay\kappa-$ and the clock variables and
quantification. The $\kw{fix}$ operator is ``implemented'' directly as
the domain-theoretic fixpoint. This semantics can be seen as a
compilation process from our typed, guarded and clocked calculus into
the untyped lazy $\lambda$-calculus. The semantic type soundness
property that we prove in \autoref{sec:semantic-soundness} shows that
our typing discipline ensures that we only compile to productive
programs.

We assume a directed complete partial order with bottom
($\mathrm{DCPO}_\bot$), $D$, satisfying the following recursive domain
equation:
\begin{displaymath}
  D \cong (D \to D)_\bot \oplus (D \times D)_\bot \oplus D_\bot \oplus D_\bot \oplus 1_\bot
\end{displaymath}
where $\oplus$ represents the coalesced sum that identifies the $\bot$
element of all the components. We are guaranteed to be able to obtain
such a $D$ by standard results about the category $\mathrm{DCPO}_\bot$
\cite{smythplotkin}. The five components of the right hand side of
this equation will be used to carry functions, products, the left and
right injections for sum types and the unit value
respectively. Accordingly, we use the following symbols to represent
the matching continuous injective maps into $D$:
\begin{center}
  \begin{tabular}{ll}
    $\semCons{Lam} : [D \to D] \to D$ & $\semCons{Pair} : D \times D \to D$ \\
    $\semCons{Inl} : D \to D$ & $\semCons{Inr} : D \to D$ \\
    $\semCons{Unit} : 1 \to D$
  \end{tabular}  
\end{center}
We will write $\semCons{Unit}$ instead of $\semCons{Unit}\ *$.

% FIXME: brief recap of DCPPO, in particular the fixpoint operator

We assume that the set of all possible identifier names is represented
by $V$. Environments $\eta$ are modelled as maps from $V$ to elements
of $D$. These are exactly as one would expect for an untyped
interpretation of the simply-typed $\lambda$-calculus. It can be
easily checked that this collection of equations defines a continuous
function $D^V \to D$, since everything is built from standard parts.
\begin{eqnarray*}
  \sem{x}\eta & = & \eta(x) \\
  \sem{*}\eta & = & \semCons{Unit} \\
  \sem{\lambda x.\ e}\eta & = & \semCons{Lam}\ (\lambda v.\ \sem{e}(\eta[x \mapsto v])) \\
  \sem{fe}\eta & = & \left\{
    \begin{array}{ll}
      d_f\ (\sem{e}\eta) & \textrm{if}\ \sem{f}\eta = \semCons{Lam}\ d_f \\
      \bot & \textrm{otherwise}
    \end{array}
    \right. \\
  \sem{(e_1,e_2)}\eta & = & \semCons{Pair}\ (\sem{e_1}\eta, \sem{e_2}\eta) \\
  \sem{\kw{fst}\ e}\eta & = & \left\{
    \begin{array}{ll}
      d_1 & \textrm{if}\ \sem{e}\eta = \semCons{Pair}\ (d_1, d_2) \\
      \bot & \textrm{otherwise}
    \end{array}
  \right. \\
  \sem{\kw{snd}\ e}\eta & = & \left\{
    \begin{array}{ll}
      d_2 & \textrm{if}\ \sem{e}\eta = \semCons{Pair}\ (d_1, d_2) \\
      \bot & \textrm{otherwise}
    \end{array}
  \right. \\
  \sem{\kw{inl}\ e}\eta & = & \semCons{Inl}\ (\sem{e}\eta) \\
  \sem{\kw{inr}\ e}\eta & = & \semCons{Inr}\ (\sem{e}\eta) \\
  \left\llbracket
    \begin{array}{l}
      \kw{case}\ e\ \kw{of}\\
      \quad\kw{inl}\ \ident{x}.f;\\
      \quad\kw{inr}\ \ident{y}.g
    \end{array}
  \right\rrbracket\eta & = &
  \left\{
    \begin{array}{ll}
      \sem{f}(\eta[x \mapsto d]) & \textrm{if }\sem{e}\eta = \semCons{Inl}\ d \\
      \sem{g}(\eta[y \mapsto d]) & \textrm{if }\sem{e}\eta = \semCons{Inr}\ d \\
      \bot & \textrm{otherwise}
    \end{array}
  \right. \\
\end{eqnarray*}
The delay and force operators are just no-ops in this interpretation,
and applicative apply is just normal function application.
\begin{eqnarray*}
  \sem{\kw{delay}\ e}\eta & = & \sem{e}\eta \\
  \sem{f \circledast e}\eta & = & \left\{
    \begin{array}{ll}
      d_f\ (\sem{e}\eta) & \textrm{if}\ \sem{f}\eta = \semCons{Lam}\ d_f \\
      \bot & \textrm{otherwise}
    \end{array}
  \right. \\
  \sem{\kw{force}\ e}\eta & = & \sem{e}\eta
\end{eqnarray*}
Our interpretation simply translates the fixpoint combinator of the
calculus as the usual fixpoint operator $\mathrm{fix}\ f = \bigsqcup_n
f^n(\bot)$ in $\mathrm{DCPO}_\bot$.
\begin{eqnarray*}
  \sem{\kw{fix}\ f}\eta & = &
  \left\{
    \begin{array}{ll}
      \mathrm{fix}\ d_f & \textrm{if }\sem{f}\eta = \semCons{Lam}\ d_f \\
      \bot & \textrm{otherwise}
    \end{array}
  \right.
\end{eqnarray*}

Finally, we interpret the constructor and primitive recursion
eliminator for types of the form $\mu X. F[X]$. The interpretation of
terms of the form $\cons{Cons}\ e$ is straightforward: the constructor
itself disappears at runtime, so the interpretation simply ignores it:
\begin{eqnarray*}
  \sem{\cons{Cons}\ e}\eta & = & \sem{e}\eta
\end{eqnarray*}

To define the semantic counterpart of the $\elim{primRec}$ operator,
we first define a higher-order continuous function $\mathrm{primrec} :
(D \to D \to D) \to (D \to D) \to D$ in the model:
\begin{displaymath}
  \begin{array}{l}
    \mathrm{primrec}\ \mathit{fmap}\ f = \\
    \quad\quad\semCons{Lam}\ (\mathrm{fix}\ (\lambda g x.\ f\ (\mathit{fmap}\ (\semCons{Lam}\ (\lambda x. \semCons{Pair}\ (x, g\ x))))\ x))
  \end{array}
\end{displaymath}
The first argument to $\mathrm{primrec}$ specifies a functorial
mapping function that are generated from a syntactic types. For each
of the syntactic types $\Delta; \Theta \vdash F : \sortType$ defined
in \autoref{fig:types}, we define a suitable $\mathrm{fmap_F} :
D^{|\Theta|} \to D \to D$, where $|\Theta|$ is the number of type
variables in $\Theta$. This definition is presented in
\autoref{fig:semantic-fmap}.

\begin{figure}[t]
  \begin{eqnarray*}
    \mathrm{fmap_X}\ \vec{f}\ x & = & d_f(x)\quad (f_X = \semCons{Lam}\ d_f) \\
    \mathrm{fmap_1}\ \vec{f}\ x & = & \semCons{Unit} \\
    \mathrm{fmap_{F \times G}}\ \vec{f}\ (\semCons{Pair}\ (x,y)) & = & \semCons{Pair}\ (\mathrm{fmap_F}\ \vec{f}\ x, \mathrm{fmap_G}\ \vec{f}\ y) \\
    \mathrm{fmap_{F + G}}\ \vec{f}\ (\semCons{Inl}\ x) & = & \semCons{Inl}\ (\mathrm{fmap_F}\ \vec{f}\ x) \\
    \mathrm{fmap_{F + G}}\ \vec{f}\ (\semCons{Inr}\ x) & = & \semCons{Inr}\ (\mathrm{fmap_G}\ \vec{f}\ x) \\
    \mathrm{fmap_{A \to F}}\ \vec{f}\ (\semCons{Lam}\ g) & = & \semCons{Lam}\ (\lambda v. \mathrm{fmap_F}\ \vec{f}\ (g\ v)) \\
    \mathrm{fmap_{\delay\kappa F}}\ \vec{f}\ x & = & \mathrm{fmap_F}\ \vec{f}\ x \\
    \mathrm{fmap_{\forall\kappa.F}}\ \vec{f}\ x & = & \mathrm{fmap_F}\ \vec{f}\ x \\
    \mathrm{fmap_{\mu X. F}}\ \vec{f}\ x & = &
    \begin{array}[t]{l}
      \mathrm{primrec}\ (\mathrm{fmap_F}\ \vec{f})\ \\
      \quad\quad\quad\quad(\mathrm{fmap}_F\ \vec{f}\ \mathrm{snd})\ x
    \end{array}
  \end{eqnarray*}
  where
  \begin{displaymath}
    \mathrm{snd}\ (\semCons{Pair}\ (x,y)) = y
  \end{displaymath}
  All unhandled cases are defined to be equal to $\bot$.
  \caption{Semantic $\mathrm{fmap_F}$ for $\Delta; \Theta \vdash F : \sortType$}
  \label{fig:semantic-fmap}
\end{figure}

With these definitions we can define the interpretation of the
syntactic $\elim{primRec$_F$}$ operator. Note that the syntactic type
constructor $F$ here always has exactly one free type variable, so
$\mathrm{fmap_F}$ has type $D \to D \to D$, as required by
$\mathrm{primrec}$.
\begin{displaymath}
  \begin{array}{l}
    \sem{\elim{primRec$_F$}\ f}\eta = \\
    \quad\quad\left\{
      \begin{array}{ll}
        \mathrm{primrec}\ \mathrm{fmap_F}\ d_f & \textrm{if }\sem{f}\eta = \semCons{Lam}\ d_f \\
        \bot & \textrm{otherwise}
      \end{array}
    \right.
  \end{array}
\end{displaymath}

This completes the interpretation of the terms of our programming
language in our untyped model. We now turn to the semantic meaning of
types, with the aim of showing, in \autoref{sec:semantic-soundness},
that each well-typed term's interpretation is semantically well-typed.

\subsection{Clock Environments}\label{sec:clock-envs}

For a clock context $\Delta$, we define $\clkSem{\Delta}$ to be the
set of clock environments: mappings from the clock variables in
$\Delta$ to the natural numbers. We use $\delta, \delta'$ etc. to
range over clock environments. For $\delta$ and $\delta'$ in
$\clkSem{\Delta}$, we say that $\delta \sqsubseteq \delta'$ (in
$\clkSem\Delta$) if for all $\kappa \in \Delta$, $\delta(\kappa) \leq
\delta'(\kappa)$. The intended interpretation of some $\delta \in
\clkSem\Delta$ is that $\delta$ maps each clock $\kappa$ in $\Delta$
to a natural number stating how much time that clock has left to
run. The ordering $\delta \sqsubseteq \delta'$ indicates that the
clocks in $\delta$ have at most as much time left as the clocks in
$\delta'$.
% FIXME: syntax for clock environment updates

\subsection{Semantic Types}\label{sec:semantic-types}

We semantically interpret types as $\clkSem\Delta$-indexed families of
partial equivalence relations (PERs) over the semantic domain
$D$. Recall that a PER on $D$ is a binary relation on $D$ that is
symmetric and transitive, but not necessarily reflexive. Since PERs
are binary relations, we will treat them as special subsets of the
cartesian product $D \times D$. We write $\PER(D)$ for the set of all
partial equivalence relations on $D$, and $\top_D$ for the PER $D
\times D$. 
% We are only interested in PERs that are upwards closed
% with respect to the partial order on $D$. Upwards closure ensures that
% if a type includes the initial segment of an infinite value, then it
% includes the remainder of that value. 
We require that our $\clkSem\Delta$-indexed families of PERs
contravariantly respect the partial ordering on elements of
$\clkSem\Delta$. This ensures that, as the time left to run on clocks
increases, the semantic type becomes ever more precise. Thus when we
interpret universal clock quantification as intersection over all
approximations (see below), we capture the common core of all the
approximations we have made. Formally, a semantic type for a clock
context $\Delta$ is a function $A : \clkSem\Delta \to \PER(D)$,
satisfying contravariant Kripke monotonicity: for all $\delta'
\sqsubseteq \delta$, $A\delta \subseteq A\delta'$.

We write $\ClkPER(\Delta)$ for the collection of all semantic types
for the clock context $\Delta$. Below, we will also consider morphisms
between semantic types, and so turn $\ClkPER(\Delta)$ into a
category. %FIXME: reference to the actual section

% FIXME: maybe say that we definitely do not want relations to be
% admissible

\paragraph{Unit, Product, Coproduct and Function Types}

To interpret the types of our programming language, we must give
semantic counterparts to each of the syntactic type constructors we
have used. The basic constructions for unit, product and coproduct
types are straightforward. The unit type will be interpreted by a
constant familiy of PERs:
\begin{displaymath}
  1\delta = \{(\semCons{Unit}, \semCons{Unit})\}
\end{displaymath}
This clearly satisfies the conditions for being a semantic type.

Given semantic types $A$ and $B$, their product is defined to be
\begin{displaymath}
  (A \times B)\delta =
  \left\{
    \begin{array}{l}
      (\semCons{Pair}(x,y),\semCons{Pair}(x',y')) \\
      \quad\quad\sepbar (x,x') \in \sem{A}\delta \land (y,y') \in \sem{B}\delta 
    \end{array}
  \right\}
\end{displaymath}
and their coproduct is
\begin{displaymath}
  (A + B)\delta =
  \begin{array}{l}
    \{ (\semCons{Inl}(x), \semCons{Inl}(x')) \mathrel| (x,x') \in \sem{A}\delta \} \\
    \quad \cup \\
    \{ (\semCons{Inr}(y),\semCons{Inr}(y')) \mathrel| (y,y') \in \sem{B}\delta \}
  \end{array}
\end{displaymath}
Since $A$ and $B$ are assumed to be semantic types, it immediately
follows that $A \times B$ and $A + B$ are semantic types.

To interpret function types, we make use of the typical definition for
Kripke-indexed logical relations, by quantifing over all smaller clock
environments. Given semantic types $A$ and $B$, we define:
\begin{displaymath}
  (A \to B)\delta = \left\{
    \begin{array}{l}
      (\semCons{Lam}(f),\semCons{Lam}(f')) \\
      \quad \sepbar \forall \delta' \sqsubseteq \delta, (x,x') \in A\delta'.\ (fx,f'x') \in B\delta'
    \end{array}
  \right\}
\end{displaymath}
It follows by the standard argument for Kripke logical relations that
this family of PERs is contravariant in clock environments.

% We also
% need to show that if $(\semCons{Lam}(f_1),\semCons{f_2}) \in (A \to
% B)\delta$, and $f_1 \sqsubseteq f'_1$ and $f_2 \sqsubseteq f'_2$, then
% $(\semCons{Lam}(f'_1),\semCons{f'_2}) \in (A \to B)\delta$. This
% follows because, for all $\delta' \sqsubseteq \delta$ and $(x_1,x_2)
% \in A\delta'$ such that $(f_1x_1,f_2x_2) \in B\delta'$, we have
% $f_1x_1 \sqsubseteq f'_1x_1$ and $f_2x_2 \sqsubseteq f'_2x_2$ by the
% definition of order on elements of $D \to D$. Since $B$ is a semantic
% type, and so upwards closed, we now know that $(f'_1x_1,f'_2x_2) \in
% B\delta'$. Thus $(\semCons{Lam}(f'_1),\semCons{Lam}(f'_2)) \in (A \to
% B)\delta$, as required.

\paragraph{Guarded Types} Given a semantic type $A$, we define the
semantic delay operation $\delay\kappa$ as follows, where $\kappa$ is
a member of the clock context $\Delta$:
\begin{displaymath}
  (\delay\kappa A)\delta = \left\{
    \begin{array}{ll}
      \top_D & \textrm{if }\delta(\kappa) = 0 \\
      A(\delta[\kappa \mapsto n]) & \textrm{if }\delta(\kappa) = n + 1
    \end{array}
  \right.
\end{displaymath}
Thus, the semantic type operator $\delay\kappa$ acts differently
depending on the time remaining on the clock $\kappa$ in the current
clock environment $\delta$. When the clock has run to zero,
$\delay\kappa A$ becomes completely uninformative, equating all pairs
of elements in the semantic domain $D$. If there is time remaining,
then $\delay\kappa A$ equates a pair iff $A$ would with one fewer
steps remaining. 
% For upwards-closure, there are two cases: if
% $\delta(\kappa) = 0$ then $D$ is trivially upwards-closed; if
% $\delta(\kappa) = n + 1$, then upwards closure follows from the
% upwards closure of $A$. 
For Kripke monotonicity, we want to prove that
$(d,d') \in (\delay\kappa A)\delta$ implies $(d,d') \in (\delay\kappa
A)\delta'$ when $\delta' \sqsubseteq \delta$. If $\delta'(\kappa) = 0$
then $(\delay\kappa A)\delta' = D \times D \ni (d,d')$. If
$\delta'(\kappa) = n' + 1$ then $\delta(\kappa) = n + 1$ with $n' \leq
n$. So $(d,d') \in A(\delta[\kappa \mapsto n])$, and so by the Kripke
monotonicity for $A$, $(d,d') \in A(\delta'[\kappa \mapsto n']) =
(\delay\kappa A)\delta'$.

\paragraph{Clock Quantification}
The semantic counterpart of clock quantification takes us from
semantic types in $\ClkPER(\Delta, \kappa)$ to semantic types in
$\ClkPER(\Delta)$. Given a semantic type $A$ in $\ClkPER(\Delta,
\kappa)$, we define
\begin{displaymath}
  \forall_\kappa A = \bigcap_{n \in \mathbb{N}} A(\delta[\kappa \mapsto n])
\end{displaymath}
% For upwards closure, $(d_1,d_2) \in (\forall_\kappa. A)\delta$ implies
% that $(d_1,d_2) \in A(\delta[\kappa \mapsto n])$ for all $n$. If $d_1
% \sqsubseteq d'_1$ and $d_2 \sqsubseteq d'_2$, then since $A$ is a
% semantic type, we know that $(d'_1,d'_2) \in A(\delta[\kappa \mapsto
% n])$ for all $n$. Thus $(d'_1,d'_2) \in (\forall_\kappa. A)\delta$, as
% required.
For Kripke monotonicity, if $(d,d') \in
(\forall\kappa. A)\delta$ then $\forall n.\ (d,d') \in A(\delta[\kappa
\mapsto n])$. Since $A$ is a semantic type, $\forall n.\ (d,d') \in
A(\delta'[\kappa \mapsto n])$, hence $(d,d') \in
(\forall\kappa. A)\delta'$.

\paragraph{Complete Lattice Structure} In order to define the semantic
counterpart of the least fixpoint type operator, we make use of the
lattice structure of $\ClkPER(\Delta)$. Given $A, B \in
\ClkPER(\Delta)$, we define a partial order: $A \subseteq B$ if
$\forall \delta. A\delta \subseteq B\delta$. It is easy to see that
$\ClkPER(\Delta)$ is closed under arbitrary intersections, and so is a
complete lattice.

Each of the semantic type constructors above is monotonic with respect
to the partial order on $\ClkPER(\Delta)$ (with the obvious proviso
that $A \to B$ is only monotonic in its second argument).

\paragraph{Least Fixpoint Types}
We make use of the Knaster-Tarski theorem \cite{tarski55lattice} to
produce the least fixpoint of a monotone function on
$\ClkPER(\Delta)$. See Loader \cite{loader97equational} for an similar
usage in a setting without guarded recursion. Given a monotone
function $F : \ClkPER(\Delta) \to \ClkPER(\Delta)$, we define:
\begin{displaymath}
  (\mu F) = \bigcap \{ A \in \ClkPER(\Delta) \sepbar FA \subseteq A \}
\end{displaymath}
For any monotone $F$, $\mu F$ is immediately a semantic type by
construction, since semantic types are closed under intersection. As
an initial step towards semantic type soundness for our calculus, we
demonstrate a semantic well-typedness result for the
$\mathrm{primrec}$ functon we defined in
\autoref{sec:semantics-of-programs}.

\begin{lemma}\label{lem:primrec-well-typed}
  Let $F : \ClkPER(\Delta) \to \ClkPER(\Delta)$ be a monotone
  function.  Let $\mathrm{fmap} : D \to D \to D$ be such that:
  \begin{enumerate}
  \item For all $\delta \in \sem{\Delta}$, for all $A,B \in
    \ClkPER(\Delta)$, and for all $(g,g') \in (A \to B)\delta$ and
    $(x,x') \in FA\delta$, we have $(\mathrm{fmap}\ g\ x,
    \mathrm{fmap}\ g'\ x') \in FB\delta$.
  \end{enumerate}
  Then, for all $\delta \in \sem{\Delta}$ and for all $(f,f') \in
  (F(\mu F \times C) \to C)\delta$, where $C$ is a semantic type, we
  have
  \begin{displaymath}
    (\mathrm{primrec}\ \mathrm{fmap}\ f, \mathrm{primrec}\ \mathrm{fmap}\ f') \in (\mu F \to C)\delta
  \end{displaymath}
\end{lemma}

\subsection{Interpretation of Syntactic Types}\label{sec:type-interp}

\begin{figure}[t]
  \centering
  \begin{eqnarray*}
    \tySem{1}\theta & = & 1 \\
    \tySem{X}\theta & = & \theta_X \\
    \tySem{A \times B}\theta & = & \tySem{A}\theta \times \tySem{B}\theta \\
    \tySem{A + B}\theta & = & \tySem{A}\theta + \tySem{B}\theta \\
    \tySem{A \to B}\theta & = & \tySem{A}\emptyset \to \tySem{B}\theta \\
    \tySem{\delay\kappa A}\theta & = & \delay\kappa (\tySem{A}\theta) \\
    \tySem{\forall \kappa. A}\theta & = & \forall_\kappa (\tySem{A}(\theta\mathord\uparrow_\kappa)) \\
    \tySem{\mu X. F}\theta & = & \mu (\lambda X. \tySem{F}(\theta, X))
  \end{eqnarray*}
  \caption{Interpretation of well-formed types}
  \label{fig:type-interp}
\end{figure}

An open type $\Delta; \Theta \vdash A : \sortType$ is interpreted as
monotonic function $\tySem{A} : \ClkPER(\Delta)^{|\Theta|} \to
\ClkPER(\Delta)$, where $|\Theta|$ denotes the number of type
variables in $\Theta$. The interpretation is defined in the clauses in
\autoref{fig:type-interp}. These clauses make use of the constructions
of semantic types defined in the previous subsection. In the clause
for $\forall \kappa$, we make use of the ``clock weakening'' operator
$-\mathord\uparrow_\kappa$ which takes a collection of semantic types
in $\ClkPER(\Delta)^{|\Theta|}$ to semantic types in
$\ClkPER(\Delta,\kappa)^{|\Theta|}$ by pointwise restriction of clock
environments in $\sem{\Delta, \kappa}$ to clock environments in
$\sem{\Delta}$.

\begin{lemma}\label{lem:substitution-lemma}
  Assume that $\Delta; \Theta \vdash A : \sortType$, $\kappa, \kappa'
  \in \Delta$ and $\kappa' \not\in \mathit{fv}(A)$. Then for all
  $\delta \in \sem{\Delta[\kappa\mapsto\kappa']}$ and $\theta \in
  \ClkPER(\Delta)^{|\Theta|}$,
  \begin{displaymath}
    \sem{A}\theta (\delta[\kappa \mapsto \delta(\kappa')]) = \sem{A[\kappa\mapsto\kappa']}(\theta[\kappa\mapsto\kappa'])\delta.
  \end{displaymath}
\end{lemma}

\begin{lemma}\label{lem:type-equality}
  If $\Delta; \Theta \vdash A \equiv B$ then for all $\theta \in
  \ClkPER(\Delta)^{|\Theta|}$, $\sem{A}\theta = \sem{B}\theta$.
\end{lemma}

The next lemma ensures that we may use the semantic $\mathrm{fmap_F}$
functions defined in \autoref{fig:semantic-fmap} with the
$\mathrm{primrec}$ operator, by showing that each of $\mathrm{fmap_F}$
satisfies the requirements of \lemref{lem:primrec-well-typed}.

\begin{lemma}\label{lem:sem-fmap-well-typed}
  For all $\Delta; \Theta \vdash F : \sortType$, the $\mathrm{fmap}_F$
  defined in \autoref{fig:semantic-fmap} are all semantically
  well-typed: For all $\delta \in \sem{\Delta}$, for all $\vec{A,B}
  \in \ClkPER(\Delta)$, and for all $\vec{(g,g') \in (A \to B)\delta}$
  and $(x,x') \in FA\delta$, we have $(\mathrm{fmap_F}\ \vec{g}\ x,
  \mathrm{fmap_F}\ \vec{g'}\ x') \in FB\delta$.
\end{lemma}

% \begin{proof}
%   By induction on the derivation of $\Delta; \Theta \vdash F :
%   \sortType$. The verifications are very similar to the verifications
%   required for \thmref{thm:semantic-soundness} below, so we omit them.
% \end{proof}

\subsection{Semantic Type Soundness}\label{sec:semantic-soundness}

We now show our first main result: the semantic type soundness
property of our system. To state this result, we must define a
semantic interpretation of contexts as a clock-environment indexed
collection of PERs over mappings from variable names in $V$ to
semantic domain elements in $D$:
\begin{displaymath}
  \begin{array}{l}
    \sem{\Gamma}\delta =\\
    \quad\{(\eta, \eta') \in D^V \times D^V \mathrel| \forall (\ident{x} : A) \in \Gamma.\ (\eta(\ident{x}), \eta'(\ident{x})) \in \sem{A}\delta \}  
  \end{array}
\end{displaymath}

\begin{theorem}\label{thm:semantic-soundness}
  If $\Delta; \Gamma \vdash e : A$, then for all $\delta \in
  \sem{\Delta}$ and $(\eta,\eta') \in \sem{\Gamma}\delta$,
  $(\sem{e}\eta,\sem{e}\eta') \in \sem{A}\delta$.
\end{theorem}

%%% Local Variables:
%%% TeX-master: "paper"
%%% End:
